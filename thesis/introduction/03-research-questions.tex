\subsection{Research Questions}

% \subsubsection{Primary Research Questions}

\subsubsection*{RQ1 — Are natural language database interfaces feasible for real world application?}

The primary research questions when it comes to natural language database interfaces evolve around their
semantic accuracy and reliability, therefore questioning their feasibility for real world usage.
LLMs have notoriously been known for their ability to hallucinate / produce false, but promising outputs.
This behaviour can be especially  dangerous when opting for data driven decisions that rely on false data
due to a mistranslation from natural language to SQL. LLMs could cause hard to understand and debug behaviour,
like false computation of distributions when the intermediate format is not being shown to the user. This
thesis tries to determine whether such hallucinations could be reasonably prevented and whether the associated
performance and hardware requirements are suitable for a real world deployment, outside of research situations.

Specifically the two big underlying questions are:

\begin{enumerate}
    \item Is the semantic accuracy of natural language database interfaces high enough to yield a noticable
          benefit to users?
    \item Is it possible to run such an interface on reasonable, mass available hardware (e.g. excluding high end research GPUs).
\end{enumerate}

\subsubsection*{RQ2 — What approaches are most effective in resolving ambiguity when translating natural language queries into SQL?}

To provide semantically correct results ambiguity in the user-provided natural language queries must be 
adequately addressed. This thesis investigates various approaches to ambiguity management and
resolution. Natural language queries can demonstrate ambiguity even at low levels of complexity —
e.g. there are two different types of "sales" in a database schema, and the user asks to retireve
"all sales".

Such situations present the second major challenge associated with the practical implementation of natural
language database interfaces. The success of this concept will significantly depend on whether suitable 
designs and mitigation techniques can be implemented without creating problems with regards to the 
aforementioned performance and hardware requirements. The research focus lies on both preventative measures
through optimized pre-processing stages and prompt engineering techniques as well as reactive strategies
that post process LLM output, either on the basis of further user input or context inference.

\subsubsection*{RQ3 — Which strategies are increasing semantic accuracy of queries?}

In order to enhance the semantic accuracy a series of improvements may be applied to the pipeline.
Potential optimizations include supplying (parts of) the schema during LLM prompting, implementation of
interactive contextual reasoning through a conversational interface which would allow for user
refinement, the implementation of a robust SQL parsing and validation mechanism and a hybrid approach
partly relying on traditional NLP preprocessing techniques. This research will quantify semantic accuracy
using popular NL2SQL benchmarks and empirically evaluate the impact each approach has on the benchmark
performance. Furthermore this research will take a look at the optimal combination of the aforementioned
solutions in order to develop a system that strikes the right balance between accuracy and performance.

% \subsubsection{Secondary Research Questions}

% \subsubsection*{RQ4 — What use cases are most suitable for natural language database interfaces?}

% \subsubsection*{RQ5 — Where are the limitations of natural language database interfaces?}

% \subsubsection*{RQ6 — How is NL2SQL model performance impacted through model selection, fine tuning and pre/post processing?}

\newpage
