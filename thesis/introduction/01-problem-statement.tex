\subsection{Problem Statement and Motivation}\label{section:introduction:problem}

Database systems represent a backbone of modern computer science, allowing
rapid advancements whilst shielding us from the problem categories that come
with managing and querying large amounts of structured data
efficiently. However, most Database Management Systems (DBMS) have
required specialized knowledge of the Structured Query
Language (SQL) to become usable. Whilst this barrier may be perceived
differently across user groups it represents a fundamental misalignment
between end-user goals (e.g. analysts, researchers, domain experts etc.) and
the underlying DBMS, requiring software engineering efforts to reduce this friction.

% TODO: Add a concrete example or case study here to illustrate the impact of this barrier
This barrier is the reason entire classes of software projects exists (for
example, admin and support panels), data analytics tools etc. which therefore
introduce significant churn and delay between the implementation of a database
system and reaching the desired end user impact. Often these projects span
multiple years, require costly staffing and yield little to no novel technical
value.

Large Language Models (LLMs) have proven
themselves as a tool for bridging fuzzy user provided input into
discrete, machine readable formats. Prominent models have
demonstrated capabilities that enable computer scientists to tackle
problem classes that used to be challenging or yield unsatisfying results
with logical programming approaches.

This thesis explores ways to overcome the barrier using
natural language queries, so domain experts, business owners, support
staff etc. are able to interact with their data,
eliminating the requirement of learning SQL (and its pitfalls). By translating
natural language to SQL using Large Language Models this translation becomes
robust (e.g. against different phrasing) and enables
applications in how businesses, researchers and professionals interact with
their data — it represents a shift (ie. moving away from SQL)
towards a more inclusive and data driven world. 
