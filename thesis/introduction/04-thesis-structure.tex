\subsection{Structure of the Thesis}\label{section:introduction:structure}

This thesis is following a research and development methodology in order to
implement a portable NL2SQL system that can be used in sovereign environments and
be embedded into databases.

\begin{enumerate}
    \item \textbf{Literature Review} — An analysis of the existing research in the fields of
        natural language interfaces (NLI) for databases, GPU integration for acceleration
        of database operations, and LLM/AI Model integration with database systems.
        This phase establishes the theoretical foundation for this research and identifies current 
        state-of-the-art approaches, their benefits and shortcomings.
    \item \textbf{Theoretical Foundations} — Introducing the theoretical concepts and frameworks 
        required for understanding the problem statement and the proposed solution. The fundamentals
        introduced in this section are referenced in the system design and implementation phases
        respectively.
    \item \textbf{System Design} — Design of a system architecture that can translate natural language
        to SQL using LLMs. The primary goal of the system design phase is to arrive at an architecture
        that yields low latency natural language processing, schema-aware SQL query generation and ambiguity
        resolution whilst maintaining a high semantic accuracy.
    \item \textbf{Implementation} — The implementation of a PostgreSQL extension according to the
        above system design that relies on \texttt{rust} and \texttt{pgrx}. This extension will
        provide a GPU accelerated framework for executing LLMs, implement a natural language
        to query generation pipeline that relies on the SQL schema and create database functions
        and operators for both query generation and execution.
    \item \textbf{Evaluation} — Benchmarking the implementation against standard evaluation datasets
        framework and benchmark that introspects the implementations performance in multiple
        dimensions. The most relevant evaluation metrics for this thesis are execution accuracy
        ($\mathbb{EA}$), exact match ($\mathbb{EM}$), error rate ($\mathbb{ER}$) and candidate 
        latency ($\mathbb{CL}$).
    \item \textbf{Discussion} — Analysis and interpretation of the evaluation phase results against
        the research goals of this thesis. Evaluating the performance and accuracy results recorded
        during the benchmarks against the question whether real world deployments of NILs are feasible.
        Furthermore the impact of approaches used are shown and it is determined whether a statistically
        significant improvement can be achieved.
    \item \textbf{Summary and Outlook} — Summarizes the contributions, addresses limitations
        of this thesis and the implementation, and proposes directions for future research alongside
        possible applications. Primary future research topics include advanced GPU optimization
        techniques (e.g. further quantization), accuracy and performance impact of model fine tuning,
        techniques, scalability of such a system in enterprise scenarios and the evaluation of security
        and privacy considerations (e.g. managing access control).
\end{enumerate}
