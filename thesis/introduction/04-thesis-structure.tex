\subsection{Structure of the Thesis}\label{section:introduction:structure}

This thesis follows a research and development methodology to implement a
portable NL2SQL system that can be used in sovereign environments and be
embedded into databases.

\begin{enumerate}
    \item \textbf{Literature Review} — Analysis of existing research in the
        fields of natural language interfaces (NLI) for databases, GPU
        integration for acceleration of database operations, and LLM/AI Model
        integration with database systems. This phase establishes the
        theoretical foundation and identifies current state-of-the-art
        approaches, their benefits and shortcomings.
    \item \textbf{Theoretical Foundations} — Introduces theoretical concepts
        and frameworks required for understanding the problem statement and the
        proposed solution. The fundamentals introduced in this section are
        referenced in the system design and implementation phases.
    \item \textbf{System Design} — Design of a system architecture that
        translates natural language to SQL using LLMs. The goal is to arrive at
        an architecture that implements robust natural language processing,
        schema-aware example selection and self refinement stages whilst
        maintaining high semantic accuracy.
    \item \textbf{Implementation} — Implementation of a portable, standalone
        NL2SQL pipeline, according to the produced system design. Porting
        machine learning algorithms from the system design phase into a
        production grade implementation.
    \item \textbf{Evaluation} — Benchmarking the implementation against
        standard evaluation datasets that introspect the implementations
        performance in multiple dimensions. The most relevant evaluation
        metrics are execution accuracy ($\mathbb{EA}$), exact match
        ($\mathbb{EM}$), error rate ($\mathbb{ER}$) and candidate latency
        ($\mathbb{CL}$).
    \item \textbf{Discussion} — Analysis and interpretation of the evaluation
        results against the research goals. Evaluating the performance and
        accuracy results recorded during the benchmarks against whether real
        world deployments of NILs are feasible. The impact of approaches used
        are shown and it is determined whether a statistically significant
        improvement can be achieved.
    \item \textbf{Summary and Outlook} — Summarizes the contributions,
        addresses limitations of the thesis and the implementation, and
        proposes directions for future research alongside possible
        applications. Future research topics include GPU optimization
        techniques (e.g. further quantization), accuracy and performance impact
        of model fine tuning, scalability in enterprise scenarios and the
        evaluation of security and privacy considerations (e.g. managing access
        control).
\end{enumerate}
