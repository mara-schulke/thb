\subsection{Thesis Structure}\label{section:introduction:structure}

This thesis follows a research and development methodology to implement a
portable NL2SQL system that can be used in sovereign environments and be
embedded into databases.

\begin{enumerate}
    \item \textbf{Literature Review} — Analysis of existing research on NLIDB.
        This review establishes the ecosystem context and identifies current
        state-of-the-art approaches, their benefits and shortcomings.
    \item \textbf{Theoretical Foundations} — Introduces theoretical concepts
        required for understanding the problem statement and the
        proposed solution. The fundamentals introduced in this section are
        referenced in the system design and implementation.
    \item \textbf{System Design} — Design of a system architecture that
        translates natural language to SQL using LLMs. The goal is to arrive at
        an architecture that implements robust natural language processing,
        schema-aware example selection and self refinement stages whilst
        maintaining high semantic accuracy.
    \item \textbf{Implementation} — Implementation of a portable, standalone
        NL2SQL pipeline, according to the system design. Porting machine
        learning algorithms from the system design into a production grade
        implementation.
    \item \textbf{Evaluation} — Benchmarking the implementation against
        standard evaluation datasets that introspect the implementations
        performance in multiple dimensions. The most relevant evaluation
        metrics are execution accuracy ($\mathbb{EA}$), exact match
        ($\mathbb{EM}$), error rate ($\mathbb{ER}$) and candidate latency
        ($\mathbb{CL}$).
    \item \textbf{Discussion} — Analysis and interpretation of the evaluation
        results against the research goals. Interpreting the results recorded
        during the evaluation and determinin the impact of approaches used.
    \item \textbf{Summary and Outlook} — Summarizes the contributions,
        addresses limitations of this thesis and the accompanying
        implementation, and proposes directions for future research alongside
        possible applications.
\end{enumerate}
