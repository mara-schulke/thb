\subsection{Structure of the Thesis}

This thesis is following a research and development methodology in order to implement a natural language
interface for databases, in particular postgres is used.

\begin{enumerate}
    \item \textbf{Literature Review} — An analysis of the existing research in the fields of
          natural language interfaces (NLI) for databases, GPU integration for acceleration
          of database operations, and LLM/AI Model integration within database systems.
          This phase establishes the theoretical foundation for this research and identifies current 
          state-of-the-art approaches, their benefits and shortcomings.
     \item \textbf{Decomposition \& Requirements} — Decomposing the problem statement into its
          fundamentals and deriving system requirements for the design phase from it. The goal
          of this section is to arrive at a list of functional and non-functional requirements that
          must be taken into account and fulfilled by the design and implementation phases respectively.
     \item \textbf{System Design} — Design of a system architecture that can utilize GPU acceleration
          for LLM integration from within postgres. The primary goals of the system design phase
          are to arrive at an architecture that yields low latency natural language processing,
          schema-aware SQL query generation, ambiguity detection and resolution whilst maintaining
          a high semantic accuracy.
     \item \textbf{Implementation} — The implementation of a PostgreSQL extension according to the
          above system design that relies on \texttt{rust} and \texttt{pgrx}. This extension will
          provide a GPU accelerated framework for executing LLMs, implement a natural language
          to query generation pipeline that relies on the SQL schema and create database functions
          and operators for both query generation and execution.
     \item \textbf{Evaluation and Benchmarking} — An assesment framework and benchmark that introspects
          the implementations performance in multiple dimensions. Namely the most relevant dimensions
          for this thesis are:
          \begin{enumerate}
              \item Semantic Accuracy — Measuring the overall accuracy of results delivered for a given
                    natural language input.
              \item Ambiguity Resolution Capabilities — How well the system performs when confronted with
                    ambiguous natural language input and database schemas.
              \item Performance Metrics — Measuring the latency, throughput and resource utilization 
                    of the implementation.
          \end{enumerate}
      \item \textbf{Discussion} — Analysis and interpretation of the evaluation phase results against
            the research goals of this thesis. Evaluating the performance and accuracy results recorded
            during the benchmarks against the question whether real world deployments of NILs are feasible.
            Furthermore the effectiveness of ambiguity resolution capabilities and semantic accuracy enhancement
            strategies are showing a statistically significant effect.
       \item \textbf{Summary and Outlook} — Summarizes the contributions, addresses limitations
            of this thesis and the implementation, and proposes directions for future research alongside
            possible applications. Primary future research topics include advanced GPU optimization
            techniques (e.g. further quantization), accuracy and performance impact of model fine tuning,
            techniques, scalability of such a system in enterprise scenarios and the evaluation of security
            and privacy considerations (e.g. managing access control).
\end{enumerate}
