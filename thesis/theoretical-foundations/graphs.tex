\subsection{Graph Theory}

Graph theory is a mathematical domain for working with structures and
relationships. It is highly relevant for databases and NL2SQL systems as
modeling relationships in databases maps intuitively to graphs. This section
introduces the theoretical framework of graphs, graph kernels and graph
similarity which are required foundations for the example selection algorithm
of \textsc{Natural}.

\subsubsection{Graph Fundamentals}

A graph is noted as $G = (V, E)$ in its simplest form and consists of vertices
$V$ and edges $E \subseteq V \times V$ which are connections between vertices.
Labeled graphs are an extension to this concept which associate labels with
vertecies and edges, encoding semantic information such as vertex types,
attributes, or constraints.

\subsubsection{Graph Kernels}

Graph kernels are functions $k: \mathcal{G} \times \mathcal{G} \rightarrow
\mathbb{R}$ which are measuring the structural similarity between two graphs:

\begin{equation}
k(G_1, G_2) = \langle \phi(G_1), \phi(G_2) \rangle
\end{equation}

where $\phi: \mathcal{G} \rightarrow \mathcal{H}$ maps graphs to a feature
space $\mathcal{H}$, which allows measuring similarity in the feature space,
rather than using raw graph matching to compute the similarity. Thus, on a high
level, graph kernels may be used to identify common super structures in graphs.

\subsubsection{Weisfeiler-Lehman Algorithm}

The Weisfeiler-Lehman (WL) algorithm is a graph kernel algorithm which
iteratively refines node labels based on the neighborhood structure in the
graph \citep{WWL}:

\begin{enumerate}
    \item \textbf{Sorting} — Represent vertices as a sorted list of neighbors ($L_v$)
    \item \textbf{Compression} — Compute the hash of $L_v$ for every vertex ($h(L_v)$)
    \item \textbf{Relabeling} — Relabel every vertex $v$ with $h(L_v)$ as its new node label
\end{enumerate}

This algorithm is repeated for $w$ iterations (typically $w = 3-5$, due to
deminishing returns). After the iterative vertex label computation, labels
encode the $w$-hop neighborhood structure. Subsequently a label histogram can
be used as the graph-level feature vector, enabling efficient structural
comparison.

% https://ethz.ch/content/dam/ethz/special-interest/bsse/borgwardt-lab/documents/slides/CA10_WeisfeilerLehman.pdf

\subsubsection{Wasserstein Distance}

The Wasserstein distance measures the minimum cost of transforming one
probability distribution into another (also known as earth mover's distance).
Given two distributions $\mu$ and $\nu$ over space $\mathcal{X}$ with metric $d$:

\begin{equation}
W_p(\mu, \nu) = \left(\inf_{\gamma \in \Gamma(\mu, \nu)} \int_{\mathcal{X} \times \mathcal{X}} d(x, y)^p \, d\gamma(x, y)\right)^{1/p}
\end{equation}

where $\Gamma(\mu, \nu)$ is the set of couplings of $\mu$ and $\nu$.
Unlike other distance metrics of probability distributions like the
Kullback-Leibler divergence, the Wasserstein distance accounts for the
underlying metric space structure \citep{WWL}.

\subsubsection{Wasserstein Weisfeiler-Leman Kernels}

WWL kernels combine the labels computed using WL with Wasserstein distance to
measure graph distance \citep{WWL}:

\begin{enumerate}
    \item Apply $w$ WL iterations to both graphs, producing node label distributions at each iteration
    \item Compute Wasserstein distance between graphs using Hamming distance
        (for the categorical features) or Euclidean distance (for continuous
        features) as the ground metric
    \item Aggregate distances into a similarity score using a Laplacian kernel:\\
          $k_{\text{WWL}}(G_1, G_2) = \exp\left(-\lambda  D_W)\right)$
\end{enumerate}

The resulting graph kernel captures both local neighborhood structures (using
the WL algorithm) and global distributional properties (using the Wasserstein
distance) \citep{WWL}. For database schemas, WWL kernels can similar structural
patterns (table structures, possible joins, constraints) while remaining robust
to minor structure variations. This makes them particularly well suited for
NL2SQL example selection algorithms that factor in schema similarity.
