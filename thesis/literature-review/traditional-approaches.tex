\subsection{Traditional NL2SQL Approaches}

% TODO: Condense

Prior to the wide-spread dominance of machine learning approaches for natural language processing a variety of
traditional, rather discrtete approaches have been explored in the field of NL2SQL / NLIDBs. These logical
programming approaches have laid the foundations for transitioning towards the application of machine learning
techniques for NL2SQL.

\subsubsection{Rule-based and Grammar-based Systems}

Foundational research of NL2SQL system mostly focused around applying rule engines that were tedious to set up and
expensive to maintain / transfer across database systems. These rule engines mostly relied on the systematic 
identification of linguistic patterns / were trying to template SQL from information that was derived from processing the 
natural language query. \citep{Rendezvous, Lunar, Ladder} These approaches mostly tried to formalize natural language 
queries into formal grammers which could then be deterministically mapped into a valid SQL query. \citep{Lunar} These 
approaches have strong downsides when it comes to the variety of natural language constructs they can process, aswell as 
runtime adoption of new / unknown databases, query constructs etc. A potential upside of this class of NL2SQL systems is 
that they can confidently and reproducably identify questions they can, and can't answer — thus leading to very reliable 
and predictable user interfaces.

\subsubsection{Semantic Parsing using String-Kernels}

A significiant milestone in parsing techniques of natural language queries was reached by \citeauthor*{StringKernels} in 
\citeyear{StringKernels}. The introduction of string kernels for semantic parsing represented a novel achievement, when 
it comes to fusing logical programming approaches using a formal grammer like LSNLIS developed by \cite{Lunar} and 
learning / training approaches to understand unseen language patterns / unknown natural language query structures. This 
allowed for more flexible pattern recognition when compared to traditional rule-based systems.

The core innovative characteristic of this approach lies in its capability to understand similarities between natural 
language expressions based on subsequence patterns rather than relying on exact matches. This made \textsc{Krisp}, the 
research NLIDB system developed by \cite{StringKernels} much more robust to language variations in phrasing and noise 
(e.g. spelling mistakes) in the input. As the \citeauthor*{StringKernels} demonstrated through experiments on real-world 
datasets, this approach compared favorably to existing systems of the time like \textsc{Chill}, especially in handling 
noisy inputs — a frequent challenge rigid rule-based systems faced in real world scenarios \citep{StringKernels, 
ILPParsing}.

\subsubsection{Graph Matching Methods}

\cite{GraphMatching} brought together several research threads and reapplied emerging graph matching research models
to natural language processing, specifically to natural language queries. Graph matching was applied once the natural 
language query was parsed using a Combinatory Categorial Grammar (CCG) approach into a semantic graph which denotes the 
relationship between semantic entities in it. This graph could then be matched against the actual graph derived from 
the database, since they share topological traits that can be used for matching \citep{GraphMatching}. This approach 
allowed to apply querying systems without having any question-answer pairs or manual annotations for training the system, 
which implies easier scalability / transferability across domains, since the system does not require any additional 
tweaks.

Even though this approach was novel and showed improved the performance over existing state-of-the-art approaches, it
was showing that graph matching quickly reaches its limitations. This approach relied heavily on the CCG parser's 
accuracy, with parsing errors accounting for 10-25\% (depending on the dataset) of system failures \citep{GraphMatching}. 
Furthermore it struggled with both ambiguous language constructs and potential mismatches between natural language 
representation of relationships and database layouts — more complicated database designs, which may not match the users 
intuitive understanding resulted in a different topology and hence could not be matched \citep[p.~387]{GraphMatching}.

\subsubsection{Interactive Systems}

In \citeyear{NALIR} \citeauthor*{NALIR} identified that perfect translation of natural language into SQL was challenging
due to natural language not being made for query expressions as it heavily relies on contextual information and clarifying
questions in order to disambiguate conversations \citep{NALIR}. These learnings relate to early prior art from 
\citeauthor{UnnaturalQueryLanguage} and \citeauthor{Rendezvous} which also made this observation — ``natural language
is not a natural query language.'' \citep{UnnaturalQueryLanguage}. The solution introduced by NaLIR further emphasized
how important an interactive, conversational usage model is, when offering a natural language interface \citep{NALIR}.

NaLIR could accept logically complex English language sentences as input and translate them into SQL queries with various 
complexities, including aggregation, nesting, and different types of joins etc. The key innovative characteristic of 
NaLIR lies in its interactive communication mechanism (much like \textsc{Rendezvous}) that could detect potential 
misinterpretations and engage users to resolve ambiguities present in their natural language query without forcing them
to entirely rephrase their query \cite{NALIR}. This approach, while showing awareness for its limitations (with regards
to entirely automating / deriving SQL generation from potentially ambiguous or faulty user input) showed that it was 
possible to overcome these limtations through choosing the right interaction model — ``In our system, we generate
multiple possible interpretations for a natural language query and translate all of them in natural language for the
user to choose from'' —, rather than optimizing the generation part of the system \cite{NALIR}.

\subsubsection{Query Synthesis}

\cite{SQLizer} introduced SQLizer, which synthesizes SQL queries from natural language \citep{SQLizer}. This paper
presents a novel approach when it comes to NL2SQL as it is merging prevalent semantic parsing techniques (outlined above)
with an program synthesis (or query synthesis) approach. SQLizer makes use of a three stage processing model for 
natural language models: first generating a sketch of the query using semantic parsing, then using type-directed
synthesis to complete the sketch and finally using automated repair, if required. 

\citeauthor*{SQLizer} show that alternating between repairing and synthesis yields results that beat state-of-the-art
NL2SQL approaches like NaLIR. SQLizer is fully automated and database-agnostic, requiring no knowledge of the underlying
schema. The authors evaluated SQLizer on 455 queries across three databases, where it ranked the correct query in the
top 5 results for roughly 90\% of the queries. This represents a significant improvement over NaLIR \citep{NALIR},
the previous state-of-the-art system \citep{SQLizer}.

Potential short commings of this approch include queries which yield empty results, dealing with language variations
as SQLizer is still using semantic parsing, and domain-specific terminology, all while still requiring users
to select from multiple query options which reduces the overall usability of the system \citep[p.22-23]{SQLizer}.

\subsubsection{Limitations of Traditional Approaches to NL2SQL}

Despite being innovative and achieving state-of-the-art results, many of the above outlined approaches face severe 
challenges when moving outside of an research environment. Many of these systems performed comparatively good on research
benchmarks that were often composed of controlled question types and limited data variety. Ultimatively no standard
benchmark existed for NL2SQL in this era, hence comparing different NL2SQL systems against each other is a problem on 
its own. Despite not having a standard benchmark that all approaches could be unifiably evaluated against, several 
fundamental challenges emerged / remained with these approaches:

\begin{enumerate}
    \item \textbf{Limited linguistic coverage} — Prevalent rule-based and semantic-parsing based systems were only able to 
          process the a small subset of the natural language they were programmed for. This severely limited their 
          ability to handle different phrasings of the same end-user goal \citep{StringKernels, UnnaturalQueryLanguage, 
          Lunar, Ladder}.
    \item \textbf{Transferability} — Traditional approaches typically required extensive manual configuration or at least
          a training phase / adaption for each database they were deployed for, hindering cross domain usage through being 
          expensive and time-consuming to adapt \citep{NLIDBs, Lunar}.
    \item \textbf{Brittleness} — Many of the systems introduced in this subchapter did not handle synonyms, paraphrasing,   
          or spelling errors well. Manual adaption / handling was needed in order to becomes resilient against each class
          of problems \citep{StringKernels, SQLizer}.
    \item \textbf{Poor scalability} — With potentially more complex underlying databases, traditional 
          solutions often showed to perform worse. \citeauthor*{GraphMatching} found, that with increasing schema
          complexity more compute was required to resolve the natural language query to a suitable query candidate
          making them less transferable and scalable than initially anticipated \citep{GraphMatching} — ``Evaluating on
          all domains in Freebase would generate a very large number of queries for which denotations would have to be 
          computed ... Our system loads Freebase using Virtuoso and queries it with SPARQL. Virtuoso is slow in dealing with
          millions of queries indexed on the entire Freebase, and is the only reason we did not work with the complete
          Freebase.'' which indicates underlying system design issues with runtime complexity.
\end{enumerate}

These flaws of traditional NL2SQL approaches made it apparent, that a different class of approaches is needed, which 
increase transferability and reduce the brittleness since users are ``unwilling to trade reliable and predictable user 
interfaces for intelligent but unreliable ones'' according to \cite{NLIDBTheory}. Whilst many approaches outlined tractable 
ways to increase user satisfaction and accuracy (like \citeauthor*{Rendezvous} did in \citeyear{Rendezvous} with a 
conversational approach), NLIDBs were and are not considered to be a solved problem.

