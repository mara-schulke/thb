\subsubsection{Deployment and Performance Gaps}

Despite impressive academic results, significant gaps remain in transitioning NL2SQL systems from research environments
to real-world, enterprise-feasible, deployments:

\begin{enumerate}
    \item \textbf{Database Integration} — While existing research has often focused on standalone systems, little attention
          has been given to the integration complexities between NL2SQL capabilities and database management systems like 
          PostgreSQL. Having two standalone systems imposes a significant data transfer need between the two systems when infact, 
          natural language queries can be treated as an extension to most existing databases. This implementation gap prevents 
          seamless adoption into existing databases as it requires additional software layers which in turn increase the overall 
          complexity and cost.
    \item \textbf{Hardware Requirement Optimization} — As mentioned above, current LLM-based approaches often require significant
          compute resources in order to achieve state-of-the-art results. There is limited research available on performance 
          optimization of NLIDBs in order to achieve practical and industry-viable hardware constraints.
    \item \textbf{Real-time Performance Considerations} — Most research papers evaluate models based on their accuracy metrics 
          alone without factoring in the latency and throughput characteristics of their solutions. This imposes a possibly 
          significant $\delta$ between academic research and production environments. Responsiveness is an important metric for 
          user experience and should therefore play a role in NLIDB research when it comes to evaluating different NL2SQL 
          architectures.
    \item \textbf{Privacy and Security} — While open-source models address some privacy concerns by enabling a local deployment of 
          LLMs, research gaps remain in ensuring that NL2SQL systems respect database access controls and security policies. This 
          is especially important for enterprise and government environments where data access is strictly regulated.
\end{enumerate}

\subsubsection{Ambiguity Resolution and Semantic Accuracy}

Despite the long existance of the research field, fundamental questions remain open with regards to handling ambiguity effectively
in natural language queries. Current, especially language model based, systems struggle to correctly identify when natural language
queries contain ambiguities that they cant confidently resolve. Although some approaches like multi-path generation and candidate 
selection are a promising way to work around ambiguous language, early systems like \textsc{NaLIR} showed that the most effective 
way to deal with inherently ambiguous language is asking clarifying questions, instead of trying to interpret the query best-effort
\citep{NALIR}. Ambiguous language is an inherent source of inaccuracy and therefore a cause for misleading query results. 
Contemporary research and benchmarks dont focus ambiguity detection and strategies resolution which therefore leaves open
questions to be further researched.

\subsubsection{Evaluation and Benchmarking Gaps}

\begin{enumerate}
    \item \textbf{Enterprise-grade Evaluation} — As highlighted by \textsc{Spider 2.0}, there is a gap between academic 
          benchmarks and enterprise realities. Further research is needed to create evaluation frameworks that better represent 
          real-world enterprise environments with thousands of tables and complex relationships. \cite{Spider2} is a promising 
          first step is this direction.
    \item \textbf{Performance Metrics} — As mentioned above, current benchmarks often dont capture latency or throughput of NL2SQL 
          systems at all, allowing for solutions to achieve state-of-the-art scores that require significantly more resources than 
          their predecessors. Having meaningful performance metrics and benchmarking would allow to further analyze the proposed 
          solutions for their real world feasibility.
    \item \textbf{User Experience Metrics} — Contemporary benchmarks focus on execution accuracy without assessing potential
          user satisfaction, trust, and overall experience. Metrics that capture these aspects are necessary for understanding the 
          actual utility of NL2SQL systems in practice. Although execution accuracy is a big factor for the trustworthiness of a 
          NLIDB, it is not the only component as indicated by \citeauthor*{NALIR} in \citeyear{NALIR}.
\end{enumerate}

\subsubsection{Thesis Placement}

This thesis addresses several of the above outlined research gaps. The primary focus of this work is the integration of open-source
NL2SQL models with advanced techniques like candidate selection, subset-encoding of database schemas, and synthetic example 
generation for in-context learning. Through the implementation of a PostgreSQL extension this work will bridge multiple critical 
gaps between theoretical advancements and their practical deployability into real world systems whilst exploring performance 
characteristics.
