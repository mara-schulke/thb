\subsection{Research Gaps}

The literature shows that significant progress was made in the development of
real world feasible NL2SQL systems. Despite these advancement several critical
research gaps remain open which limit the practical deployment of NL2SQL
systems.

\subsubsection{Advanced Open-Source Approaches}

\begin{enumerate}
    \item \textbf{Limited Exploration of Technique Synergies} — Contemporary
        research primarily focuses on techniques like candidate selection, RAG,
        and self-correction in isolation or with primarily closed-source LLMs.
        A gap remains in understanding how these techniques can be optimally
        combined.
    \item \textbf{Efficiency-Accuracy Tradeoffs} — Whilst recent research
        efforts focus primarily on achieving a SOTA execution accuracy metric,
        the relationship of computational requirements and performance gains
        remains unclear.
\end{enumerate}

\subsubsection{Deployment and Performance Gaps}

Despite impressive academic results, significant gaps remain in transitioning
NL2SQL systems from research environments to real-world deployments:

\begin{enumerate}
    \item \textbf{Database Integration} — While existing research has often
        focused on standalone systems, little attention has been given to the
        integration complexities between NL2SQL capabilities and database
        management systems like PostgreSQL.
    \item \textbf{Latency Performance Considerations} — Most research papers
        evaluate models based on their accuracy metrics alone without factoring
        in the latency and throughput characteristics of their solutions,
        even though responsiveness is an important metric for user experience.
    \item \textbf{Privacy and Security} — While open-source models address some
        privacy concerns by enabling a local deployment of LLMs, research gaps
        remain in ensuring that NL2SQL systems respect database access controls
        and security policies.
\end{enumerate}

\subsubsection{Ambiguity Resolution and Semantic Accuracy}

Fundamental questions remain open with regarding effective ambiguity
resolution. Current LLM-based systems struggle to identify queries containing
ambiguities that they can't confidently resolve. Current research and
benchmarks does not focus sufficiently on ambiguity detection and resolution.

\subsubsection{Evaluation and Benchmarking Gaps}

\begin{enumerate}
    \item \textbf{Performance Metrics} — Current benchmarks often don't capture
        latency or throughput of NL2SQL systems at all, allowing for solutions
        to achieve SOTA scores that require significantly more
        resources than their predecessors.
    \item \textbf{User Experience Metrics} — Contemporary benchmarks focus on
        execution accuracy without assessing potential user satisfaction,
        trust, and overall experience. Although execution accuracy is a big
        factor for the trustworthiness of a NL2SQL system, it is not the only
        component as indicated by \citeauthor*{NALIR} in \citeyear{NALIR}.
\end{enumerate}

\subsubsection{Thesis Placement}

The three primary identified gaps: Performance gaps between proprietary and
open-source models, inadequate ambiguity resolution and semantic accuracy and
the lack of production-ready implementations exploring poerformance
characteristics, motivate the implementation of \Natural. Specifically this
thesis researches the ability of schema-aware example selection for in-context
learning using graph-kernels to improve the semantic accuracy of NL2SQL
systems.

By providing a production-ready implementation of a \Natural, this work
further bridges the gaps between theoretical advancements and their practical
deployability, whilst exploring performance and portability characteristics.
