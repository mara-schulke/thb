\subsubsection{Deployment and Performance Gaps}

Despite impressive academic results, significant gaps remain in transitioning
NL2SQL systems from research environments to real-world, enterprise-feasible,
deployments:

\begin{enumerate}
    \item \textbf{Database Integration} — While existing research has often
        focused on standalone systems, little attention has been given to the
        integration complexities between NL2SQL capabilities and database
        management systems like PostgreSQL.
    \item \textbf{Hardware Requirement Optimization} — Current LLM-based
        approaches often require significant compute resources in order to
        achieve state-of-the-art results. There is limited research available
        on performance optimization of NL2SQL systems in order to achieve
        practical hardware constraints.
    \item \textbf{Latency Performance Considerations} — Most research papers
        evaluate models based on their accuracy metrics alone without factoring
        in the latency and throughput characteristics of their solutions.
        Responsiveness is an important metric for user experience and should
        therefore play a role in NL2SQL research when it comes to evaluating
        different NL2SQL architectures.
    \item \textbf{Privacy and Security} — While open-source models address some
        privacy concerns by enabling a local deployment of LLMs, research gaps
        remain in ensuring that NL2SQL systems respect database access controls
        and security policies. This is especially important for enterprise and
        government environments where data access is strictly regulated.
\end{enumerate}

\subsubsection{Ambiguity Resolution and Semantic Accuracy}

Despite the long history of this field, fundamental questions remain
open with regards to handling ambiguity effectively. Current LLM-based systems
struggle to correctly identify when queries contain ambiguities that they
can't confidently resolve. Some approaches like multi-path generation
and candidate selection are a promising way to handle ambiguous language,
early systems like \textsc{NaLIR} showed that the most effective way to deal
with ambiguous language is asking clarifying questions, instead of outputting a
best-effort query candidate \citep{NALIR}. Contemporary research and benchmarks
does not focus sufficiently on ambiguity detection and resolution.

\subsubsection{Evaluation and Benchmarking Gaps}

\begin{enumerate}
    \item \textbf{Enterprise-grade Evaluation} — As highlighted by
        \textsc{Spider 2.0}, there is a gap between academic benchmarks and
        enterprise realities. Further research is needed to create evaluation
        frameworks that better represent real-world enterprise environments
        with thousands of tables and complex relationships. \cite{Spider2} is a
        promising step is this direction.
    \item \textbf{Performance Metrics} — Current benchmarks often don't capture
        latency or throughput of NL2SQL systems at all, allowing for solutions
        to achieve state-of-the-art scores that require significantly more
        resources than their predecessors. Having meaningful performance
        metrics and benchmarking would allow to further analyze the proposed
        solutions for their real world feasibility.
    \item \textbf{User Experience Metrics} — Contemporary benchmarks focus on
        execution accuracy without assessing potential user satisfaction,
        trust, and overall experience. Metrics that capture these aspects are
        necessary for understanding the actual utility of NL2SQL systems in
        practice. Although execution accuracy is a big factor for the
        trustworthiness of a NL2SQL system, it is not the only component as
        indicated by \citeauthor*{NALIR} in \citeyear{NALIR}.
\end{enumerate}

\subsubsection{Thesis Placement}

The three primary identified gaps: Performance gaps between proprietary and
open-source models, inadequate ambiguity resolution and semantic accuracy and
the lack of production-ready implementations exploring poerformance
characteristics, motivate the implementation of \Natural~a NL2SQL system
bridging the gap between theoretical research performance and real world
deployability. Specifically this work researches the ability of schema-aware
example selection for in-context learning using graph-kernels to improve the
semantic accuracy of NL2SQL systems.

By providing a production-ready implementation of a \Natural, this work
further bridges the apparent gaps between theoretical advancements and their
practical deployability, whilst exploring performance and portability
characteristics.
