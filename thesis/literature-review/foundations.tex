\subsection{Foundations of Natural Language Interfaces to Databases}

% NOTE: Condense (reduce from 515 words to ~250-300 words - historical systems from 1970s-1990s have limited relevance to LLM approaches; keep core concepts like semantic tractability and reliability trade-offs but reduce detailed system descriptions)

Earlier papers in the research landscape on Natural Language Database Interfaces (NLIDBs) date over half
a century back, into the early 1970s. Two decades after the first major research systems where developed
in this domain, \citeauthor*{NLIDBs} have published an introduction and an overview over NLIDBs where an overview of
state-of-the-art approaches were provided. \citep{NLIDBs} Their work outlined multiple key issues and challenges
associated with NLIDBs, and compared them against existing / competing solutions like formal query languages,
form-based interfaces and graphical interfaces. These challenges (like unobvious limits, linguistic ambiguities,
semantic inaccuracy, tedious configuration etc.) have shaped this field of research and are still considered relevant
metrics today.

Early NLIDBs primarily relied on traditional natural language processing (NLP) techniques in order to achieve
natural language understanding capabilities. With \textsc{Chill} an inductive logic programming (ILP) approach
was first introduced for NL2SQL systems, marking one of the key events when it comes to machine learning usage.
\citep{ILPParsing} In \citeyear{ILPParsing2} \citeauthor*{ILPParsing2} have extended the approach of ILP
parsing for natural language database queries with multi clause construction, yielding promising results
in the field of NLIDBs. \citep{ILPParsing2}

Building on the systematic overview of \citeauthor*{NLIDBs} and the first machine learning approaches from
\citeauthor*{ILPParsing} aswell as \citeauthor*{ILPParsing2}, \citeauthor{NLIDBTheory} have proposed a novel 
approach for implementing NLIDBs and outperformed at the time state-of-the-art solutions from \cite{ILPParsing} 
\cite{ILPParsing2} — achieving 80\% semantic accuracy. \citep{NLIDBTheory} The novelty of the \textsc{Percise}
system lies in its natural language processing approach, specifically its lexical mapping strategy, allowing 
\textsc{Percise} to identify questions it can, and can't answer (introducing the concept of \textit{semantically 
tractable questions}) which therefore results in a better and interactive end user experience. Their experiements 
also showed that this approach is \textit{transferrable} and \textit{unbiased} — it is possible to feed in new, 
unknown questions into the system and maintain performance characteristics, whereas it was shown that 
\cite{ILPParsing} were suffering from a distribution drift of the questions asked. \citep{NLIDBTheory}

The theoretical foundations and research questions highlighted by the aforementioned works, shaped the research
field and highlighted the following, ongoing research:

\begin{enumerate}
    \item The trade-off characteristics derived from choosing a machine learning vs. traditional NLP approach (e.g. 
          \textsc{Chill} versus \textsc{Percise}). E.g. coverage versus correctness. \citep{ILPParsing, NLIDBTheory}
    \item The linguistic challenges associated with bringing NLIDBs into use (e.g. semantic inaccuracy, linguistic 
          ambiguity, unclear language coverage etc.) \citep{NLIDBs}
    \item The value of systems and approaches which double down on reliability and semantic accuracy rather than giving
          promising but incorrect answers. \citep{NLIDBs, NLIDBTheory}
\end{enumerate}

Fundamentally this highlights the tension and mismatch between the characteristics of natural language, which is 
able to be ambiguous, \textit{semantically untractable} or able to be incomplete in meaning and formal languages
like SQL which always have on deterministic and \textit{semantically tractable} meaning they convey in each statement. 
As Schneiderman and Norman have pointed out according to \citeauthor*{NLIDBTheory}, users are ``unwilling to trade 
reliable and predictable user interfaces for intelligent but unreliable ones'' which induces performance expectations
on NLIDB implementations to be highly certain about the questions it can, and can't answer, whilst maintaing as high
as possible natural language coverage. \citep{NLIDBTheory}
