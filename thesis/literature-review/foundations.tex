\subsection{Foundations of Natural Language Interfaces to Databases}

Earlier research on Natural Language Database
Interfaces (NLIDBs) dates back over half a century, to the early 1970s. Two
decades after the first major research systems were developed in this domain,
\citeauthor*{NLIDBs} published an overview of NLIDBs and at the time,
state-of-the-art approaches \citep{NLIDBs}. Their work outlined key issues and
challenges associated with NLIDBs, and compared them against existing solutions
like formal query languages, form-based interfaces and graphical interfaces.
These challenges like unobvious limits, linguistic ambiguities, semantic
inaccuracy, tedious configuration etc. have shaped this field and remain
relevant metrics today.

Early NLIDBs relied on traditional natural language processing (NLP)
techniques to achieve natural language understanding capabilities.
With \textsc{Chill} an inductive logic programming (ILP) approach was first
introduced for NL2SQL systems, marking a key event for
machine learning usage \citep{ILPParsing}. In \citeyear{ILPParsing2}
\citeauthor*{ILPParsing2} extended the approach of ILP parsing for natural
language database queries with multi clause construction, yielding promising
results in NLIDBs \citep{ILPParsing2}.

\citeauthor{NLIDBTheory} proposed a novel approach for implementing NLIDBs and
outperformed the existing solutions, achieving 80\% semantic accuracy
\citep{NLIDBTheory}. The novelty of the \textsc{Precise} system lies in its
natural language processing approach, its lexical mapping strategy, allowing
\textsc{Precise} to identify questions it can, and can't answer, introducing
the concept of semantically tractable questions, which results in an improved
end user experience. Their approach was shown to be transferable and unbiased,
able to handle unknown questions and maintain performance characteristics.
Previous ILP-based approaches were suffering from a distribution drift of the
questions asked.

This research era predominately highlighted the tension between natural
language, which can be ambiguous, semantically intractable or incomplete in
meaning and formal languages like SQL which always have one deterministic and
semantically tractable meaning in each statement.
