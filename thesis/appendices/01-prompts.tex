\subsection{Prompts}

\subsubsection{Natural Inference Prompt}\label{appendix:prompt:natural:inference}

The following prompt is used by the ICL generator stage ($\pi$) when relevant examples
are available. Template variables are denoted \texttt{\{variable\}}: \texttt{\{schema\}}
is the (optionally subsetted) database schema, \texttt{\{examples\}} is the rendered
example block (see below), and \texttt{\{query\}} is the natural language question.

\begin{verbatim}
### Task Overview
1. You are a text to sql model that is meant to answer the question
   using valid sql.
2. Below, you are provided with a database schema, a set of examples
   and a natural language question.
3. Your task is to understand the schema, the examples and generate a
   valid SQL query to answer the question.

### SQLite Schema
```
{schema}
```
This schema describes the database's structure, including tables,
columns, primary keys, foreign keys, and any relevant relationships
or constraints.

### Examples
{examples}

### Instructions
- Make sure you only output the information that is asked in the
  question. If the question asks for a specific column, make sure to
  only include that column in the SELECT clause, nothing more.
- The generated query should return all of the information asked in
  the question without any missing or extra information.
- Before generating the final SQL query, please think through the
  steps of how to write the query.

### Question
`{query}`

### Output Format
In your answer, please enclose the generated SQL query in a code block:

```sql
-- Your SQL Query
```

Take a deep breath and think step by step to find the correct SQL query.
\end{verbatim}

\noindent The \texttt{\{examples\}} block is rendered from the retrieved sample
candidates (filtered to distance $< 0.5$) as follows. Each example is annotated
with its cosine similarity score:

\begin{verbatim}
Some example questions and corresponding SQL queries are provided
based on similar problems:

```sql
-- Question: {question_masked} [similarity={score}]
{sql}

-- Question: {question_masked} [similarity={score}]
{sql}
```

These examples can help you find the solution to the following
question, you must choose the most relevant example!
\end{verbatim}

\noindent When no examples pass the relevance threshold, \texttt{\{examples\}} is
replaced with:

\begin{verbatim}
No relevant examples found, *carefully* think through the given
question and schema to arrive at a valid query
\end{verbatim}

\subsubsection{Natural Refinement Prompt}\label{appendix:prompt:natural:refinementprompt}

The following prompt is used by the self-refinement stage ($\rho$). It provides the
candidate query generated in the projection stage alongside the schema and original
question for correction. Template variables: \texttt{\{schema\}}, \texttt{\{query\}}
(the natural language question), and \texttt{\{candidate\}} (the SQL candidate to
be revised).

\begin{verbatim}
### Task Overview:
1. You are a text to sql model that is meant to answer the question
   using valid sql.
2. Below, you are provided with a database schema, a set of examples
   and a natural language question.
3. Your task is to understand the schema, revisit the given SQL query
   to answer the question.
4. If you spot any errors in the SQL query, correct them!

### Database Engine:
SQLite

### Database Schema:
```
{schema}
```
This schema describes the database's structure, including tables,
columns, primary keys, foreign keys, and any relevant relationships
or constraints.

### Question
```sql
-- Question: {query}
{candidate}
```

### Instructions
- Make sure you only output the information that is asked in the
  question. If the question asks for a specific column, make sure to
  only include that column in the SELECT clause, nothing more.
- The generated query should return all of the information asked in
  the question without any missing or extra information.
- Before generating the final SQL query, please think through the
  steps of how to write the query.

### Output Format
In your answer, please enclose the generated SQL query in a code block:

```sql
-- Your SQL Query
```

Take a deep breath and think step by step to find the correct SQL query.
\end{verbatim}

\subsubsection{OmniSQL Inference Prompt}\label{appendix:prompt:omnisql:inference}

The following simplified prompt is used for the zero-shot \textit{Baseline}
configuration and as the base prompt for the \OmniSQL~model directly (without
in-context examples). Template variables: \texttt{\{schema\}} and \texttt{\{query\}}.

\begin{verbatim}
Task Overview:
You are a data science expert. Below, you are provided with a database
schema and a natural language question. Your task is to understand the
schema and generate a valid SQL query to answer the question.

Database Engine:
SQLite

Database Schema:
{schema}
This schema describes the database's structure, including tables,
columns, primary keys, foreign keys, and any relevant relationships
or constraints.

Question:
{query}

Instructions:
- Make sure you only output the information that is asked in the
  question. If the question asks for a specific column, make sure to
  only include that column in the SELECT clause, nothing more.
- The generated query should return all of the information asked in
  the question without any missing or extra information.
- Before generating the final SQL query, please think through the
  steps of how to write the query.

Output Format:
In your answer, please enclose the generated SQL query in a code block:
```
-- Your SQL query
```

Take a deep breath and think step by step to find the correct SQL query.
\end{verbatim}
