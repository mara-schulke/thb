\subsection{Problem Statement and Motivation}

Database systems represent a backbone of modern computer science, allowing for rapid advancements
whilst shielding us from the problem categories that come along with managing and querying large amounts
of, usually structured, data efficiently. However, most Database Management Systems (DBMS) have
traditionally required specialized knowledge, usually of the Structured Query Language (SQL), in order
to become useable. Whilst this barrier may be percieved differently across diverse usergroups it
represents a fundamental misalignment between end-user goals (e.g. analysts, researchers, domain experts
etc.) and the underlying DBMS, thus often requiring software engineering efforts in order to reduce this friction.

This barrier is the reason entire classes of software projects exists (for example, admin / support panels),
data analytics tools etc. which therefore introduce significant churn and delay between the implementation
of a database system and reaching the desired end user impact. Often these projects span multiple years, require
costly staffing and yield little to no novel technical value.

Emerging technologies such as Large Language Models (LLMs) have proven themselves as a sensible tool for bridging
fuzzy user provided input into discrete, machine readable formats. Prominent models in this field have demostrated
outstanding capabilities that enable computer scientists to tackle new problem classes, that used to be
challenging / yielded unsatisfying results with logical programming approaches.

This thesis is exploring ways to overcome the above outlined barrier using natural language queries, so that domain experts,
business owners, support staff etc. are able to seamlessly interact with their data, essentially eliminating the
requirement of learning SQL (and its pitfalls). By translating natural language to SQL using Large Language Models
this translation becomes very robust (e.g. against different kinds of phrasing) and enables novel applications
in how businesses, researchers and professionals interact with their data — it represents a fundamental shift 
(ie. moving away from SQL) towards a more inclusive and data driven world. 
