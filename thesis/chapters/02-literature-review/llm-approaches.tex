\subsection{Large Language Models}

The emergence of LLMs such as GPT-3, GPT-4, and Claude fundamentally transformed the landscape of NL2SQL research. Early 
experiments with LLMs for NL2SQL tasks showed state-of-the-art capabilities in comparsion with contemporary PLM approaches 
\citep{DAIL-SQL}. \cite{T2SQL-LLM-Bench-3} demonstrated that \textsc{Codex} (a contemporary model based on GPT-3), without any 
fine-tuning efforts, could achieve competitive performance on \textsc{Spider}, outperforming many state-of-the-art approaches that 
required extensive training. This breakthrough challenged the contemporary assumption that further specialization of model 
architectures would yield increases in NL2SQL performance (e.g. \textsc{Graphix-T5}) \citep{GRAPHIX}.

\subsubsection{In-Context Learning}

In-Context Learning (ICL) is a foundational approach for leveraging the ability of LLMs to utilize larger context windows for
inference than traditional PLMs. Typical context windows of state-of-the-art LLMs can reach up to hundred thousands of tokens.
This characteristic of LLMs enabled researchers to utilize this context window to provide examples of accurate NL2SQL translation
instead of applying parameter updates. This paradigm shift has made developing NL2SQL systems significantly more accessible.

The fundamental principle of few-shot learning for NL2SQL involves providing the LLM with a small number of example pairs
of natural language and their corresponding SQL representation. These examples can benefit the model's understanding of
mapping between natural language and SQL syntax. This essentially builds on top of prior research like \textsc{Grappa}
and \textsc{StruG}, but applying these examples at inference time, rather than training time. Although this increases
the inference cost of such a system, the upsides lie primarily in the flexibility of such an approach — database content
/ prior usage of the system can be dynamically utilized, rather than requiring retraining.

Example selection strategies showed to have a considerable impact on ICL performance. \cite{DAIL-SQL} evaluated
various example selection methods like \textit{Random}, \textit{Question Similarity Selection (QTS)}, 
\textit{Masked Question Similarity Selection (MQS)}, and \textit{Query Similarity Selection (QRS)}. \citeauthor*{DAIL-SQL}
propose a novel strategy to select, organize and present ICL examples to LLMs. DAIL-SQL utilizes both question and
query similarity, masking domain-specific words and prioritizing examples that exceed a similarity threshold of $\tau$
\citep[p.~5]{DAIL-SQL}. DAIL-SQL encodes examples as question-SQL-pairs without the respective schema to improve
token efficiency. Using a Code Representation Prompt (CR) for question and schema encoding yielded DAIL-SQL to achieve
state-of-the-art 86.6\% execution accuracy on \textsc{Spider}.

The comparison between zero-shot and few-shot performance reveals the accuracy gain potential through supplying examples
to models in the inference context. While contemporary LLMs (such as GPT-4) have demonstrate impressive zero-shot performance
(achieving 72.3\% execution accuarcy on benchmarks like \textsc{Spider}) \citep[Table 1, p.~8]{DAIL-SQL}, few-shot learning 
still shows to substantially improve model performance. \cite{DAIL-SQL} shows that even one-shot learning boosts GPT-4's
execution accuracy to 80.2\%, representing a 7.9\% increase, while five-shot learning reaches 82.4\% \citep[Table 2, p.~8]{DAIL-
SQL}. 

Especially with complex queryies which can involve multiple tables, nested queries and complex joins, zero-shot approaches
often dramatically underperform $k$-shot ones. NL2SQL approaches that dont supply examples to the model during inference time
fail more frequently to generate semantically accurate SQL queries \citep{DAIL-SQL}. Notable is the leap in exact match ratio
measured by the \textsc{Spider} benchmark — jumping from 22.1\% for GPT-4 using zero-shot to 71.9\% with five-shot. The
results presented by \cite{DAIL-SQL} show signficant correlation between $k$ and the execution accuracy of $k$-shot approaches.

This effectiveness has established ICL approaches as a standard technique applied in LLM-based NL2SQL approaches. Contemporary
approaches like \textsc{XiYan-SQL, Chase-SQL and Din-SQL} all utilize variations of ICL to achieve state-of-the-art results
\citep{XiYan, CHASE, DINSQL}.

\subsubsection{Self-Correction and Iterative Refinement}

\cite{DINSQL} proposed DIN-SQL as an innovative approach to NLIDBs that rely on LLMs. DIN-SQL decomposes complex queries into
sub-parts and utilizes in-context-learning and self-correction during the generation phase. Compared to DAIL-SQL which relies
on example selection during the in-context-learning phase, DIN-SQL focuses on a refinement loop that allows the model to self-
correct errors it made during the initial generation phase — thus the model can repair schema linking, syntactic or semantic
errors. By explicitly instructing the LLM to review its work against a specfic schema, the user input and potential database
errors, DIN-SQL achieves a high execution accuracy on \textsc{Spider} with 85.3\%. Therefore DIN-SQL outperforms contemporary
approaches but is surpassed by by DAIL-SQL by 1.3\% \citep{DINSQL, DAIL-SQL}. Furthermore DIN-SQL makes observations on the
impact that the self-correction prompt can steer results significantly — \citeauthor{DINSQL} found that using \textit{generic
self-correction} (ie. assuming the query contains errors) lowers the execution accuracy by 4.2\% on \textsc{Spider} compared to
\textit{gentle self-correction} (ie. assuming nothing about the validity of the query). It was noted that the impact of the
self-correction mechanism relies on the model size, with smaller models performing better with \textit{generic self-correction}
and larger models performing better with \textit{gentle self-correction} \citep{DINSQL}. The self-correcting nature of DIN-SQL
represents a diversion from DAIL-SQL's emphasis on input optimization towards output refinement. \citeauthor{DINSQL} demonstrate
how structured introspection can play a significant role in enhancing LLM performance for formal language generation tasks.

Building upon DIN-SQL's self-correction module, \cite{MAGIC} proposed MAGIC (Multi-Agent Guideline for In-Context Text-to-SQL),
which further advances the self-correction mechanism through harnessing a set of specialized agents to automate the self-correction
prompt engineering \citep{MAGIC}. MAGIC consists of a manager agent, a correction agent and a feedback agent that collaboratively
refine LLM instructions during the refinement loop. Further MAGIC derives common faliure patterns of the initial query generation
phase from training data, allowing it to efficiently spot the most common mistakes that the model makes at generation time. This
approach represent a further advancement on \citeauthor{DINSQL}'s DIN-SQL, effectively supersetting the \textit{generic} and
\textit{gentle} correction mechanisms through an intelligent, self-adapting one \citep{MAGIC}. The autogenerated guidelines from
MAGIC yield 85.6\% execution accuracy on the \textsc{Spider} development set — representing a 5.31\% improvement over DIN-SQL's
human written correction guidelines. These results emphasize that optimized self-correction mechanisms have the ability to
significantly drive up overall system performance of NLIDBs \citep{MAGIC}.

While DIN-SQL and MAGIC focus on automated self-correction in single-turn settings, \cite{CoE-SQL} introduced the concept of
Chain-of-Editions (CoE-SQL), which addresses the unique challenges of multi-turn conversational NLIDBs. Conversational interfaces
for NL2SQL systems enable human-in-the-loop refinement. Interactive information seeking from the user has shown to be an
effective way to drive overall accuracy of the system and improve user satisfaction \citep{NALIR}. Rather than approaching
each query independently, CoE-SQL recognizes that in a conversational context, successive SQL queries usuallly require only
small and incremental modifications of the previous queries. Interactive user input is an effective measure for dealing with
ambiguous natural language queries \citep{CoE-SQL, UnnaturalQueryLanguage, NALIR}.

\subsubsection{Candidate Selection Frameworks}

Contemporary NL2SQL approaches have increasingly emphasized on the generation of query candidates and their selectiion as a
promising architecture. Candidate selection strategies have shown significant performance improvements on challenging benchmarks.
These approach acknowledge the inherent difficulty of generating perfect SQL queries in one attempt / using one generation
mechanism, even with capable LLMs and modern self-correction mechanisms.

\cite{CHASE} introduced CHASE-SQL in \citeyear{CHASE}, a framework that leverages multiple reasoning paths to generate multiple
query candidates. After the initial generation phase CHASE selects the most promising solution to the natural query input.
CHASE-SQL harnesses three different generation strategies: A divide-and-conquer approach which breaks down complex natural
language queries into multiple sub tasks that can be individually tackled, a chain-of-thought based generation approach which
inspects execution plans of SQL queries and a schema-aware generation of synthetic examples that can be used for in-context
learning \citep{CHASE}. These different generation mechanisms produce a set of query candidates that each have different
characteristics. For candidate selection CHASE harnesses a fine-tuned LLM that can do binary selection of candidates.
\citeauthor{CHASE} have demostrated that their query selection approach is more robust than apparent alternatives and yields
state-of-the-art performance with 73\% execution accuracy on \textsc{Bird} and 87.6\% on \textsc{Spider} \citep{CHASE}.

Conceptually similar work has been done by \citeauthor*{XiYan} with \textsc{XiYan-SQL} which is architected as multi-generation
ensemble strategy with better schema representation. \textsc{XiYan-SQL} integrated in-context learning alongside supervised
fine-tuning approaches to generate query candidates \citep{XiYan}. A key contribution of \cite{XiYan} is their M-Schema
representation of database schemas, which improves the models schema awareness and reduces frequent schema linking errors.
\textsc{XiYan-SQL} enhances accuracy by utilizing multiple different strategies that have complementary characteristics
during query generation to enhance the robustness of the overall system. The query generation stage utilizes both a
fine-tuned SQL generation model aswell as ICL strategies to achieve a breadth of candidate coverage. Following to the query
generation stage a self-correction stage (referred to as \textit{refinement} stage by \citeauthor{XiYan}) is utilized
to correct common errors. Lastly a selection model is used to choose the most accurate candidate that was produced
during the generation stage \citep{XiYan}. Through this sophisticated and diverse architecture \textsc{XiYan-SQL} was
able to achieve impressive results across contemporary NL2SQL benchmarks — achieving 89.65\% execution accuracy on
\textsc{Spider} and 73.34\% on \textsc{Bird}, which renders \textsc{XiYan-SQL} state-of-the-art \citep{XiYan}.

Both CHASE and \textsc{XiYan-SQL} show that diversifing candidate generation and training specialized models for candidate
selection yield state-of-the-art execution accuracy which significantly outperforms single-path generation approaches.
The success of these two approaches indicates that for increasingly complex NL2SQL tasks (such as \textsc{Spider2.0}),
the capability to generate multiple valid interpretations of the natural language query is an important stepping stone
to achieving meaningful execution accuracy. Both CHASE and \textsc{XiYan-SQL} rely on specialized candidate selection
models which renders these approaches to combine the strengths of LLMs when it comes to language understanding and
transferability with the robustness of specialized model architectures for candidate ranking and selection.

% Maybe: Agent specialization and collaboration patterns
% Maybe: Voting and selection mechanisms for query ranking

\subsubsection{Retrieval-Augumented Generation}

Retrieval-Augmented Generation (RAG) has emerged as a powerful paradigm for enhancing NL2SQL systems by integrating
external knowledge retrieval with the generative capabilities of LLMs. This technique is seen in all above introduced
papers in varying forms \citep{XiYan, CHASE, CoE-SQL, MAGIC, DINSQL, DAIL-SQL}. The most prevalent form of RAG in
NL2SQL is the encoding of the database schema into the LLM prompt for in-context-learning. This allows LLMs to be
aware of the table structures and names, foreign key relationships, primary keys etc. \cite{XiYan} proposed the
serialization of database schemas in the \textsc{M-Schema} format, a semi structured, text-based schema description.
\cite{DAIL-SQL} proposed to encode the schema using a Code Representation Prompt (CR) which refers to the encoding of
the raw SQL statements need to construct the schema. \cite{XiYan} provided an ablation study for the M-Schema which
yielded questionable results on the optimization of this approach. While the M-Schema format yielded best results when
\textsc{XiYan-SQL} was combined with GPT-4o or Claude 3.5 Sonnet, it performed worse than alternatives on DeepSeek
and Gemini models \citep{XiYan}.

A relevant optimization technique for RAG is the selection of a schema subset before encoding the schema for the model.
\textsc{Resdsql} was one of the earlier approaches to explore subset-encoding of database schemas with \cite{RetAug}
building on top of this \citep{RetAug, RESDSQL}. \cite{RetAug} introduced \textsc{ASTReS} which dynamically
retrieves database schemas and uses abstract syntax trees (ASTs) to select optimal few-shot examples for ICL. 
By pruning the ASTs down to the most relevant subset, \textsc{ASTReS} achieved the highest at-the-time (\citeyear{RetAug})
execution accuracy on \textsc{Spider} with 86.6\% indicating that subset-encoding is a sensible optimization mechanism.
\textsc{ASTReS} was combined with GRAPHIX-T5 in order to achieve this result \citep{RetAug}.

The impact of RAG when NL2SQL systems face large and complex database schemas has been particularly significant. Traditional
approaches struggle when database schemas contain hundreds or thousands of tables and columns, as the complete schema may not
fit within model context windows depending on their size. RAG-based systems address this by dynamically retrieving only the most
relevant portions of the schema based on the natural language query. The technique of subset-encoding becomes especially relevant
in enterprise environments where database schemas can be extremely large and complex. Recent benchmarks like \textsc{Spider2.0}
emphasize enterprise environments and show that existing solutions underperform in those scenarios, often reaching single digit
execution accuracy.

The work of \citeauthor{RetAug} indicates that prefiltering of the environment of language models is an effective
and promising technique that has the ability to reduce computational requirements of contemporary NL2SQL systems.
As introduced above, recent state-of-the-art systems often utilize closed-source models like Gemini, GPT-4(o), Claude 3.5/3.7
Sonnet etc which often come with massive parameter sizes (reaching hundreds of billions of parameters). \textsc{ASTReS}
demonstrated that efficient schema retrieval mechanisms enable smaller models (e.g. GRAPHIX-T5) to achieve state-of-the-art
performance against LLM based approaches like DAIL-SQL \citep{DAIL-SQL, RetAug}.

\subsubsection{Specialized LLMs and Fine-tuning}

While general-purpose LLMs demonstrated state-of-the-art natural language understanding capabilities, the research
domain of NLIDBs increasingly explored the potential of fine-tuned LLMs for NL2SQL tasks, which offer a promising
tradeof between natural language understanding (ie. breadth of the model) and concrete SQL generation capabilities
(ie. depth of the model). Multiple research works have been done on the fine-tuning of language models which
yielded a series of dedicated models for optimized SQL generation.

A significant limitation of many LLM-based NL2SQL solutions is their dependency on proprietary and closed-source
LLMs like GPT-4(o), Claude and Gemini. Whilst they are useful for initially proving the potential of LLMs on
contemporary benchmarks, this dependency introduces significant concerns related to data-privacy, high-side use 
(e.g. in classified environments), transparency of data-flow and deployment costs. To address these challenges
\cite{CodeS} have introduced \textsc{CodeS} in \citeyear{CodeS}, a series of of open-source language models dedicated
for NL2SQL tasks with parameter sizes ranging from 1B to 15B. The \textsc{CodeS} models were evaluated against
contemporary benchmarks like \textsc{Spider} and \textsc{Bird} and showed promising inference results when compared
to their closed-source counterparts. \cite{CodeS} showed that \textsc{CodeS} 7B achieves 85.4\% execution accuracy
on the \textsc{Spider} development set, outperforming both fine-tuning approaches like RESDSQL and GRAPHIX-T5-PICARD
aswell as prompting based methods DIN-SQL+GPT-4 and DAIL-SQL+GPT-4 \citep{CodeS}. The same tendency was observed
on \textsc{Bird} with \textsc{CodeS}-15B achieving 60.37\% execution accuracy, compared to 57.41\% for DAIL-SQL+GPT-4
and 55.90\% for DIN-SQL+GPT4 \citep{CodeS}. This marks a significant advancement in the open-source language model
research area, with \textsc{CodeS} reaching new state-of-the-art performance in \citeyear{CodeS}. While more recent
approaches like \textsc{XiYan-SQL} and CHASE both outperform \textsc{CodeS}, CHASE relies on proprietary models and
\textsc{XiYan-SQL} doesnt provide any information on what base models where used.

\cite{CodeS} addressed serveral critical research challenges in the NL2SQL domain and proved that open-source models
could perform competitively with proprietary models whilst maintaining a significantly smaller parameter footprint
(ie. 7B and 15B) compared to GPT-4 which is a multi-hundred-billion parameter model. This makes it feasibile to deploy
\textsc{CodeS} locally, instead of relying on an enterprise API like OpenAI's one \citep{CodeS}, therefore making it
highly practical for real-world deployments where computation resource are constrained. 

\citeauthor*{OmniSQL} published a follow-up paper in \citeyear{OmniSQL} which introduced the next-generation \textsc{OmniSQL}
models with 7B, 14B and 32B sizes, trained using synthetic data generation. \textsc{OmniSQL} achieves state-of-the-art
performance when compared to alternative LLMs - including both open-source and closed-source competitors. The models
achieve significant execution accuracy improvements on both \textsc{Spider} and \textsc{Bird}, the 7B model reaches 88.9\%
on \textsc{Spider} and 66.1\% execution accuracy on \textsc{Bird} which represents a significant (5\%+) improvement over
comparable alternatives \citep{OmniSQL}. These results were achieved without combining \textsc{OmniSQL} with advanced
in-context-learning, self-correction, retrieval-augumented-generation or candidate-selection techniques, which indicates
that even higher accuracy scores are possible.

\subsubsection{Limitations and Challenges}

Despite the impressive advancements of LLM-based approaches for NL2SQL, several significant limitations and challenges
are apparent that impact their practical implementation and real-world deployability.

\begin{enumerate}
    \item \textbf{Hallucination and Accuracy Concerns} — Since LLMs demonstrate a tendency to hallucinate, LLM-based
           NL2SQL systems face the challenge of detecting when LLMs generate plausible but incorrect SQL queries.
           The breadth of possible errors ranges from detectable errors like invalid table or columns references,
           invalid SQL syntax etc. to undetectable, slight errors in semantics, like reversing the order of
           aggregations or using the wrong aggregation method. Contrary to traditional approaches which fail visibly with
           unknown structures, LLM-based approaches might produce semantically flawed queries that execute without
           errors and yet return semantically incorrect data. As noted by \citeauthor{NL2SQLUnsolved}, ``achieving
           Enterprise-Grade NL2SQL is still far from being resolved'' even with state-of-the-art models, particularly
           when handling complex real-world database schemas and ambiguous queries \citep{NL2SQLUnsolved, Spider2}.
    \item \textbf{Computational Resource Requriements} — The resource intensity of LLM-based systems presents barriers
           to widespread adoption. High-performance state-of-the-art models require substantial computational resources
           during inference, making them expensive to deploy at scale. While smaller models exist, they typically show
           reduced performance on complex queries, creating a challenging trade-off between accuracy and resource
           efficiency. Studies like DAIL-SQL demonstrate a direct correlation between model size and performance,
           where the most accurate systems are also the most challenging to deploy economically \citep{DAIL-SQL}.
           Even state-of-the-art specialized LLMs like \textsc{OmniSQL} degrade in performance as parameter sizes
           shrink, although they maintain a significantly higher parameter efficiency, outperforming enterprise-level
           closed-source competitors. 
    \item \textbf{Data Privacy and Security Implications} — Using LLMs particularly through proprietary APIs brings
           along considerable concerns with regards to data privacy and security as sensitive data may be transferred
           to respective model vendors. Potential senstive data aswell as database schemas alongside user questions
           needs to be communicated to external parties in order to form a usable system if closed source models are
           used. Recent open-source model developments like \textsc{CodeS} and \textsc{OmniSQL} seem to mitigate this
           problem partially, but as mentioned above using local inference with LLMs brings along stark computational
           requirements.
    \item \textbf{Ambiguity Handling in Complex Scenarios} — Following previous paradigms LLMs also continue to struggle
           with effective ambiguity resolution in natural language queries over complex database schemas. These challenges
           are particularly prominent in large scale enterprise environments where similar column names exist across multiple
           tables, or domain-specific terminology has multiple potential interpretations that can result in different queries.
           Though approaches like multi-path reasoning and candidate selection show promising improvements in execution
           accuracy, they often increase inference time and complexity as they run multiple parallel inference steps and
           rank their results \citep{CHASE, XiYan}.
    \item \textbf{Competitive PLM Approaches} — Prevalent PLM-based approaches like RESDSQL and GRAPHIX-T5 achieved 
           impressive execution accuracy while using significantly fewer parameters than LLMs, making them more computationally
           efficient \citep{RESDSQL}. Despite reducing inference-time requirements, these approaches typically utilize
           re-training and domain specific fine-tuning which both limits their flexibility and introduces training costs.
           Yet for certain scenarios PLM-based approaches offer an interesting tradeoffs: Frontloading the computational effort
           can be interesting in compute-constrained environments and help to reduce ongoing costs. LLMs dont require the upfront
           cost of fine-tuning and adaption but require significantly more resource for their deployment. This indicates
           that certain systems and environments might benefit more from PLM approaches than LLM ones. Contemporary research,
           which typically focuses around state-of-the-art-performance on benchmarks, shifted primarily towards LLMs due to
           their transferability and natural language coverage.
\end{enumerate}
