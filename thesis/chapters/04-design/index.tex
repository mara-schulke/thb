


















\newpage

\section{System Design}

This chapter describes the design of \textsc{Natural}, the proposed NL2SQL system, 
addressing limitations and research gaps identified in the literature review.
The design follows a modular architecture that leverages both traditional computer
science algorithms and machine learning models to achieve robust, portable and
extensible natural language to SQL translation capabilities.

\subsection{System Architecture Overview}

The core of the proposed design is the query function $nq$ which utilizes
example selection ($\sigma$), query preprocessing, schema subsetting ($\phi$),
model inference ($\pi$), self-correction ($\rho$) and majority voting ($\nu$)
in order to convert a natural language query $q$ into a corresponding SQL query $\omega$.

% NOTE Draw diagram of nq
<pipeline-diagram>
%

This design expands on concepts from DAIL-SQL \citep{DAIL}, X, Y and Z \textcolor{red}{look up}.
The most notable characteristics of \textsc{Natural} lie in it's capabilities for continuous
learning, example selection and portability which allow the system to iteratively
adapt to unseen database schemas.


\subsection{Definitions}

The following sets are subsequently used to describe the systems capabilities: 

\begin{enumerate}
    \item $\mathcal{Q}$ – The set of all possible natural language queries
    \item $\mathcal{S}$ – The set of all possible database schemas
    \item $\mathcal{E}$ – The set of execution functions for SQL validation
    \item $\mathcal{V}$ – The embedding space of historically observed samples
    \item $\mathcal{D}$ – The space of distance indices over historically observed databases
    \item $\mathcal{C}$ – The set of all possible candidate sets $\mathcal{C}'$
    \item $\mathcal{Q}_{\mathcal{S}}$ – The set of all valid queries over the database schema $S$.
\end{enumerate}

\subsection{Query Function – $nq$}

\vspace{0.5em}

$$
nq: \mathcal{Q} \times \mathcal{S} \times E \times \mathcal{V} \times \mathcal{D} \rightarrow \Omega_{\mathcal{S}}
$$

\vspace{0.5em}

For a specific input instance, we write:

$$
nq(q, s, e, v, d) = \omega
$$

\vspace{0.5em}

where $q \in \mathcal{Q}$, $s \in \mathcal{S}$, $e \in \mathcal{E}$, $v \in \mathcal{V}$, $d \in \mathcal{D}$, and $\omega \in \Omega_{\mathcal{S}}$.

\subsection{Selection Function – $\sigma$}

The selection function $\sigma$ retrieves the most relevant examples from the
historical embedding space $\mathcal{V}$ based on semantic similarity to the
input query. This function implements the core example selection mechanism 
that subsequently enables few-shot learning for SQL generation.

$$
\sigma: \mathcal{Q} \times \mathcal{S} \times \mathcal{E} \times \mathcal{V} \rightarrow \mathcal{C}'
$$

\vspace{0.5em}

For a specific input instance, we write:

$$
\sigma(q, s, e, v) = \{(q_1, s_1, \omega_1, d_k), ..., (q_k, s_k, \omega_k, d_k)\}
$$

\vspace{0.5em}

where $q \in \mathcal{Q}$, $s \in \mathcal{S}$, $e \in \mathcal{E}$, $v \in \mathcal{V}$,
and the result is a candidate set $\mathcal{C}' \in \mathcal{C}$ containing $k$ tuples of
natural language queries, schemas, corresponding SQL queries and their combined \textit{distance}.

The function utilizes cosine similarity in the embedding space to identify examples
that are semantically closest to the input query $q$, considering both masked natural
language representations and schema distance (\textcolor{red}{Described in section XYZ}).
This approach ensures that examples are weighted by the following properties:

\begin{enumerate}
    \item \textbf{Question similarity} – 
    \item \textbf{Query similarity} –
    \item \textbf{Database similarity} –
\end{enumerate}


\subsection{Subsetting Function – $\phi$}

The subsetting function $\phi$ intelligently reduces the database schema to only include
tables, columns, and relationships that are relevant to the current query context.
This schema pruning mechanism reduces the complexity of the SQL generation task and
improves model performance by focusing on pertinent schema elements.

$$
\phi: \mathcal{C}' \times \mathcal{S} \rightarrow \mathcal{S}'
$$

\vspace{0.5em}

For a specific input instance, we write:

$$
\phi(c, s) = s'
$$

\vspace{0.5em}

where $c \in \mathcal{C}'$ is a candidate set containing selected examples, $s \in \mathcal{S}$
is the full database schema, and $s' \subseteq s$ is the reduced schema subset containing
only relevant tables and columns.

The function analyzes the selected examples in $c$ to identify which schema elements
(tables, columns, foreign key relationships) are commonly referenced in similar queries.
This analysis enables the system to steer the model's attention on the most relevant
portions of potentially large and complex database schemas, thereby improving both accuracy
and computational efficiency.

\subsection{Projection Function – $\pi$}

The projection function $\pi$ represents the core translation mechanism that converts
natural language queries into SQL statements using the selected examples and reduced schema.
This function encapsulates the language model inference process that generates candidate
SQL queries based on the provided context.

$$
\pi: \mathcal{Q} \times \mathcal{S}' \times \mathcal{C}' \rightarrow \Omega_{\mathcal{S}'}
$$

\vspace{0.5em}

For a specific input instance, we write:

$$
\pi(q, s', c) = \omega
$$

\vspace{0.5em}

where $q \in \mathcal{Q}$ is the input natural language query, $s' \in \mathcal{S}'$ is
the reduced schema subset, $c \in \mathcal{C}'$ is the set of selected examples,
and $\omega \in \Omega_{\mathcal{S}'}$ is the generated SQL query.

The function operates by constructing a prompt that combines the natural language
query, the relevant schema information, and the selected examples in a format
optimized for large language model inference. The model then generates a SQL query
that attempts to capture the semantic intent of the natural language input while
adhering to the constraints imposed by the database schema.


\subsection{Refinement Function – $\rho$}

The refinement function $\rho$ implements a self-correction mechanism that iteratively
improves generated SQL queries by identifying and correcting syntax errors,
semantic inconsistencies, and execution failures. This function enables the system
to learn from its mistakes and produce higher-quality outputs through automated
feedback loops.

$$
\rho: \mathcal{Q} \times \mathcal{S}' \times \mathcal{E} \times \Omega_{\mathcal{S}'} \rightarrow \Omega_{\mathcal{S}'}
$$

\vspace{0.5em}

For a specific input instance, we write:

$$
\rho(q, s', e, \omega_{prev}) = \omega_{refined}
$$

\vspace{0.5em}

where $q \in \mathcal{Q}$ is the original natural language query, $s' \in \mathcal{S}'$
is the relevant schema subset, $e \in \mathcal{E}$ is the execution environment for
validation, $\omega_{prev} \in \Omega_{\mathcal{S}'}$ is the previously generated SQL
query, and $\omega_{refined} \in \Omega_{\mathcal{S}'}$ is the improved SQL query.

The function operates by executing the candidate query against the database, analyzing
any errors or unexpected results, and then prompting the language model to generate
an improved version based on the identified issues. This iterative refinement
process continues until a valid, executable query is produced or a maximum number
of refinement attempts is reached.

\subsection{Voting Function – $\nu$}

The voting function $\nu$ implements a consensus mechanism that selects the most
reliable SQL query from multiple candidate solutions generated through the pipeline.
This function enhances robustness by leveraging the wisdom of multiple generation
attempts to identify the most likely correct answer.

$$
\nu: \mathcal{C} \times \mathcal{E} \rightarrow \Omega_{\mathcal{S}'}
$$

\vspace{0.5em}

For a specific input instance, we write:

$$
\nu(C, e) = \omega_{consensus}
$$

where $C \in \mathcal{C}$ is a set of candidate SQL queries, $e \in \mathcal{E}$ is
the execution environment for validation, and $\omega_{consensus} \in \Omega_{\mathcal{S}'}$
is the selected consensus query.

\vspace{0.5em}

The voting function $\nu$ implements a majority voting algorithm similar to that
described by OmniSQL \citep{OmniSQL}, where the result that appears most frequently
across multiple generation attempts is deemed to be the most likely correct answer.
The function first validates all candidates for syntactic correctness and executability,
then applies frequency-based selection among the valid candidates. In cases where
no clear majority exists, the function may apply additional heuristics such as query
complexity or execution performance to determine the final selection.

\subsection{Masked Embedding Space – $\mathcal{V}$}

Building upon DAIL-SQL's masked example selection, \textsc{Natural} utilizes an
embedding space to both traverse the masked natural language query $q'$
and the respective masked SQL query $\omega'$. This allows $\sigma$ to use
cosine distance to search through historically observed question answer pairs;
eg, from previous user interactions or prevalent training datasets like
\textsc{Spider}, \textsc{Bird} or \textsc{SynSQL-2.5m}.

\subsubsection{Natural Language Masking}

The natural language masking process replaces all words which are not deemed
\textit{structurally relevant} with the \texttt{<mask>} token. All words $w$
in a predefined whitelist $\mathcal{W}$ are determined structually relevant.
Choosing a sound set $\matchal{W}$ is described in chapter 5.

This process enables the embedding space to capture semantic patterns independent
of specific database artifacts and domain terminology, improving the generalizability
of example selection across different schemas.

This abstraction allows the system to recognize that queries like:

\begin{verbatim}
Show all customers with the firstname John
List all products with the name iPhone
\end{verbatim}

in fact share similar semantic structures, even though they operate on different
entities. Given $\matchal{W} = \{\text{List}, \text{Show}, \text{all}, \text{the}, \text{with}, \text{name}\}$
the masking would result in:

\begin{verbatim}
Show all <mask> with the <mask> <mask>
List all <mask> with the name <mask>
\end{verbatim}

Embeddings can now be computed for the masked versions of the natural language queries.

\subsubsection{Query Masking}

Query masking applies similar abstraction principles to the SQL queries themselves.
This enables the system to identify SQL patterns and structures rather than
memorizing specific database schemas.

The SQL masking process transforms all identifiers to \texttt{<mask>} and all literals
to \texttt{<value>}. For example, the query:

\begin{verbatim}
SELECT name FROM customers WHERE age > 25
\end{verbatim}

gets masked to:

\begin{verbatim}
SELECT <mask> FROM <mask> WHERE <mask> > <value>
\end{verbatim}

This abstraction enables pattern recognition across different domains while
preserving the essential logical structure of the original SQL statements.









\subsection{Algorithm Design}

\begin{algorithm}

\caption{The $nq$ function}\label{alg:cap}
\begin{algorithmic}[1]
\Require $q \in \mathcal{Q}$, $s \in \mathcal{S}$, $e \in \mathcal{E}$, $v \in \mathcal{V}$, $d \in \mathcal{D}$
\Require $k \in \{ n \in N\ |\ 2n + 1 \}, k \geq 1$
\State $S \gets nsel(q,s,e,v)$             \Comment{Example selection is performed}
\State $C \gets \emptyset$                 \Comment{The set of query candidates}
\While{$|C| < k$}                          \Comment{While not enough candidates exist}
    \State $sub \gets nsub(C, s)$          \Comment{Identify relevant schema subset}
    \State $nc \gets nprompt(S, q, sub)$   \Comment{Prompt model for next candidate}

    \If{$e(nc)$}                           \Comment{Evaluate the next candidate}
        \State $c \gets \{nc\} \cup c$     \Comment{Extend the candidates}
    \EndIf

    \State $ref \gets nref(q, sub, e, v)$  \Comment{Refine via self correction}

    \If{$e(ref)$}                          \Comment{Evaluate the refined candidate}
        \State $c \gets \{ref\} \cup c$    \Comment{Extend the candidates}
    \EndIf
\EndWhile
\State \Return $nvote(C, e)$               \Comment{Return most prevalent result}
\end{algorithmic}

\end{algorithm}


