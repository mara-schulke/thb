\begin{abstract}

As current state-of-the-art NL2SQL systems largely rely on proprietary models,
open-source models like OmniSQL recently emerged and show competitive NL2SQL
performance on prevalent benchmarks. As proprietary models are primarily
accessible through external APIs, they require transmitting database schemas
and contents across the wire. This induces serious data sovereignty, data
prvivacy and reliability concerns for sensity industries such as healthcare,
finance and aerospace \& defense. A fundamental challenge in NL2SQL is the gap
between theoretical research performance and production-ready systems that
leverage open-source models for sensitive and potentially air gapped
environments.

This thesis presents \textsc{Natural}, a sovereign NL2SQL system built
on the open-source OmniSQL-7B model. The system employs a five-stage pipeline:
graph-based example selection using Wasserstein-Weisfeiler-Lehman kernels
($\sigma$), schema subsetting ($\phi$), few-shot query projection ($\pi$),
self-refinement ($\rho$), and consensus voting ($\nu$). Implemented in Rust,
\textsc{Natural} enables both a deployment as a database extension or
standalone inference service without any external dependencies.

Evaluations on the \textsc{Spider} and \textsc{Bird} benchmarks demonstrate a
81.0\% execution accuracy on \textsc{Spider} (dev), 81.0\% on \textsc{Spider}
(test) and 53.8\% on Bird, representing +3.1pp and +21.4pp improvements over
baseline respectively. An ablation study reveals synergistic component
interactions, with example selection providing the largest gains (+19.8pp on
Bird). Synthetic training examples outperform smaller reference datasets by
5.4pp, validating structure-aware selection. A 28.1pp reproducibility gap
between the published model performance of \textsc{OmniSQL} and local
measurements exposes substantial evaluation challenges for LLM-based NL2SQL
research. The resulting candidate latency of 7-16 seconds on \textsc{Spider}
queries suits analytical workloads but limits interactive applications.

This work further demonstrates that open-source NL2SQL systems can achieve
competitive performance while maintaining portability and sovereignty,
establishing a foundation for secure database access in enterprise and
government environments.

\end{abstract}

