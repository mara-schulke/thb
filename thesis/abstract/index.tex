\begin{abstract}

Current state-of-the-art NL2SQL systems largely rely on proprietary models
for inference, which in turn poses serious data sovereignty, data privacy and
reliability concerns on their users as they require transmitting database
schemas and contents across the wire. This applies especially to sensitive
industries such as healthcare, finance, aerospace and defense. Recently
open-source modele like \textsc{OmniSQL} emerged which show competitive NL2SQL
performance on prevalent benchmarks, addressing the fundamental gap between
theoretical research performance and production-ready systems that leverage
open-source models for sensitive air gapped environments.

This thesis presents \textsc{Natural}, a sovereign NL2SQL system built
on the open-source \textsc{OmniSQL}-7B model. The system employs a five-stage
pipeline: graph-based example selection using Wasserstein-Weisfeiler-Lehman
kernels ($\sigma$), schema subsetting ($\phi$), few-shot query projection
($\pi$), self-refinement ($\rho$), and consensus voting ($\nu$). Implemented in
Rust, \textsc{Natural} enables both a deployment as a database extension or
standalone inference service without any external dependencies.

Evaluations on the \textsc{Spider} and \textsc{Bird} benchmarks demonstrate a
81.0\% execution accuracy on \textsc{Spider} (dev), 81.4\% on \textsc{Spider}
(test) and 53.8\% on Bird, representing +3.1pp and +21.4pp improvements over
baseline respectively. An ablation study reveals synergistic component
interactions, with example selection providing the largest gains (+19.8pp on
Bird). Synthetic training examples outperform smaller reference datasets by
5.4pp, validating structure-aware selection. A 28.1pp reproducibility gap
between the published model performance of \textsc{OmniSQL} and local
measurements highlighted substantial evaluation challenges in LLM-based NL2SQL
research. The resulting candidate latency of 7-16 seconds on \textsc{Spider}
queries suits analytical workloads but limits interactive applications.

This work demonstrates that open-source NL2SQL systems can achieve
competitive performance while maintaining portability and sovereignty,
establishing a foundation for secure database access in enterprise and
government environments.

\end{abstract}

