\begin{abstract}

Current state-of-the-art NL2SQL systems largely rely on proprietary models
for inference, which in turn poses serious data sovereignty, data privacy and
reliability concerns on their users as they require transmitting database
schemas and contents across the wire. This applies especially to sensitive
industries such as healthcare, finance, aerospace and defense. Recently
open-source models like \textsc{OmniSQL} emerged which show competitive NL2SQL
performance on benchmarks and thus address the gap between theoretical research
systems and production-ready systems that leverage open-source models for
sensitive air gapped environments.

This thesis introduces \textsc{Natural}, a sovereign NL2SQL pipeline built
around the open-source \textsc{OmniSQL} model series. The system consists of
five components: graph-based example selection using
Wasserstein-Weisfeiler-Lehman kernels ($\sigma$), schema subsetting ($\phi$),
few-shot query projection ($\pi$), self-refinement ($\rho$), and consensus
voting ($\nu$). \textsc{Natural} can be deployed as a database extension or as
a inference service without external dependencies.

Benchmark results on \textsc{Spider} and \textsc{Bird} demonstrate a 81.0\%
execution accuracy on \textsc{Spider} (dev), 81.4\% on \textsc{Spider}
(test) and 53.8\% on \textsc{Bird}, representing +3.1pp and +21.4pp
improvements over the locally reproduced \textsc{OmniSQL}-7B-gguf baseline
respectively. The ablation reveals that in-context learning is the
component impacting the performance the most. Adding example selection
to the pipeline providing the largest execution accuracy improvements with
+19.8pp on \textsc{Bird}). Other pipeline components contribute marginally to
this outcome. Furthermore, synthetic examples outperformed reference training
datasets by +5.4pp on \textsc{Bird}, which suggests that the structural
similarity of examples and the database has a greater impact than the domain
similarity. A 27.1pp gap between the officially published \textsc{OmniSQL}
results and the locally reproduced measurements highlight reproduction
challenges in LLM-based NL2SQL systems and limit the ability of this thesis to
perform absolute performance comparisons and derive definite conclusions. The
7-16 seconds candidate latency of \textsc{Natural} on \textsc{Spider} enables
real-world adoption for analytical workloads but is still limited for
interactive use cases.

The results of this work demonstrate that pipeline composition yield
substantial (relative) execution accuracy improvements within controlled
environments. \textsc{Natural} thus builds a foundation for sovereign NL2SQL
systems in privacy and security sensitive contexts which prevents reliance on
proprietary models.

\end{abstract}
