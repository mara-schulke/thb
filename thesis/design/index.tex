\section{System Design}

This chapter describes the design of \textsc{Natural}, the proposed NL2SQL system, 
addressing limitations and research gaps identified in the literature review.
The design follows a modular architecture that leverages both traditional computer
science algorithms and machine learning models to achieve robust, portable and
extensible natural language to SQL translation capabilities.

\subsection{Overview}

The \textsc{Natural} system implements a multi-stage pipeline that transforms natural
language questions into executable SQL queries through example selection,
schema subsetting, and iterative refinement. The proposed architecture consists
of five core components:

\begin{enumerate}
    \item \textbf{Example Selection} – Identifies semantically and structurally similar
        historical examples using masked embeddings and schema distances.
    \item \textbf{Schema Subsetting} – Reduces database schema complexity by focusing
        on relevant tables and relationships
    \item \textbf{Query Generation} – Few-shot learning with fine tuned models translate natural language queries to SQL
    \item \textbf{Self-Correction} – Iteratively refines queries through execution feedback and error analysis
    \item \textbf{Consensus Voting} – Selects the most reliable result from multiple
        generation attempts
\end{enumerate}

The system processes queries through the following pipeline:

% TODO: Make a nice diagram for this
\begin{verbatim}
  Natural Language Query 
→ Sketch Generation
→ Example Selection
→ k-times [
    → Schema Subsetting
    → Few-Shot Generation
    → Self-Correction
  ]
→ Consensus Voting
→ SQL Query
\end{verbatim}

This design builds upon concepts from DAIL-SQL \citep{DAIL} and incorporates novel
contributions in schema-aware example selection and graph-based structural similarity.

\subimport{}{initialization}

\subimport{}{components}

\subimport{}{composition}
