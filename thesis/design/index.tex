\section{System Design}

This chapter describes the design of \textsc{Natural}, our proposed NL2SQL system, 
addressing limitations and research gaps identified in the literature review.

\textsc{Natural} is architected as a pipeline that transforms natural
language questions into SQL queries using example selection ($\sigma$),
schema subsetting ($\phi$), and self refinement ($\rho$) and voting ($\nu$).
The proposed system consists of five components:

\begin{enumerate}
    \item \textbf{Example Selection} $\sigma$ – Identifies semantically and structurally similar examples using cosine similarity and schema distance.
    \item \textbf{Schema Subsetting} $\phi$ – Reducing schema complexity by subsetting the schema to the relevant subset of tables and relationships.
    \item \textbf{Query Projection} $\pi$ – Few-shot learning with a finetuned model to project natural language queries to SQL statements.
    \item \textbf{Self Refinement} $\rho$ – Self refinement of generated SQL statements through execution feedback and error analysis.
    \item \textbf{Voting} $\nu$ – Self consensus voting mechanism to choose the most likely result from multiple generation attempts.
\end{enumerate}

The system processes queries using the following algorithm:

% TODO: Make a nice diagram for this

\begin{verbatim}
  Natural Language Query 
→ Sketch Generation
→ Example Selection
→ k-times [
    → Schema Subsetting
    → Few-Shot Generation
    → Self-Correction
  ]
→ Consensus Voting
→ SQL Query
\end{verbatim}

This design largely builds upon few-shot learning concepts from DAIL-SQL \citep{DAIL-SQL}
and OmniSQL \citep{OmniSQL} but incorporates novel contributions in schema-aware example
selection by harnessing a graph-based structural similarity metric.

\subimport{}{initialization}
\subimport{}{functions}
\subimport{}{composition}
