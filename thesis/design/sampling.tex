\subsection{Sampling}\label{design:sampling}

Building upon DAIL-SQL's masked example selection, \textsc{Natural} utilizes an
embedding space to both traverse the masked natural language query $q'$
and the respective masked SQL query $\omega'$. This allows $\sigma$ to use
cosine distance to search through historically observed question answer pairs;
eg, from previous user interactions or prevalent training datasets like
\textsc{Spider}, \textsc{Bird} or \textsc{SynSQL-2.5m}.

\subsubsection{Natural Language Masking}

The natural language masking process replaces all words which are not deemed
\textit{structurally relevant} with the \texttt{<mask>} token. All words $w$
in a predefined whitelist $\mathcal{W}$ are determined structually relevant.
% TODO: Fix reference
Choosing a sound set $\mathcal{W}$ is described in section~\ref{implementation:whitelisting}.

This process enables the embedding space to capture semantic patterns independently
of specific database artifacts and domain terminology, improving the generalizability
across different schemas and domains.

This masking allows \textsc{Natural} to recognize that queries like:

\begin{verbatim}
Show all customers with the firstname John
List all products with the name iPhone
\end{verbatim}

in fact share similar semantic structures, even though they operate on different entities.
Given $\mathcal{W} = \{\text{List}, \text{Show}, \text{all}, \text{the}, \text{with}, \text{name}\}$

the masking would result in:

\begin{verbatim}
Show all <mask> with the <mask> <mask>
List all <mask> with the name <mask>
\end{verbatim}

Embeddings can now be computed for the masked versions of the natural language queries.

\subsubsection{Query Masking}

Query masking applies similar abstraction principles to the corresponding SQL queries.
This enables \textsc{Natural} to identify SQL patterns and structures.

The SQL masking process transforms all identifiers to \texttt{<mask>} and all
literals to \texttt{<value>}. Thus, the query:

\begin{verbatim}
SELECT name FROM customers WHERE age > 25
\end{verbatim}

gets masked to:

\begin{verbatim}
SELECT <mask> FROM <mask> WHERE <mask> > <value>
\end{verbatim}

This abstraction enables pattern recognition across different domains while
preserving the essential logical structure of the original SQL statements,
thus focusing on the kind of query, rather than the specific instance.

\subsubsection{Sample Embedding Space – $\mathcal{V}$}


\subsubsection{Schema Distance Index – $\mathcal{D}$}

The schema distance index is a set of id and distance tuples $\{(s_1, d_1), ..., (s_i,d_i)\}$
where $s$ is the database schema name and $d$ is the distance to the
\textit{current} schema. Thus a schema distance index has to be computed for each database schema
\textsc{Natural} is running on.



























