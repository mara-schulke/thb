\documentclass{article}

\usepackage[a4paper]{geometry}
\usepackage[english]{babel}
\usepackage[utf8]{inputenc}
\usepackage[T1]{fontenc}
\usepackage{graphicx}
\usepackage{fancyhdr}
\usepackage{amsmath}
\usepackage{xcolor}
\usepackage{float}
\usepackage{hyperref}
\usepackage{titlesec}

\graphicspath{{./images/}}

\pagestyle{fancyplain}
\fancyhf{}
\lhead{\fancyplain{}{Mara Schulke} }
\rhead{\fancyplain{}{\today}}
\cfoot{\fancyplain{}{\thepage}}

\newcommand{\figuresource}[1]{
	\begin{center}Quelle: {#1}\end{center}
}

\titleformat{\chapter}{\normalfont\Large\bfseries}{\thechapter.}{20pt}{\Large}
\titleformat{\section}{\normalfont\large\bfseries}{\thesection}{1em}{}
\titleformat{\subsection}{\normalfont\normalsize\bfseries}{\thesubsection}{1em}{}
\titleformat{\subsubsection}{\normalfont\small\bfseries}{\thesubsubsection}{1em}{}

\begin{document}

\begin{titlepage}
    \begin{center}
        \begin{Large}
            Brandenburg University of Technology \\[1em]
        \end{Large}
        IT Security \\
        Computerscience and Media \\
        Prof. Dr. Oleg Lobachev \\
        Florian Eich
    \end{center}

    \vfill

    \begin{center}
        \Large{Natural Language Queries using Large Language Models}\\[0.5em]
        \large{Bachelor Thesis}\\[1em]
        
        \begin{normalsize}
            Summer semester 2025\\[0.25em]
            \today
        \end{normalsize}
    \end{center}

    \vfill

    \begin{center}
        Mara Schulke – Matr-Nr. 20215853
    \end{center}
\end{titlepage}

\begin{abstract}
This thesis explores the integration of large language models (LLMs) into PostgreSQL database systems in order
to make the database accessible via natural language instead of the postgres SQL dialect. The research focuses
on implementation strategies, performance optimization, and practical applications of this concept.
\end{abstract}

\tableofcontents

\listoffigures

\section*{List of Abbreviations}
\begin{tabular}{ll}
GPT & Generative Pretrained Transformer \\
SQL & Structured Query Language \\
API & Application Programming Interface \\
LLM & Large Language Model \\
DBMS & Database Management System \\
NL2SQL & Natural Language to SQL \\
\end{tabular}

\newpage

% Introduction
\section{Introduction}
\subsection{Problem Statement and Motivation}

Database systems represent a backbone of modern computer science, allowing for rapid advancements
whilst shielding us from the problem categories that come along with managing and querying large amounts
of, usually structured, data efficiently. However, most Database Management Systems (DBMS) have
traditionally required specialized knowledge, usually of the Structured Query Language (SQL), in order
to become useable. Whilst this barrier may be percieved differently across diverse usergroups it
represents a fundamental misalignment between end-user goals (e.g. analysts, researchers, domain experts
etc.) and the underlying DBMS, thus often requiring software engineering efforts in order to reduce this friction.

This barrier is the reason entire classes of software projects exists (for example, admin / support panels),
data analytics tools etc. which therefore introduce significant churn and delay between the implementation
of a database system and reaching the desired end user impact. Often these projects span multiple years, require
costly staffing and yield little to no novel technical value.

Emerging technologies such as Large Language Models (LLMs) have proven themselves as a sensible tool for bridging
fuzzy user provided input into discrete, machine readable formats. Prominent models in this field have demostrated
outstanding capabilities that enable computer scientists to tackle new problem classes, that used to be
challenging / yielded unsatisfying results with discrete programming approaches.

This thesis is exploring ways to overcome the above outlined barrier using natural language queries, so that domain experts,
business owners, support staff etc. are able to seamlessly interact with their data, essentially eliminating the
requirement of learning SQL (and its pitfalls). By translating natural language to SQL using Large Language Models
this translation becomes very robust (e.g. against different kinds of phrasing) and enables novel applications
in how businesses, researchers and professionals interact with their data — it represents a fundamental shift 
(ie. moving away from SQL) towards a more inclusive and data driven world. 


\subsection{Objectives of the Thesis}

This thesis aims to address the aforementioned challanges when it comes to database accessibility.
The following objectives are the core research area of this thesis:

\begin{enumerate}
    \item Develop a database extension that can translate natural language queries into semantically
          accurate SQL queries using Large Language Models.
    \item To evaluate the effectiveness and feasibility of different Models aswell as prompt engineering
          techniques in order to improve the performance of the system.
    \item Identify and address issues when it comes to handling amibguous, complex and domain specific user input.
    \item Benchmark the performance of the implementation against common natural language to SQL (NL2SQL) benchmarks.
    \item Idenitfy potential usecases for real world scenarios that could deliver a noticable upsides to users.
    \item Analyze the short commings and limitations of this approach and propose potential solutions to overcome them.
\end{enumerate}  

\subsection{Research Questions}
\subsection{Methodological Approach}
\subsection{Structure of the Thesis}

\newpage

% Theoretical Foundations
\section{Theoretical Foundations}

\subsection{Generative Pretrained Transformers (GPT)}
\subsubsection{Architecture and Functionality}
\subsubsection{Training and Fine-tuning}
\subsubsection{Application Areas}

\subsection{PostgreSQL as a Database System}
\subsubsection{Architecture of PostgreSQL}
\subsubsection{Extension Capabilities}
\subsubsection{PostGIS and Other Extensions as Examples}

\subsection{Embedding AI Models in Database Systems}
\subsubsection{Current Approaches and Solutions}
\subsubsection{Technical Challenges}
\subsubsection{Benefits of Integration}

\newpage

% Conceptual Design
\section{Conceptual Design of GPT Embedding in PostgreSQL}

\subsection{Requirements Analysis}
\subsubsection{Functional Requirements}
\subsubsection{Non-functional Requirements}

\subsection{Architecture Design}
\subsubsection{Interface Design}
\subsubsection{Data Model}
\subsubsection{Integration into PostgreSQL}

\subsection{Technical Implementation Strategies}
\subsubsection{Foreign Data Wrapper}
\subsubsection{Extension Using C/C++}
\subsubsection{PL/Python or Other Procedural Languages}
\subsubsection{Comparison of Approaches}

\newpage

% Implementation
\section{Implementation}

\subsection{Development Environment and Tools}

\subsection{Integration of the GPT Model}
\subsubsection{Model Selection and Optimization}
\subsubsection{API Connection or Local Embedding}

\subsection{Development of the PostgreSQL Extension}
\subsubsection{SQL Functions for GPT Interactions}
\subsubsection{Data Type Conversion and Processing}
\subsubsection{Error Handling and Logging}

\subsection{Optimization}
\subsubsection{Performance Tuning}
\subsubsection{Memory Usage}
\subsubsection{Parallelization}

\newpage

% Evaluation
\section{Evaluation}

% https://github.com/petavue/NL2SQL-Benchmark/blob/main/results/Report/NL2SQL%20Benchmark%20Report.pdf

\subsection{Test Environment and Methodology}

\subsection{Performance Tests}
\subsubsection{Latency}
\subsubsection{Throughput}
\subsubsection{Scalability}

\subsection{Use Cases}
\subsubsection{Natural Language Queries}
\subsubsection{Text Generation Within the Database}
\subsubsection{Semantic Search and Text Classification}

\subsection{Comparison with Alternative Approaches}

\newpage

% Discussion
\section{Discussion}

\subsection{Interpretation of Results}
\subsection{Limitations of the Implementation}
\subsection{Ethical and Data Privacy Considerations}
\subsection{Potential Future Developments}

\newpage

% Summary and Outlook
\section{Summary and Outlook}

\subsection{Summary of Results}
\subsection{Addressing the Research Questions}
\subsection{Outlook for Future Research and Development}

\newpage

\begin{thebibliography}{99}
\bibitem{ref1} Author, A. (Year). Title of the reference. Journal/Publisher, Volume(Issue), Pages.
% Add more references as needed
\end{thebibliography}

\newpage

\appendix
\section*{Appendix}
\subsection*{Installation Guide}
\subsection*{API Documentation}
\subsection*{Code Examples}
\subsection*{Test Data and Results}

\end{document}
