\documentclass[journal]{IEEEtran}

\usepackage[a4paper,top=5cm,bottom=5cm]{geometry}
\usepackage[ngerman]{babel}
\usepackage[utf8]{inputenc}
\usepackage[T1]{fontenc}
\usepackage{listings}
\usepackage{hyperref}
\usepackage{graphicx}
\usepackage{xcolor}

\pagenumbering{Roman}

\begin{document}

\markboth{THB – Ethik in der IT Sicherheit – Mara Schulke — 20215853}{}

\begin{onecolumn}

\begin{center}
	\Large{Geistiges Eigentum im Zeitalter der Informationstechnik}\\
	\vspace{18pt}\hline
\end{center}

\tableofcontents
    
\section{Einleitung}

Das geistige Eigentum und die Sicherstellung von Eigentumsrechten sind von elementarer 
Bedeutung für unsere Gesellschaft und ihre Ordnung. Das Schützen der Werke einzelner 
Individuen oder ganzer Organisationen ist eine fundamentale ethische Verpflichtung, der 
die Gesellschaft während der voranschreitenden Digitalisierung gerecht werden muss. Der 
technologische Fortschritt darf nicht zur Verwahrlosung des Eigentumsrechts führen, aber 
digitales Eigentum lässt sich gleichzeitig deutlich schwerer kontrollieren als physisches 
Eigentum. Es ist ein komplexer Balanceakt zwischen dem Schutz von Einzelnen und dem Wohl 
der Mehrheit.

Bei der Betrachtung von Schutzregelungen für Eigentum ergeben sich interessante
Fragestellungen. Zum Beispiel stellt sich die Frage, ob in bestimmten Situationen 
Ausnahmen von diesem Schutz gemacht werden dürfen oder müssen, wenn dies dem Gemeinwohl 
der Gesellschaft dient?

\section{Einfluss der Informationstechnik}

Die Digitalisierung hat einen erheblichen Einfluss auf unseren Umgang 
mit geistigen und digitalen Eigentum. 
Der technologische Fortschritt hat in unserer Gesellschaft zu einer weitreichenden
Digitalisierung von Waren geführt – viele Gegenstände die früher 
nur in physischer Form und dementsprechend nur begrenzt erhältlich waren, sind nun 
jederzeit auf Abruf verfügbar und lassen sich unbegrenzt oft vervielfältigen. 
Hierdurch sind ganze Teilbereiche der Kriminalität im Bezug auf den digitalen Bereich 
entstanden, wie zum Beispiel die Piraterie oder die illegale Verbreitung von
lizensierten / gekauften Inhalten.\cite{oe-it}

Diese Veränderungen werfen neue Fragen und Herausforderungen auf wie zum Beispiel 
die Notwendigkeit für einen Kopierschutz, Lizenzierungen von Software etc.
Mit diesen neuen Herausforderung bring der Fortschritt allerdings auch 
neue Möglichkeiten zur Sicherung des Eigentums in der digitalen Welt mit sich.
Konzepte wie der Einsatz von Kryptographie und Plagiatserkennungs-Algorithmen 
eröffnen Möglichkeiten, die Integrität und Nachvollziehbarkeit des geistigen und 
digitalen Eigentums sicherzustellen.

Es ist wichtig diese Probleme und Herausforderungen ernst zunehmen um die Wahrung von 
Ideen und geistigem Eigentum heute und künftig sicherzustellen. Hier zu wurde im Jahre 
1967 die weltweite Organisation WIPO (World Intellectual Property Organization) gegründet. 
\cite{wipo}

\section{Ethische Betrachtung von Eigentum}

Bei der ethischen Betrachtung dieses Themas stellen sich Fragen im Bezug auf die 
Verantwortung, wie zum Beispiel wer ist für die Sicherstellung der Eigentumsrechte
im digitalen Raum verantwortlich?

Im Falle von Musik ist ein bekannter Anbieter der Musik-Streaming-Dienst Spotify: Wer ist 
in diesem Fall für Sicherstellung von Urheberrechten verantwortlich und wo ziehen sich 
Verantwortungsbereiche durch diese Thematik? Ist bei der Verhinderung von Raubkopien 
das Unternehmen, dass die Plattform zur Verbreitung bereitstellt, der Gesetzgeber, oder 
das Individuum, das die Plattform zur Verbreitung verwendet gefragt? In der Realität zeigt 
sich, dass die Wahrung des geistigen Eigentums im Verantwortungsbereich von allen drei 
Akteuren liegt. Die Plattform muss die Verletzung feststellen – ein gutes Beispiel ist 
hier YouTube, die die Monetarisierung von urheberrechtlich geschützten Inhalten verbieten. 
Der Gesetzgeber muss einen rechtlichen Rahmen stellen um strafrechtlich gegen die 
Verletzung von Urheberrechten vorzugehen – und jedes Individuum trägt einen Teil 
Eigenverantwortung keine Urheberrechtsverletzung vorzunehmen, und dessen eigene Werke 
so gut es geht davor zu schützen.

Die zugrundeliegende ethische Fragestellung führt auf das gesellschaftliche Wertesystem 
zurück und der Umgang einer jeden Gesellschaft ist abhängig davon, wie sehr Fairness, 
Respekt und Gerechtigkeit in diesem einen Platz finden.

Eine Anerkennung der Rechte der Hersteller und Besitzer stehen der Problematik der 
gerechten Verteilung und dem Grundprinzip der Chancengleichheit gegenüber. Bildet sich
konzentriertes Besitztum innerhalb einer Gesellschaft hat dies ab einem gewissen Grat
einen Einfluss auf die Machtverteilung – sowohl in der physischen als auch in der 
digitalen Welt. Ein gutes Beispiel hierfür sind die Registrare von Domains im Internet, 
oder Firmen die über eine immense Rechenleistung verfügen. Die Entscheidungsposition in
die diese Akteure durch ihr Eigentum geraten bringt ein hohes Maß an Verantwortung mit 
sich, um eine möglichst ausgeglichene und demokratische Gesellschaft beizubehalten.

Die ethische Betrachtung des Themas Eigentum im Zeitalter der Informationstechnik 
erfordert daher eine tiefgreifende Auseinandersetzung mit ethischen Prinzipien wie 
Machtverteilung, Chancengleichheit und Gerechtigkeit in der digitalen Welt.

\section{Dilemmata}

Das Konzept von digitalen Besitztümern wirft eine Reihe von Konflikten und Dilemmata bei 
näherer Betrachtung auf. Hierbei fallen das Thema Datenschutz (und damit einhergehende 
Besitzrechte über Daten) und Privatsphäre als ein Kernthema heraus. Mit der immer größer
werdenden Datenspur im Internet und der Verlagerung von physischen Gütern (zum Beispiel 
Fotoalben) hinein in die digitale Welt werden Fragen werden Fragen aufgeworfen, wie diese
Spur unter Kontrolle zu behalten ist. Gerade im Falle des verscheiden einer Person ist die 
rechtliche Lage zur Behandlung der Nachlass-Angelegenheiten noch ungeklärt beziehungsweise
uneinheitlich. \cite{law-journal} Hier steht der Gesetzgeber in der Pflicht, ein 
einheitliches Regelwerk aus Richtlinien und Gesetzen zu schaffen, um den Umgang mit 
digitalen Daten und Eigentumsfragen lückenlos zu klären.

Der Zugang zu Informationen und Wissen stellt eine weitere problematische Dimension der
Digitalisierung dar. Obwohl es mit dem Internet eine generelle Erleichterung gab, an 
Informationen und Wissen zu gelangen, bestehen noch immer Kluften, was die Verfügbarkeit 
und den Zugang zu Ressourcen und Informationen angeht. Diese Ungleichheit für zu einer 
Fragmentierung der Gesellschaft in zwei Gruppen: Diejenigen die Zugang zu Bildung und 
Wissen haben, und denjenigen, die davon ausgeschlossen sind.

\section{Schlussfolgerung}

Insgesamt zeigt sich, dass Eigentum im digitalen Raum mit einer Reihe von 
Herausforderungen für die Gesellschaft als Ganzes einhergeht. Datenschutz, Urheberrechte 
und die gerechte Verteilung von Zugangsmöglichkeiten sind Fragen, denen sich Unternehmen 
und Gesetzgeber stellen müssen. Es ist von fundamentaler Bedeutung für unser Wertesystem 
digitales Eigentum zu schützen und gleichzeitig eine offene Wissenskultur zu fördern.

Geistiges Eigentum und technologischer Fortschritt sind eng miteinander verwebt: Künftige 
Entwicklungen werden weitere Herausforderungen mit sich bringen (Beispielsweise generative 
KI) aber ebenso neue Lösungen zur Sicherung von Urheberrechten und Demokratisierung von 
Wissen bereitstellen. Es ist wichtig, dass sich die Gesellschaft, Unternehmen und einzelne 
Individuen der Herausforderungen bewusst sind und daran arbeiten, eine gerechtere Zukunft 
zu schaffen. 

\begin{thebibliography}{1}

\bibitem{oe-it} G. G. Goldacker. (2019, Dezember). {\em ``Vom Eigentum zum digitalen Nutzungsrecht''}.
Kompetenzzentrum Öffentliche IT. [Online]. Verfügbar:
\url{https://www.oeffentliche-it.de/-/vom-eigentum-zum-digitalen-nutzungsrecht}

\bibitem{wipo} WIPO. (1979, September).
{\em ``Convention Establishing the World Intellectual Property Organization''}.
WIPO [Online]. Verfügbar: \url{https://www.wipo.int/wipolex/en/text/283854/}

\bibitem{law-journal} K. S. Strutin. (2011, Sommer). {\em``The Digital Estate''}.
NYSBA Trusts and Estates Law Section Newsletter. Vol 44. Nr 2. Seite 20. Verfügbar:
\url{https://nysba.org/NYSBA/Publications/Section\%20Publications/Trusts\%20and\%20Estates/
PastIssues2000present/Summer2011/Summer2011Assets/TENewsSum11.pdf#page=20}

\end{thebibliography}

\end{onecolumn}

\end{document}