\documentclass[aspectratio=169]{beamer}

\usepackage[ngerman]{babel}
\usepackage[utf8]{inputenc}
\usepackage[T1]{fontenc}
\usepackage{hyperref}

\usecolortheme{seagull}
\usefonttheme{serif}

\title{Sichere Löschung von Festplatten und Datenspeichern}
\author{Mara Schulke\\\tiny{Matrikelnr. 20215853, SS22 B.Sc. IT Security, THB}}

\addtobeamertemplate{frametitle}{\vspace*{0.75em}\hspace*{0.5em}}{\vspace*{0.25cm}}

\begin{document}

\begin{frame}
	\begin{center}IT Forensik\end{center}
	\vspace{1em}
	\titlepage
\end{frame}

\begin{frame}{Übersicht}
	\tableofcontents
\end{frame}

\section{Wieso ist eine sichere Löschung so wichtig?}
\begin{frame}{Wieso ist eine sichere Löschung so wichtig?}
	\begin{itemize}
		\item verhinderung von Datendiebstahl
		\item rechtliche Vorgaben
		\item Hardware wiederverwendbar machen
	\end{itemize}
\end{frame}

\section{Gefahren durch falsche Löschung}
\begin{frame}{Gefahren durch falsche Löschung}
	\begin{itemize}
		\item Häufige Gerätewechsel bei mangelnder Aufbereitung des Speichermediums
			bergen die Gefahr dass ein Angreifer diese nach Verkauf / Entsorgung
			rekonstruieren kann
		\item Selbst normale Endanwender können mittels Software wie z.B.
			EaseUS schlecht gelöschte Speichermedien wiederherstellen 
	\end{itemize}
\end{frame}

\section{Wie funktioniert die Rekonstruktion von Datenspeichern?}
\begin{frame}{Wie funktioniert die Rekonstruktion von Datenspeichern?}
	\begin{itemize}
		\item Rekonstruktion ist möglich, wenn Daten nicht physisch gelöscht
			wurden
			\begin{itemize}
				\item Formatierung entfernt Daten nur oberflächlich da i.d.R.
					nur die Zugriffstabellen angepasst werden und nicht die
					Register geleert werden.
			\end{itemize}
		\item Durch Durchsuchen der Register einer Festplatten mit entfernter
			Zugriffstabelle lassen sich softwareseitig Daten rekonstruieren
		\item Bei Beschädigten oder mittels SMR gesicherten Festplatten ist
			eine softwareseitige Rekonstruktion oft nicht mehr möglich, hier
			gibt es allerdings noch die Möglichkeit über physischen Zugriff auf
			die Festplatte Informationen wiederherzustellen (bspw.\ von
			spezialisierten Unternehmen unter Reinraum-Bedingungen)
	\end{itemize}
\end{frame}

\section{Methoden zur sicheren Speicherlöschung}
\begin{frame}{Methoden zur sicheren Speicherlöschung}
	\begin{itemize}
		\item Physische Zerstörung des Datenträgers
		\item Überschreiben der Daten innerhalb des Datenträgers
	\end{itemize}
\end{frame}

\section{Überschreiben gesicherter Bereiche}
\begin{frame}{Überschreiben gesicherter Bereiche}
	\begin{itemize}
		\item Um vollständige Sicherheit zu gewährleisten, muss sämtliche
			Bereiche eines Datenträgers überschrieben werden
		\item Zu typischen gesicherten Bereichen zählen die HPA (Host protected
			area) und das DCO (device configuration overlay)
		\item Programme zur Datenlöschung, die vom Betriebssystem des Nutzers
			ausgeführt werden haben oft keinen uneingeschränkten Zugriff, da das
			Betriebssystem während der Löschung noch Teile des Datenträgers
			verwendet
	\end{itemize}
\end{frame}

\section{Übersicht etablierter Standards}
\begin{frame}{Übersicht etablierter Standards}
	\begin{itemize}
		\item Cryptographic Erasure (Crypto Erase)
			\begin{itemize}
				\item Keine Runden
				\item Löscht den kryptografischen Schlüssel, mit der die zu
					löschenden Daten hardwareseitig verschlüsselt wurden
				\item Funktioniert nicht mit jeder Hardware
			\end{itemize}
		\item Peter Gutmann's Algorithm
			\begin{itemize}
				\item Veröffentlicht: 1996
				\item 1 bis 35 Runden
				\item Muster: Kombination aus allen anderen Standards
			\end{itemize}
		\item Bruce Scheier's Algorithm
			\begin{itemize}
				\item Veröffentlicht: 1996
				\item 7 Runden
				\item Muster: 1, 0, 5x pseudozufällige Sequenz
			\end{itemize}
		\item U.S. Navy Staff Office Publication NAVSO P-5239-26, 1993
			\begin{itemize}
				\item Veröffentlicht: 1993
				\item 3 Runden
				\item Muster: Zeichen, Komplement, Zufallswert
			\end{itemize}
	\end{itemize}
\end{frame}

\begin{frame}{Übersicht etablierter Standards}
	\begin{itemize}
		\item BSI-2011-VS	
			\begin{itemize}
				\item Veröffentlicht: 2011
				\item 4 Runden
				\item Verschlüsselung mit AES-128-CBC mit anschließender
					löschung des Schlüssls
			\end{itemize}
		\item DoD 5220.22-M (E)
			\begin{itemize}
				\item Veröffentlicht: 1995
				\item 3 Runden
				\item Muster: 0, 1, Zufallswerte
			\end{itemize}
		\item DoD 5220.22-M (C)
			\begin{itemize}
				\item 1 Runden
				\item Muster: aperiodische Zufallswerte
			\end{itemize}
		\item DoD 5220.22-M (ECE)
			\begin{itemize}
				\item Veröffentlicht: 2001
				\item 7 Runde
				\item Muster: DoD 5220.22-M (E), DoD 5220.22-M (C),
					DoD 5220.22-M (E)
			\end{itemize}
	\end{itemize}
\end{frame}

\begin{frame}{Übersicht etablierter Standards}
	\begin{itemize}
		\item U.S. Air Force System Security Instruction 5020
			\begin{itemize}
				\item Veröffentlicht: 1993
				\item 3 Runden
				\item Muster: 1, 0, zufälliges Zeichen
			\end{itemize}
		\item HMG Infosec Standard 5, Lower Standard	
			\begin{itemize}
				\item 1 Runden
				\item Muster: Zufällige Zeichenkette
			\end{itemize}
		\item HMG Infosec Standard 5, Higher Standard	
			\begin{itemize}
				\item 3 Runde
				\item Muster: 0, 1, Zufallswert
			\end{itemize}
	\end{itemize}
\end{frame}

\section{Vorteile kryptografiebasierter Datenlöschung}
\begin{frame}{Vor- und Nachteile kryptografiebasierter Datenlöschung}
	\large{Vorteile:}
	\begin{itemize}
		\item Konstanter Aufwand (d.h.\ unabhängig von der Festplattengröße)
		\item Impliziert Festplattenverschlüsselung, förderlich für die
			allgemeine Sicherheit
		\item Resistent gegen alle bekannten und zukünftigen
			Datenwiederherstellungsverfahren
	\end{itemize}

	\large{Nachteile:}
	
	\begin{itemize}
		\item Sicherheit der Daten ist abhängig davon, dass die Verschlüsselung
			nicht gebrochen wird
	\end{itemize}
\end{frame}

\section{Etablierte Programme zur Datenlöschung}
\begin{frame}{Etablierte Programme zur Datenlöschung}
	\begin{itemize}
		\item DBAN
		\item EaseUS
		\item Blancco
		\item DiskWipe
	\end{itemize}
\end{frame}

\section{Quellen}
\begin{frame}{Quellen}
	\begin{itemize}
		\item BSI.\ (2021, 28. September). Daten auf Festplatten und Smartphones
			endgültig Löschen. Bundesamt für Sicherheit in der
			Informationstechnik. Abgerufen am 25. Juni, 2022, von
			\href{https://www.bsi.bund.de/DE/Themen/Verbraucherinnen-und-Verbraucher/Informationen-und-Empfehlungen/Cyber-Sicherheitsempfehlungen/Daten-sichern-verschluesseln-und-loeschen/Daten-endgueltig-loeschen/daten-endgueltig-loeschen_node.html}{https://www.bsi.bund.de/DE/Themen/Verbrauche..}
		\item Köhler, T.\ (2021, August 20).\ HDD und SSD - Festplatte Sicher
			Löschen.\ HDD und SSD - Festplatte sicher löschen | experte.de.
			Abgerufen am 25. Juni, 2022, von
			https://www.experte.de/it-sicherheit/festplatte-loeschen
		\item EaseUS.\ (2022,\ 19. Januar).\ Formatierte Festplatte
			Wiederherstellen - so geht's.\ EaseUS.\ Abgerufen am 25. Juni, 2022,
			von
			https://www.easeus.de/festplatte-wiederherstellen/formatierte-festplatte-wiederherstellen-freeware.html 
	\end{itemize}
\end{frame}

\begin{frame}{Quellen}
	\begin{itemize}
		\item Blancco. (2022, 19. Mai). A comprehensive list of data wiping and
			erasure standards. Blancco. Abgerufen am 25. Juni, 2022, von
			https://www.blancco.com/resources/blog-comprehensive-list-data-wiping-erasure-standards/
		\item Remus, Dipl.-Ing. Patric. Archicrypt Shredder 7. BSI-2011-VS.\
			Abgerufen am 25. Juni, 2022, from
			https://www.archicrypt.de/shredder7/bsi-2011-vs.html
	\end{itemize}
\end{frame}

\end{document}
