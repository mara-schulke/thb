\documentclass{article}

\usepackage[style=apa]{biblatex}
\usepackage[a4paper, left=2cm, top=2cm, right=2cm, bottom=2cm]{geometry}
\usepackage[ngerman]{babel}
\usepackage[utf8]{inputenc}
\usepackage[T1]{fontenc}
\usepackage{hyperref}
\usepackage{xcolor}
\usepackage{csquotes}
\usepackage{doc}

\hypersetup{colorlinks=true, allcolors=black}

\title{Die Sicherheit von energiesparenden 2,4-GHz-Kommunikationsprotokollen für vermaschte Netzwerke}
\author{\vspace{-0.1cm}\normalsize{Friedemann R. Pruß, Mara Schulke}\\\small{prussf@th-brandenburg.de 20215742, schulke@th-brandenburg.de 20215853}}
\date{\small{\today}\vspace{-0.1cm}}

\bibliography{references}

\begin{document}
\newgeometry{top=0.4cm, left=2cm, right=2cm, bottom=2cm}
\maketitle
\thispagestyle{empty}
\pagestyle{empty}

Die vorliegende Arbeit beleuchtet die Möglichkeiten zur sicheren Kommunikation
mit energiesparenden 2,4-GHz-Protokollen im Smart-Home-Kontext. Dazu werden zwei
weit verbreitete WPAN-Mesh-Protokolle – Thread und Bluetooth Mesh –
hinsichtlich ihrer Sicherheit verglichen. Vermaschte Netzwerke bieten im Smart
Home viele Vorteile, wie z.B. eine hohe Ausfallsicherheit und leichte
Erweiterbarkeit. Gegen die Verwendung von WAN-Protokollen wie LoRa, SigFox und
NB-IoT für Smart-Home-Netzwerke sprechen die hohen Kosten des initialen Aufbaus
eines WAN und die häufig niedrigeren Datenraten. Außerdem ist die Anzahl
möglicher Erweiterungen für das Netzwerk bei 2,4-GHz-Protokollen
deutlich höher \parencite{ThreadMeshVsOtherWirelessIEEE}.

Die Threadgroup, zu deren Mitgliedern Firmen wie Apple, Amazon und Google
gehören, bezeichnet Thread als Nachfolger von Zigbee
\parencite{ThePromiseOfThread}. Des Weiteren
setzt der IoT-Standard Matter für die Datenübertragung auf Thread \parencite{Matter}.
Matter wurde von der Connectivity Standards Alliance (ehem. Zigbee Alliance)
entwickelt, zu deren Mitgliedern unter anderen auch die vorher genannten Firmen
gehören \parencite{Matter}.

Das Forschungsergebnis wurde erreicht, in dem im Rahmen einer Survey und
Literaturanalyse die Spezifikationen der Protokolle und bereits getätigte
Sicherheitsanalysen ausgewertet wurden. Auch nach ausführlicher Recherche ließ
sich, trotz der bereits großen und immer weiter zunehmenden Bekanntheit der
beiden Protokolle, keine Publikation finden, in der ein detaillierter
Sicherheitsvergleich stattgefunden hat. In der Regel wurde eines der beiden
Protokolle alleinstehend betrachtet
\parencite{ThreadApplicationIEEE,BluetoothMeshIntro}, oder es fand ein
allgemeiner konzeptioneller Vergleich statt – zum Beispiel hinsichtlich
Architektur, Protokollaufbau, Funktionsumfang etc
\parencite{ComparativeAnalysisIEEE,ThreadMeshVsOtherWirelessIEEE}. Des Weiteren
sind die Spezifikationen für beide Protokolle öffentlich beziehungsweise
kostenlos einsehbar \parencite{BluetoothSpec,ThreadSpec}. Zu Bluetooth bzw.
Bluetooth Mesh gibt es weitreichende Sicherheitsanalysen und gut dokumentierte
Sicherheitslücken \parencite{BluetoothLowEnergyAttackOxford, BluetoothIssues}.
In weiteren Publikationen wurden Angriffe auf Thread durchgeführt
\parencite{ThreadEMAttack} und mögliche Schwachstellen untersucht
\parencite{ThreadSecurityCSIAC}. Die vorliegende Arbeit grenzt sich durch einen
sicherheitsbezogenen Vergleich von den oben gennanten Arbeiten ab.

In den letzten Jahren wurden mehrere, teils kritische, Sicherheitslücken in der
Bluetooth Spezifikation entdeckt \parencite{BluetoothIssues}. Diese
beeinträchtigen zum Großteil die Sicherheit von Direktverbindungen - also
Bluetooth BR/EDR und Bluetooth Low Energy. Zu den schwerwiegendsten
Schwachstellen in der Bluetooth Core Spezifikation zählen die sogenannten
``Bluetooth Impersonation Attacks''. Diese macht es möglich, dass sich ein
Angreifer als bereits authentifiziertes Gerät ausgibt und somit die komplette
Authentifizierung bei einem Verbindungsaufbau umgehen kann. Dies ermöglicht
Man-In-The-Middle-Attacken \parencite{BluetoothLowEnergyAttackOxford}.

Bluetooth Mesh verwendet als Übertragungsprotokoll Bluetooth Low Energy und
anders als beim Bluetooth-Low-Energy-Pairing werden elliptische 256-Bit-Kurven
und Out-of-Band-Authentifizierung verwendet, um das Hinzufügen von
Netzwerkknoten abzusichern \parencite{BluetoothSpec}. Allerdings wurden
diesbezüglich im Jahr 2020 die Sicherheitswarnungen CVE-2020-26556,
CVE–2020-26557, CVE-2020-26559, CVE-2020-26560 veröffentlicht
\parencite{BluetoothIssues}.

Thread ist im Gegensatz zu Bluetooth Mesh ein offenes Protokoll, da es kaum
einen Einfluss auf die Anwendungsebene des Netzwerks hat. Im Jahr 2020 wurde
eine Sicherheitsanalyse von Thread anhand der 10 relevantesten
Sicherheitsbedenken dem Open Web Application Security Project (OWASP) für
IoT-Netzwerke durchgeführt \parencite{ThreadSecurityCSIAC}.

Diese Analyse zeigt, dass Thread den Fokus primär auf die Verschlüsselung und
die Sicherung der Übertragung legt, aber Risiken durch die fehlende
Spezifikation der Anwendungsebene entstehen könnten. Hersteller müssen selbst
für die Sicherheit auf der Anwendungsebene sorgen. Ein großer Vorteil der
offenen Spezifikation ist die Möglichkeit angemessene und erforschte
Kryptografie einzusetzen.

Die aktuell bekannten Sicherheitslücken von Bluetooth Mesh betreffen das
Hinzufügen von neuen Netzwerkknoten \parencite{BluetoothIssues} und vermindern
somit die Integrität des Netzwerks. Bislang wurden bei Thread, trotz ausgiebiger
Untersuchungen \parencite{ThreadSecurityCSIAC, ThreadSecurityEmbeddedCom}
noch keine vergleichbaren Schwachstellen gefunden. Auf der anderen Seite tritt
Thread die Verantwortung zu Absicherung der Anwendungsschicht an den Hersteller
ab, ist somit zwar flexibler, könnte aber mögliche Sicherheitslücken durch
nicht abgesicherte Geräte öffnen.

Eine eindeutige Empfehlung für oder gegen eines der beiden Protokolle nur auf
Basis der Sicherheit kann nicht klar gegeben werden, da beide Protokolle
unterschiedliche Vorteile und Sicherheitsrisiken mit sich bringen. Eine
Entscheidung muss vom Implementierungskontext abhängig gemacht werden.

\restoregeometry

\printbibliography[heading=none]

\end{document}
