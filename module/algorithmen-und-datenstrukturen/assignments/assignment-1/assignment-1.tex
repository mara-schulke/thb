\documentclass{article}

\usepackage[a4paper]{geometry}
\usepackage[ngerman]{babel}
\usepackage[utf8]{inputenc}
\usepackage[T1]{fontenc}
\usepackage{amsmath}
\usepackage{listings}
\usepackage{color}

\lstset{
	language=Python,
	aboveskip=3mm,
	belowskip=3mm,
	showstringspaces=false,
	columns=flexible,
	basicstyle={\small\ttfamily},
	numbers=none,
	breaklines=true,
	breakatwhitespace=true,
	tabsize=4
}

\renewcommand\thesubsection{\alph{subsection})}

\begin{document}

\begin{titlepage}
	\begin{flushleft}
		TH Brandenburg \\
		Online Studiengang Medieninformatik \\
		Fachbereich Informatik und Medien \\
		Algorithmen und Datenstrukturen \\
		Prof. Dr. rer. nat. Ulrich Baum
	\end{flushleft}

	\vfill

	\begin{center}
		\Large{Einsendeaufgabe 1}\\[0.5em]
		\large{Sommersemester 2021}\\[0.25em]
		\large{Abgabetermin 25.04.2021}
	\end{center}

	\vfill

	\begin{flushright}
		Mara Schulke \\
		Matrikel-Nr. 20215853
	\end{flushright}
\end{titlepage}

\newpage

\section{Zweitkleinstes Element einer Folge  von $n \geq 2$ Zahlen}

\subsection{Algorithmus in Pseudocode}

\begin{lstlisting}
def second_minimum(list):
    second = list[0]
    minimum = list[0]

    for n in list[1:]:
        if n > minimum:
            second = n
            break

    for n in list[1:]:
        if n < minimum:
            second = minimum
            minimum = n

    return second
\end{lstlisting}

\subsection{Laufzeit-Analyse}

Der Algorithmus braucht im \textbf{Best-Case n} Vergleiche, liegt also dementsprechend in $\Omega(n)$. Der Best-Case tritt ein,
wenn direkt das zweite Element größer als das Erste ist, da dann die erste Schleife nach dem ersten Schritt abgebrochen wird und
die 2. Schleife immer genau $n - 1$ Vergleiche ausführt.

Er braucht im \textbf{Worst-Case 2(n - 1)} Vergleiche und liegt daher in $O(n)$. Der Worst-Case kommt zustande, wenn wir z.B. eine
List der Länge n betrachten, die n mal das gleiche Element enthält. Dann benötigen wir bei dem Durchlaufen der ersten und der zweiten
Schleife $n - 1$ Vergleiche.

\section{Asymptotische Notation}

Gegeben sei die Funktion $f(n) = 2n^2 + 3n\log_2n - 72$

\subsection{Beweis von $f(n) \in O(n^2)$}

\begin{align*}
	f(n) & = 2n^2 + 3n\log_2n - 72 \\
	     & \leq 2n^2 + 3n\log_2n   \\
	     & \leq 2n^2 + 3n^2        \\
	     & = 5n^2
\end{align*}

\begin{flushleft}
	Somit können wir sagen, dass mit $c \geq 5$ und $n_0 = 1$ die Behauptung $f(n) \in O(n^2)$ gilt.
\end{flushleft}

\subsection{Beweis von $f(n) \in \Omega(n^2)$}

\begin{align*}
	f(n) & = 2n^2 + 3n\log_2n - 72 \\
	     & \geq 2n^2 - 72          \\
	     & \geq n^2
\end{align*}

\begin{flushleft}
	Nun können wir $n_0$ als Schnittpunkt der beiden Funktionen $2n^2 - 72$ und $n^2$ berechnen.
\end{flushleft}

\begin{align*}
	2n^2 - 72 & = n^2 \mid - 2n^2 \\
	- 72      & = -n^2 \mid * -1  \\
	72        & = n^2             \\
	n         & = \sqrt{72}
\end{align*}

\vspace{1em}

\begin{flushleft}
	Also mit $c = 1$ und $n_0 = \lceil\sqrt{72}\rceil = 9$ gilt $f(n) \in \Omega(n^2)$
\end{flushleft}

\subsection{Gilt $f(n) \in \Theta(n^2)$ ?}

\begin{flushleft}
	$\Theta(g)$ ist im Skript mit der \emph{Definition 2.5} als $\{ \, f \mid f \in O(g) \wedge f \in \Omega(g) \, \}$ definiert.

	\vspace{0.5em}

	Somit wissen wir, dass $f(n) \in \Theta(n^2)$, da wir in 2 a) und 2 b) gezeigt haben, dass $f \in O(g)$ und $f \in \Omega(g)$ gelten.
\end{flushleft}

\section{Average-Case-Aufwand der binären Suche}

\subsection{Durchschnittliche Anzahl der Vergleiche für einen Hit}

\vspace{1em}

\begin{center}
	\begin{tabular}{|c|c|c|c|c|c|c|c|c|c|c|}
		\hline
		Element    & 0 & 1 & 2 & 3 & 4 & 5 & 6 & 7 & 8 & 9 \\
		\hline
		Vergleiche & 3 & 2 & 3 & 4 & 1 & 3 & 4 & 2 & 3 & 4 \\
		\hline
	\end{tabular}
\end{center}

\vspace{1em}

Macht in Summe 29 Vergleiche und somit $\frac{29}{10} = 2.9$ Vergleiche im Durchschnitt.

\subsection{Summenformel für Vergleiche bei $2^k - 1$ Elementen}

\begin{center}
	\vspace{1em}

	Beispiel für $k = 3$

	\vspace{0.5em}

	\begin{tabular}{|c|c|c|c|c|c|c|}
		\hline
		  &   &   & 3 &   &   &   \\
		\hline
		  & 1 &   &   &   & 5 &   \\
		\hline
		0 &   & 2 &   & 4 &   & 6 \\
		\hline
	\end{tabular}

	\vspace{1.5em}

	Beispiel für $k = 4$

	\vspace{0.5em}

	\begin{tabular}{|c|c|c|c|c|c|c|c|c|c|c|c|c|c|c|}
		\hline
		  &   &   &   &   &   &   & 7 &   &   &    &    &    &    &    \\
		\hline
		  &   &   & 3 &   &   &   &   &   &   &    & 11 &    &    &    \\
		\hline
		  & 1 &   &   &   & 5 &   &   &   & 9 &    &    &    & 13 &    \\
		\hline
		0 &   & 2 &   & 4 &   & 6 &   & 8 &   & 10 &    & 12 &    & 14 \\
		\hline
	\end{tabular}
\end{center}

\newpage

Es gibt bei $2^k - 1$ immer einen perfekten, gleichmäßigen Baum und immer genau $\log_2n$ bzw. $k$ Ebenen.
Auf (einer 0 indizierten) Ebene $i$ haben wir den Baum $i$ Mal geteilt und haben $(i + 1) 2^i$ Vergleiche
auf dieser Ebene. Wenn wir nun alle Ebenen addieren möchten, um die Gesamtanzahl der Vergleiche zu bekommen,
müssen wir lediglich alle Ebenen addieren. Also bei $k = 4$ wären wir bei $1 * 2^0 + 2 * 2^1 + 3 * 2^2 + 4 * 2^3$
Vergleichen. Dies lässt sich durch die Gaußsche Summenformel eleganter (und allgemeingültiger) Zusammenfassen
zu $\sum_{i=0}^{k - 1} (i + 1) * 2^i$. Um jetzt auf die durchschnittlichen Vergleiche zu kommen, muss nun einfach
die Gesamtanzahl durch die Anzahl der Elemente geteilt werden. Also entweder $\sum_{i=0}^{k - 1} \frac{(i + 1) * 2^i}{2^k - 1}$
oder alternativ $\frac{1}{2^k - 1}\sum_{i=0}^{k - 1} (i + 1) * 2^i$

\subsection{Laufzeit-Analyse für Average-Case und Vergleich mit Worst-Case}

\vspace{1em}

$$\frac{1}{2^k - 1}\sum_{i=0}^{k - 1} (i + 1) * 2^i = \frac{1}{2^k - 1}\sum_{i=1}^k i * 2^{i - 1} = k + \frac{k}{2^k - 1} - 1$$

\begin{center}
	Von \texttt{http://www.mcs.sdsmt.edu/ecorwin/cs251/binavg/binavg.html}
\end{center}

\vspace{1em}

\begin{flushleft}
	Der Worst-Case der binären Suche ist $O(\log_2 n)$. Wenn wir in der o.g. geschlossenen Formel $k$
	durch $\lceil\log_2 n\rceil$ ersetzen erhalten wir:
\end{flushleft}

\vspace{0em}

$$\lceil\log_2 n\rceil + \frac{\lceil\log_2 n\rceil}{n} - 1$$

\vspace{0em}

\begin{flushleft}
	Wenn wir diese Gleichung nun asymptotisch betrachten entfällt $\frac{\lceil\log_2 n\rceil}{n}$
\end{flushleft}

\vspace{0em}

$$\lim_{n\to\infty} \lceil\log_2 n\rceil + \frac{\lceil\log_2 n\rceil}{n} - 1 = \lceil\log_2 n\rceil + 0 - 1 = \lceil\log_2 n\rceil - 1$$

\vspace{0em}

\begin{flushleft}
	Somit können wir sagen, dass der Average-Case der binären Suche asymptotisch gesehen nur um 1 Schritt
	besser ist als der Wort-Case $\log_2n$
\end{flushleft}

\section{Analyse einer rekursiven Funktion}

\subsection{Für welche n terminiert die Rekursion, für welche nicht?}

Die Funktion terminiert nur für gerade postive Zahlen und $n = 0$. Bei ungeraden positiven Zahlen verfehlen wir den
Basis-Fall immer um genau 1 und landen danach in einer Endlosschleife. Bei negativen Zahlen sind wir schon initial
``unter`` dem Basis-Fall.

\subsection{Geben Sie F als geschlossene nicht-rekursive Formel an und beweisen Sie Ihre Formel durch Induktion.}

Geschlossene Formel für $F$:

$$F(n) = \frac{n}{2} + \frac{n^2}{4}$$

\vspace{1em}

\begin{flushleft}
	Induktionsbeweis:
\end{flushleft}

\begin{align*}
	Induktionsvoraussetzung & = \frac{n}{2} + \frac{n^2}{4} = F(n)                           & n \in \{ x \mid x \in N_0 \wedge x \; mod \; 2 = 0 \} \\
	Induktionsbehauptung    & = \frac{n + 2}{2} + \frac{(n + 2)^2}{4} = F(n + 2)                                                                     \\
	Induktionsanfang        & = \frac{0}{2} + \frac{0^2}{4} = F(0) \Leftrightarrow 0 = 0                                                             \\
	Induktionsschritt       & = \frac{n + 2}{2} + \frac{(n + 2)^2}{4}                                                                                \\
	                        & = \frac{n}{2} + \frac{n^2 + 4n + 4}{4} + \frac{2}{2}                                                                   \\
	                        & = \frac{n}{2} + \frac{n^2}{4} + \frac{4n + 4}{4} + \frac{2}{2}                                                         \\
	                        & = F(n) + (n + 2) = F(n + 2)                                                                                            \\
\end{align*}

\subsection{Stellen Sie eine Rekursionsgleichung für $T(n)$ auf und geben Sie eine geschlossene Formel für $T(n)$ an.}

Rekursionsgleichung:

\begin{align*}
	T(0) & = 0               \\
	T(n) & = \, T(n - 2) + 2
\end{align*}

\vspace{1em}

\begin{flushleft}
	Geschlossene Formel für $T(n)$:
\end{flushleft}

$$T(n) = n$$

\end{document}
