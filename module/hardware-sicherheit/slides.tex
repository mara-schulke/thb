\documentclass{beamer}

\usepackage[ngerman]{babel}
\usepackage[utf8]{inputenc}
\usepackage[T1]{fontenc}
\usepackage{hyperref}

\usecolortheme{seagull}
\usefonttheme{serif}

\title{Proof-of-Concept-Implementierung einer Anwendung mit FIDO2 Support}
\author{Mara Schulke\\\tiny{Matrikelnr. 20215853, SS22 B.Sc. IT Security, THB}}

\begin{document}

\begin{frame}
	\begin{center}Hardware Sicherheit\end{center}
	\vspace{1em}
	\titlepage
\end{frame}

\begin{frame}{Übersicht}
	\tableofcontents
\end{frame}

\section{Der FIDO2 Standard}
\begin{frame}{Der FIDO2 Standard}
	\begin{itemize}
		\item Industriestandard für hardwaregestützte Authentifizierung
		\item Setzt sich im wesentlichen aus zwei
			Protokollen zusammen:
			\begin{itemize}
				\item CTAP
				\item WebAuthn
			\end{itemize}
		\item Der Hardwaretoken kann innerhalb des Systems verbaut sein (z.B.
			über ein TPM) oder ein externes Gerät sein (bspw. Yubikey oder
			Google Titan)
		\item Quelloffen und nicht proprietär
			\begin{itemize}
				\item Durch Projekte wie OpenSK lassen sich sogar eigene
					Hardwaretokens bauen
			\end{itemize}
	\end{itemize}
\end{frame}

\section{Ablauf einer Registrierung}
\begin{frame}{Ablauf einer Registrierung}
	\begin{itemize}
		\item Client erhält vom Server Schlüsselgenerierungsparameter
		\item Client reicht diese an den Authenticator weiter
		\item Authenticator speichert das Schlüsselpaar für die Relying Party und gibt den Public Key an den Client
		\item Client leitet den Public Key an den Server weiter
		\item Server persistiert den Public Key und ordnet ihn dem Nutzer zu
	\end{itemize}
\end{frame}

\section{Ablauf einer Authentifizierung}
\begin{frame}{Ablauf einer Authentifizierung}
	\begin{itemize}
		\item Client erhält vom Server eine Challenge
		\item Client reicht diese an den Authenticator weiter
		\item Authenticator signiert die Challenge mit dem Private Key
		\item Client leitet die Signatur an den Server weiter
		\item Server überprüft mit dem Public Key ob die Signatur gültig ist
	\end{itemize}
\end{frame}

\section{Vorteile gegenüber einfacher asymmetrischer Kryptografie}
\begin{frame}{Vorteile gegenüber einfacher asymmetrischer Kryptografie}
	\begin{itemize}
		\item Erhöhte Sicherheit durch separate Schlüsselpaare pro Relying Party
		\item Verlust eines Schlüsselpaars wäre verhältnismäßig unkritisch da
			dieses nur Zugriff auf einen Online-Dienst gibt
		\item Schlüsselpaare werden nicht auf dem System des Nutzers gespeichert
			\begin{itemize}
				\item Kompromitierte Verbindungen und Nutzersysteme stellen keine schwerwiegende Gefahr dar
				\item Verlust des Systems unkritisch
			\end{itemize}
		\item Physikalischer Diebstahl des Tokens notwendig um erfolgreiche
			Impersonation-Attacken auszuführen
	\end{itemize}
\end{frame}

\section{Möglichkeiten durch den Einsatz von FIDO2}
\begin{frame}{Möglichkeiten durch den Einsatz von FIDO2}
	\begin{itemize}
		\item viele Angriffsvektoren lassen sich zu weiten Teilen durch den
			Einsatz von FIDO2 ausschließen bzw.\ verharmlosen (bspw.
			Man-In-The-Middle, Phishing etc.)
		\item Nutzer müssen keine Passwörter mehr verwalten
		\item Nutzer können sich einfach und sicher passwortlos Anmelden
	\end{itemize}
\end{frame}

\section{Details zur Implementierung des PoC}
\begin{frame}{Details zur Implementierung des PoC}
	\begin{itemize}
		\item Siehe Bildschirmübertragung
	\end{itemize}
\end{frame}

\section{Anhänge / Verweise}
\begin{frame}{Anhänge / Verweise}
	\begin{itemize}
		\item Quellcode: \href{https://github.com/mara214/fido2-auth}{github.com/mara214/fido2-auth}
		\item Authenticator: \href{https://store.google.com/de/product/titan_security_key?pli=1&hl=de}{Google Titan}
	\end{itemize}
\end{frame}

\end{document}
