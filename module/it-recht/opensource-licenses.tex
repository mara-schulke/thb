\documentclass{beamer}

\usepackage[ngerman]{babel}
\usepackage[utf8]{inputenc}
\usepackage[T1]{fontenc}

\usecolortheme{seagull}
\usefonttheme{serif}

\title{Open-Source-Lizenzen}
\subtitle{Einordnung und Abgrenzung}
\author{Mara Schulke\\\tiny{Matrikelnr. 20215853, SS22 B.Sc. IT Security, THB}}
\date{20. Mai 2022}

\AtBeginSection[]{
	\begin{frame}{Übersicht}
		\tableofcontents[currentsection]
	\end{frame}
}

\begin{document}

\begin{frame}
	IT Recht \hfill betreut durch Vilma Niclas
	\vspace{1em}
	\titlepage
\end{frame}

\section{Wieso sind Lizenzen wichtig?}
\begin{frame}{Wieso sind Lizenzen wichtig?}
	\begin{itemize}
		\item Lizenzverstöße sind Urheberrechtsverletzungen nach dem Urheberrechtsgesetz
		\item ethische Grundsätze und soziale Verträge
		\item dienen der Unterstützung von Open Source Projekten und
			Entwicklern
	\end{itemize}
\end{frame}

\section{Rechtliche Betrachtung von Open-Source-Lizenzen}
\begin{frame}{Rechtliche Betrachtung von Open-Source-Lizenzen}
	\begin{itemize}
		\item Open-Source-Lizenzen sind Lizenzverträge
		\item treten Verwertungsrechte ab:
			\begin{itemize}
				\item Nutzungsrechte (UhrG §31)
				\item Distributionsrechte (UhrG §17 / §19a)
				\item Erlaubnis zur Modifikation (UhrG §23)
			\end{itemize}
		\item enthalten bestimmte Bedingungen unter denen die Rechte ausgeübt
			werden dürfen
		\item das Urheberrecht bzw. Eigentum der Software kann nicht durch eine
			Lizenz übertragen werden
		\item private Nutzung kann nicht von einer Lizenz unterbunden werden
	\end{itemize}
\end{frame}

\section{Lizenztypen}
\begin{frame}{Lizenztypen}
	\begin{itemize}
		\item Permissive
		\item Copyleft
	\end{itemize}
\end{frame}

\subsection{Permissive Lizenzen}
\begin{frame}{Lizenztypen:\ Permissive Lizenzen}
	\begin{itemize}
		\item haben einen freien und erlaubenden Charakter
		\item lassen Redistributionen unter anderen Lizenzen zu
		\item bestehen oft primär aus einem Haftungsausschluss, ohne
			nennenswerte weitere Vorgaben
	\end{itemize}
\end{frame}

\begin{frame}{Lizenztypen:\ Beispiele für Permissive Lizenzen}
	\begin{itemize}
		\item MIT License
		\item Apache License 2.0
		\item BSD-2-Clause License
		\item BSD-3-Clause License
	\end{itemize}
\end{frame}

\subsection{Copyleft Lizenzen}
\begin{frame}{Lizenztypen:\ Copyleft Lizenzen}
	\begin{itemize}
		\item bekannt für sog.\ virale Lizensierung,
			verbieten Redistributionen unter anderen Lizenzen
		\item haben i.d.R. einen eher extremen / politischen Charakter (bspw. GPLv3)
		\item spiegeln Ideologie der Free-Software-Foundation wieder
		\item gemieden von propräitärer Software
	\end{itemize}
\end{frame}

\begin{frame}{Lizenztypen:\ Beispiele für Copyleft Lizenzen}
	Schwaches Copyleft
	\vspace{0.5em}
	\begin{itemize}
		\item Mozilla Public License
		\item Eclipse Public License
		\item GNU Lesser General Public License
	\end{itemize}
	\vspace{1em}

	Starkes Copyleft
	\vspace{0.5em}
	\begin{itemize}
		\item GNU General Public License
		\item GNU Affero General Public License
	\end{itemize}
\end{frame}

\section{Die wichtigsten Lizenzen im Detail}
\subsection{MIT License}
\begin{frame}{MIT License}
	Erlaubt:
	\begin{itemize}
		\item Kommerzielle Nutzung
		\item (Re-) Distribution
		\item Modifikation
	\end{itemize}

	Bedingungen:
	\begin{itemize}
		\item Lizenz und Copyright-Vermerk müssen mit ausgeliefert werden
	\end{itemize}

	Schließt aus:
	\begin{itemize}
		\item Haftung
		\item Gewährleistung
	\end{itemize}
\end{frame}

\subsection{Apache License 2.0}
\begin{frame}{Apache License 2.0}
	Erlaubt:
	\begin{itemize}
		\item Kommerzielle Nutzung
		\item (Re-) Distribution
		\item Modifikation
		\item Patent Nutzung
	\end{itemize}

	Bedingungen:
	\begin{itemize}
		\item Lizenz und Copyright-Vermerk müssen mit ausgeliefert werden
		\item Änderungen müssen kenntlich gemacht werden
	\end{itemize}

	Schließt aus:
	\begin{itemize}
		\item Haftung
		\item Gewährleistung
		\item Verwendung der Marke
	\end{itemize}
\end{frame}

\subsection{Mozilla Public License}
\begin{frame}{Mozilla Public License}
	Erlaubt:
	\begin{itemize}
		\item Kommerzielle Nutzung
		\item (Re-) Distribution
		\item Modifikation
		\item Patent Nutzung
	\end{itemize}

	Bedingungen:
	\begin{itemize}
		\item Lizenz und Copyright-Vermerk müssen mit ausgeliefert werden
		\item Quellcode muss bei Distribution veröffentlicht werden
		\item Änderungen von bestehenden Dateien müssen unter der gleichen
			Lizenz veröffentlicht werden
	\end{itemize}

	Schließt aus:
	\begin{itemize}
		\item Haftung
		\item Gewährleistung
		\item Verwendung der Marke
	\end{itemize}
\end{frame}

\subsection{GNU Lesser General Public License v3}
\begin{frame}{GNU Lesser General Public License v3}
	Erlaubt:
	\begin{itemize}
		\item Kommerzielle Nutzung
		\item (Re-) Distribution
		\item Modifikation
		\item Patent Nutzung
	\end{itemize}

	Bedingungen:
	\begin{itemize}
		\item Lizenz und Copyright-Vermerk müssen mit ausgeliefert werden
		\item Quellcode muss bei Distribution veröffentlicht werden
		\item Änderungen müssen kenntlich gemacht werden
		\item Abstammende Arbeiten müssen unter der gleichen
			Lizenz veröffentlicht werden, außer sie verwenden die ursprüngliche
			Arbeit nur als Bibliothek
	\end{itemize}

	Schließt aus:
	\begin{itemize}
		\item Haftung
		\item Gewährleistung
	\end{itemize}
\end{frame}

\subsection{GNU General Public License v3}
\begin{frame}{GNU General Public License v3}
	Erlaubt:
	\begin{itemize}
		\item Kommerzielle Nutzung
		\item (Re-) Distribution
		\item Modifikation
		\item Patent Nutzung
	\end{itemize}

	Bedingungen:
	\begin{itemize}
		\item Lizenz und Copyright-Vermerk müssen mit ausgeliefert werden
		\item Quellcode muss bei Distribution veröffentlicht werden
		\item Änderungen müssen kenntlich gemacht werden
		\item Abstammende Arbeiten müssen unter der gleichen
			Lizenz veröffentlicht werden
	\end{itemize}

	Schließt aus:
	\begin{itemize}
		\item Haftung
		\item Gewährleistung
	\end{itemize}
\end{frame}

\subsection{GNU Affero General Public License v3}
\begin{frame}{GNU Affero General Public License v3}
	Erlaubt:
	\begin{itemize}
		\item Kommerzielle Nutzung
		\item (Re-) Distribution
		\item Modifikation
		\item Patent Nutzung
	\end{itemize}

	Bedingungen:
	\begin{itemize}
		\item Lizenz und Copyright-Vermerk müssen mit ausgeliefert werden
		\item Quellcode muss bei Distribution veröffentlicht werden
		\item Änderungen müssen kenntlich gemacht werden
		\item Abstammende Arbeiten müssen unter der gleichen
			Lizenz veröffentlicht werden
		\item Netzwerk-Interaktion mit der Arbeit zählt als Distribution
	\end{itemize}

	Schließt aus:
	\begin{itemize}
		\item Haftung
		\item Gewährleistung
	\end{itemize}
\end{frame}

\section{Unlizensierte Software}
\begin{frame}{Unlizensierte Software}
	\begin{itemize}
		\item Quellcode darf ohne explizite Erlaubnis des Urhebers nicht verwendet werden.
	\end{itemize}
\end{frame}

\section{Rechtsfolgen von Lizenzverstößen}
\begin{frame}{Rechtsfolgen von Lizenzverstößen: Rechte des Urhebers}
	Urheber kann folgende Ansprüche geltend machen:
	\begin{itemize}
		\item Anspruch auf Unterlassung
		\item Anspruch auf Beseitigung
		\item Anspruch auf Schadensersatz 
		\item Anspruch auf Vernichtung, Rückruf und Überlassung
		\item Anspruch auf Auskunft
		\item Anspruch auf Vorlage und Besichtigung
	\end{itemize}
\end{frame}

\begin{frame}{Rechtsfolgen von Lizenzverstößen: Potentielle Kosten}
	Potentielle Kosten:
	\begin{itemize}
		\item Abmahnungsgebühr
		\item Schadensersatz
		\item Anwaltskosten
		\item Gebühren für Gutachter
	\end{itemize}
\end{frame}

\section{Verständnisfragen}
\begin{frame}{Verständnisfragen}
	\begin{itemize}
		\item Welche Faktoren müssen bei der Lizenzwahl beachtet werden?
		\item Wodurch zeichnet sich der Copyleft-Effekt aus?
	\end{itemize}
\end{frame}

\begin{frame}{Antworten auf die Verständnisfragen}
	\begin{itemize}
		\item Frage 1: 
			\begin{itemize}
				\item Soll meine Arbeit immer Open Source bleiben? % Permissive vs Copyleft
				\item Habe ich eine Marke die ich Schützen will? % Trademark vs no Trademark
				\item Möchte ich Änderungen von Dritten mitbekommen? % Kenntlichmachung
			\end{itemize}
		\item Frage 2: 
			\begin{itemize}
				\item Durch die virale Lizensierung; Sobald eine Copyleft
					eingeführt wurde müssen alle abstammenden Arbeiten auch
					unter dieser lizensiert werden.
			\end{itemize}
	\end{itemize}
\end{frame}

\section{Quellen}
\begin{frame}{Quellen}
	\begin{itemize}
		\item Bundesministerium für Justiz (2021, 23. Juni): UrhG - nichtamtliches Inhaltsverzeichnis:
			\href{https://www.gesetze-im-internet.de/urhg/}{https://www.gesetze-im-internet.de/urhg/}

			[Abgerufen am: 15.05.2022]
		\item Urheberrecht.de (2022, 22. März): Urheberrechtsverletzung: Welche Folgen sind zu erwarten?:
			\href{https://www.urheberrecht.de/urheberrechtsverletzung/}{https://www.urheberrecht.de/urheberrechtsverletzung/}

			[Abgerufen am: 15.05.2022]
		\item GitHub, Inc. (2022, 29. April): Licenses:
			\href{https://choosealicense.com/licenses/}{https://choosealicense.com/licenses/}

			[Abgerufen am: 15.05.2022]
		\item Free Software Foundation, Inc. (2018, 15. Dezember): Verschiedene Lizenzen und Kommentare:
			\href{https://www.gnu.org/licenses/license-list.de.html}{https://www.gnu.org/licenses/license-list.de.html}

			[Abgerufen am: 15.05.2022]
		\item Free Software Foundation, Inc. (2022, 03. März): Copyleft. Was ist das?:
			\href{https://www.gnu.org/licenses/copyleft}{https://www.gnu.org/licenses/copyleft}

			[Abgerufen am: 15.05.2022]
	\end{itemize}

\end{frame}

\end{document}
