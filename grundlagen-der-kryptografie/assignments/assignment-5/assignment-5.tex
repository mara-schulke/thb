\documentclass{article}

\usepackage[a4paper]{geometry}
\usepackage[ngerman]{babel}
\usepackage[utf8]{inputenc}
\usepackage[T1]{fontenc}
\usepackage{amsmath}
\usepackage{amsfonts} 
\usepackage{listings}
\usepackage{color}

\renewcommand\thesubsubsection{\textnormal{(\alph{subsubsection})}}

\begin{document}

\begin{titlepage}
	\begin{flushleft}
		TH Brandenburg\\
		Online Studiengang IT Sicherheit\\
		Fachbereich Informatik und Medien\\
		Algorithmen und Datenstrukturen\\
		Prof.\ Dr.\ rer.\ nat.\ Ulrich Baum
	\end{flushleft}

	\vfill

	\begin{center}
		\Large{Einsendeaufgabe 5}\\[0.5em]
		\large{Sommersemester 2022}\\[0.25em]
		\large{Abgabetermin \today}
	\end{center}

	\vfill

	\begin{flushright}
		Mara Schulke \\
		Matrikel-Nr. 20215853
	\end{flushright}
\end{titlepage}

\newpage

\section*{Einsendeaufgabe 5}
\stepcounter{section}
\stepcounter{section}
\stepcounter{section}
\stepcounter{section}
\stepcounter{section}

\subsection{Rechnen mit Restklassen}

\begin{align*}
	4^6 + 464 -24 * 36 \equiv -1 * -1 * -1 + 5 -7 * 2 \equiv -10 \equiv 7 \mod 17
\end{align*}

\subsection{Äquivalenzrelationen}

\begin{align*}
	& R = \{\,(a,b)R(c,d)\,|\,a + d = b + c,\, (a,b) \in \mathbb{N}^2,\, (c,d) \in \mathbb{N}^2\,\}\\
	& \textit{reflexiv}: a + b = b + a\\
	& \textit{symmetrisch}: a + d = b + c \Rightarrow c + b = d + a\\
	& \textit{transitiv}: a + d = b + c = a - b = c - d \land c - d = e - f \Rightarrow a - b = e - f\\
\end{align*}

\subsection{Subtraktion von Restklassen}

\begin{align*}
	\textnormal{Wenn}\,     & a \equiv b \mod n,\,c \equiv d \mod n\\
	\textnormal{dann gilt}\,& a = b + xn\,\textnormal{und}\,c = d + yn & x,y \in \mathbb{Z}
\end{align*}

\begin{align*}
				 & a - c \equiv b - d           & \mod n\\
	\Rightarrow\,& b + xn - d + yn \equiv b - d & \mod n\\
	\Rightarrow\,& b - d \equiv b - d           & \mod n\\
\end{align*}

\end{document}
