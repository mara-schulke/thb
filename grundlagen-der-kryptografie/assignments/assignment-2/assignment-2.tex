\documentclass{article}

\usepackage[a4paper]{geometry}
\usepackage[ngerman]{babel}
\usepackage[utf8]{inputenc}
\usepackage[T1]{fontenc}
\usepackage{amsmath}
\usepackage{listings}
\usepackage{color}

\renewcommand\thesubsubsection{\textnormal{(\alph{subsubsection})}}

\begin{document}

\begin{titlepage}
	\begin{flushleft}
		TH Brandenburg\\
		Online Studiengang IT Sicherheit\\
		Fachbereich Informatik und Medien\\
		Algorithmen und Datenstrukturen\\
		Prof.\ Dr.\ rer.\ nat.\ Ulrich Baum
	\end{flushleft}

	\vfill

	\begin{center}
		\Large{Einsendeaufgabe 2}\\[0.5em]
		\large{Sommersemester 2022}\\[0.25em]
		\large{Abgabetermin \today}
	\end{center}

	\vfill

	\begin{flushright}
		Mara Schulke \\
		Matrikel-Nr. 20215853
	\end{flushright}
\end{titlepage}

\newpage

\section*{Einsendeaufgabe 2}
\stepcounter{section}
\stepcounter{section}

\subsection{OTP-Verschlüsselung}

\subsubsection{}

\begin{align*}
	  M & = 1101 1110 1010 1101 1011 1110 1110 1111_2\\
	  K & = 1100 1010 1111 1110 1011 1010 1011 1110_2\\
	  C & = M_2 \oplus K_2\\
		& = 1101 1110 1010 1101 1011 1110 1110 1111_2\\
		& \oplus 1100 1010 1111 1110 1011 1010 1011 1110_2\\
		& = 0001 0100 0101 0011 0000 0100 0101 0001_2\\
		& = 145300451_{16}
\end{align*}

\subsubsection{}

\begin{align*}
	C & = 3522988314_{10}\\
	  & = 1101 0001 1111 1100 1000 1001 0001 1010_2\\
	K & = 11223344_{16}\\
	  & = 0001 0001 0010 0010 0011 0011 0100 0100_2\\
	M & = C_2 \oplus K_2\\
	  & = 1101 0001 1111 1100 1000 1001 0001 1010_2\\
	  & \oplus 0001 0001 0010 0010 0011 0011 0100 0100_2\\
	  & = 1100 0000 1101 1110 1011 1010 0101 1110_2\\
	  & = C0DEBA5E_{16}
\end{align*}

\subsection{Square-and-Multiply}

\begin{align*}
	  y & = 1 0000 0001_2										\\
	a_0 & = x 				& (y_0\,\textnormal{muss 1 sein wenn $y > 0 \rightarrow$ a = x})\\
	a_1 & = x^2				& (y_1\,\%\,2 = 0 \rightarrow a^2)  \\
	a_2 & = x^4				& (y_2\,\%\,2 = 0 \rightarrow a^2)  \\
	a_3 & = x^8				& (y_3\,\%\,2 = 0 \rightarrow a^2) 	\\
	a_4 & = x^{16}			& (y_4\,\%\,2 = 0 \rightarrow a^2) 	\\
	a_5 & = x^{32}			& (y_5\,\%\,2 = 0 \rightarrow a^2) 	\\
	a_6 & = x^{64}			& (y_6\,\%\,2 = 0 \rightarrow a^2) 	\\
	a_7 & = x^{128}			& (y_7\,\%\,2 = 0 \rightarrow a^2) 	\\
	a_8 & = x^{257}			& (y_8\,\%\,2 = 1 \rightarrow a^2 * x)\\
		& \rightarrow \textnormal{9 Multiplikationen}
\end{align*}


\end{document}
