\documentclass{article}

\usepackage[a4paper]{geometry}
\usepackage[ngerman]{babel}
\usepackage[utf8]{inputenc}
\usepackage[T1]{fontenc}
\usepackage{amsmath}
\usepackage{amsfonts} 
\usepackage{listings}
\usepackage{color}

\renewcommand\thesubsubsection{\textnormal{(\alph{subsubsection})}}

\begin{document}

\begin{titlepage}
	\begin{flushleft}
		TH Brandenburg\\
		Online Studiengang IT Sicherheit\\
		Fachbereich Informatik und Medien\\
		Algorithmen und Datenstrukturen\\
		Prof.\ Dr.\ rer.\ nat.\ Ulrich Baum
	\end{flushleft}

	\vfill

	\begin{center}
		\Large{Einsendeaufgabe 3}\\[0.5em]
		\large{Sommersemester 2022}\\[0.25em]
		\large{Abgabetermin \today}
	\end{center}

	\vfill

	\begin{flushright}
		Mara Schulke \\
		Matrikel-Nr. 20215853
	\end{flushright}
\end{titlepage}

\newpage

\section*{Einsendeaufgabe 3}
\stepcounter{section}
\stepcounter{section}
\stepcounter{section}

\subsection{Hauptsatz der elementaren Zahlentheorie}

\subsubsection{}

Es existiert eine bijektive Abbildung $\nu: \mathbb{N} \setminus \{0,1\} \mapsto (p_0,..,p_n)$
bei der gilt $p_i \in P, n \in \mathbb{N}, x \in \mathbb{N} \setminus \{0,1\}$ und $\prod_{i=0}^{n} p_i = x$.
Daraus folgt, dass $\mathbb{N}$ sich in Abhängigkeit der abzählbar unendlichen
Menge $T$ aller möglichen Tupel $(p_n)$ definieren lässt:

\[\mathbb{N} = \{ x\,|\,\forall\,(p_0,..,p_n) \in T: x = \prod_{i=0}^{n} p_i  \} \cup \{0,1\}\]

\subsubsection{}

Würde nun die 1 zu den Primzahlen zählen wäre $\nu$ nicht mehr bijektiv und die
Menge $T$ wäre nicht mehr abzählbar unendlich.

\subsection{Primzahlen}

\begin{align*}
	M = 20215853\\
	X = 20215853
\end{align*}

\subsubsection{Fermat}

\begin{align*}
	a^{X-1} \mod X = 2^{20215852} \mod 20215853 = 14439959 \Rightarrow x \notin P
\end{align*}

\subsubsection{Miller-Rabin}

\begin{align*}
	20215852 / 2 & = 10107926\\
	10107926 / 2 & = 5053963 \Rightarrow k = 2,\,m = 5053963\\
\end{align*}
\begin{align*}
	3^{5053963} \mod X & = 15376602\\
	3^{10107926} \mod X & = 19030657\\
	3^{20215852} \mod X & = 2888708 \Rightarrow x \notin P
\end{align*}

\subsection{Anzahl von Primzahlen}

\begin{align*}
	A & = \frac{10^{101}}{\ln(10^{101})} - \frac{10^{100}}{\ln(10^{100})}\\
	  & = \frac{10^{101}}{101*\ln(10)} - \frac{10^{100}}{100*\ln(10)}\\
	  & = \frac{100*10^{101}}{101*100*2.3} - \frac{101*10^{100}}{101*100*2.3}\\
	  & = \frac{1000*10^{100} - 101*10^{100}}{101*100*2.3}\\
	  & = \frac{899*10^{100}}{101*100*2.3}\\
	  & = \frac{899*10^{97}}{23.23}\\
	  & = 3.87*10^{98}
\end{align*}


\begin{align*}
	W & = {(\frac{10^{101} - 10^{100}}{2})}^{-1} * 3.87*10^{98}\\
	  & = \frac{3.87*10^{98}}{4.5*10^{100}}\\
	  & = 0.0086
\end{align*}

\end{document}
