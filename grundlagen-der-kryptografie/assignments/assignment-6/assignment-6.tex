\documentclass{article}

\usepackage[a4paper]{geometry}
\usepackage[ngerman]{babel}
\usepackage[utf8]{inputenc}
\usepackage[T1]{fontenc}
\usepackage{amsmath}
\usepackage{amsfonts} 
\usepackage{listings}
\usepackage{color}

\renewcommand\thesubsubsection{\textnormal{(\alph{subsubsection})}}

\begin{document}

\begin{titlepage}
	\begin{flushleft}
		TH Brandenburg\\
		Online Studiengang IT Sicherheit\\
		Fachbereich Informatik und Medien\\
		Algorithmen und Datenstrukturen\\
		Prof.\ Dr.\ rer.\ nat.\ Ulrich Baum
	\end{flushleft}

	\vfill

	\begin{center}
		\Large{Einsendeaufgabe 6}\\[0.5em]
		\large{Sommersemester 2022}\\[0.25em]
		\large{Abgabetermin \today}
	\end{center}

	\vfill

	\begin{flushright}
		Mara Schulke \\
		Matrikel-Nr. 20215853
	\end{flushright}
\end{titlepage}

\newpage

\section*{Einsendeaufgabe 6}
\stepcounter{section}
\stepcounter{section}
\stepcounter{section}
\stepcounter{section}
\stepcounter{section}
\stepcounter{section}

\subsection{Zyklische Gruppen}

\subsubsection{}

\begin{gather*}
	O = \{1, 2, 3, 4, 6, 12\}\\
	\vspace{0.5em}\\
	\forall o \in O: o\, |\, 12
\end{gather*}

\subsubsection{}

\begin{align*}
	a^4 \neq a^6 \neq 1 \Rightarrow a\, \textnormal{ist ein Generator für}\, Z_{13}^*,\, \textnormal{da gilt}:
\end{align*}
\begin{align*}
	\forall o_n \in O: \exists n \in Z_{13}^* \land \textnormal{|<}n\textnormal{>|} = o_n: \forall o_m \in \{ o_m \in O: o_n \,|\, o_m \}: n^{o_m}\, = 1
\end{align*}

Anders ausgedrückt gilt für alle Elemente der Ordnung $n$ (also $a^n = 1$) dass
bei einem Vielfachen der Ordnung $nk$ mit $k \in \mathbb{N}: a^{nk} = 1$. Somit
lässt sich aus dem Fakt das $a^{nk} = 1$ ist nicht ableiten ob $a$ die Ordnung
$n$ oder $nk$ (mit $k \neq 1$) hat.

Ist allerdings $a^{nk} \neq 1$ schließt man somit alle Teiler von $nk$ aus.
Also konkreter: Gilt $a^4 \neq 1$ kann $a$ nicht die Ordnung 1, 2 oder 4
haben. Wenn nun zusätzlich $a^6 \neq 1$ gilt kann $a$ auch nicht die Ordnung 1,
2, 3 oder 6 haben. Da jedes Element eine Ordnung haben muss bleibt nun nurnoch
die Ordnung 12 über und wir sind sicher dass $a$ ein Generator für $Z_{13}^*$
ist.


\subsubsection{}

\begin{align*}
	g = xy
\end{align*}

\end{document}
