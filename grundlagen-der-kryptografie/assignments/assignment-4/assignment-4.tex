\documentclass{article}

\usepackage[a4paper]{geometry}
\usepackage[ngerman]{babel}
\usepackage[utf8]{inputenc}
\usepackage[T1]{fontenc}
\usepackage{amsmath}
\usepackage{amsfonts} 
\usepackage{listings}
\usepackage{color}

\renewcommand\thesubsubsection{\textnormal{(\alph{subsubsection})}}

\begin{document}

\begin{titlepage}
	\begin{flushleft}
		TH Brandenburg\\
		Online Studiengang IT Sicherheit\\
		Fachbereich Informatik und Medien\\
		Algorithmen und Datenstrukturen\\
		Prof.\ Dr.\ rer.\ nat.\ Ulrich Baum
	\end{flushleft}

	\vfill

	\begin{center}
		\Large{Einsendeaufgabe 4}\\[0.5em]
		\large{Sommersemester 2022}\\[0.25em]
		\large{Abgabetermin \today}
	\end{center}

	\vfill

	\begin{flushright}
		Mara Schulke \\
		Matrikel-Nr. 20215853
	\end{flushright}
\end{titlepage}

\newpage

\section*{Einsendeaufgabe 4}
\stepcounter{section}
\stepcounter{section}
\stepcounter{section}
\stepcounter{section}

\subsection{Stochastische Unabhängigkeit}

\begin{align*}
	\Pr(\bar{A} \cap B) & = \Pr(B \setminus A)\\
						& = \Pr(B) - \Pr(B \cap A)\\
						& = \Pr(B) - \Pr(A) * \Pr(B)\\
						& = \frac{|B|}{|\Omega|} - \frac{|A| * |B|}{|\Omega|^2}\\
						& = \frac{|B| * |\Omega|}{|\Omega|^2} - \frac{|A| * |B|}{|\Omega|^2}\\
						& = \frac{|B| * (|\Omega| - |A|)}{|\Omega|^2}\\
						& = \frac{|B| * |\bar{A}|}{|\Omega|^2}\\
						& = \frac{|B|}{|\Omega|} * \frac{|\bar{A}|}{|\Omega|}\\
						& = \Pr(B) * \Pr(\bar{A})
\end{align*}

\subsection{Kombinatorik}

Anzahl aller möglichen Aufteilungen in Teams mit 5 Mitgliedern:

\begin{align*}
	{10 \choose 5} & = \frac{10!}{5! * 5!}\\
				   & = \frac{10 * 9 * 8 * 7 * 6}{5 * 4 * 3 * 2 * 1}\\
				   & = \frac{2 * 9 * 8 * 7 * 6}{4 * 3 * 2 * 1}\\
				   & = \frac{2 * 9 * 2 * 7 * 6}{3 * 2 * 1}\\
				   & = \frac{2 * 3 * 2 * 7 * 6}{2 * 1}\\
				   & = 3 * 2 * 7 * 6\\
				   & = 7 * 6^2\\
				   & = 252\\
\end{align*}

Da die Reihenfolge nicht gefragt ist, sondern nur die möglichen Aufteilung sind
$(\{1,2,3,4,5\}, \{6,7,8,9,10\})$ und $(\{6,7,8,9,10\}, \{1,2,3,4,5\})$
in diesem Kontext äquivalent. Demnach:

\begin{align*}
	\frac{{10 \choose 5}}{2} = \frac{252}{2} = 126
\end{align*}

\end{document}
