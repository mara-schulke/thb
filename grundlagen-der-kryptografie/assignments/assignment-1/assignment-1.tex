\documentclass{article}

\usepackage[a4paper]{geometry}
\usepackage[ngerman]{babel}
\usepackage[utf8]{inputenc}
\usepackage[T1]{fontenc}
\usepackage{amsmath}
\usepackage{listings}
\usepackage{color}

\begin{document}

\begin{titlepage}
	\begin{flushleft}
		TH Brandenburg\\
		Online Studiengang IT Sicherheit\\
		Fachbereich Informatik und Medien\\
		Algorithmen und Datenstrukturen\\
		Prof.\ Dr.\ rer.\ nat.\ Ulrich Baum
	\end{flushleft}

	\vfill

	\begin{center}
		\Large{Einsendeaufgabe 1}\\[0.5em]
		\large{Sommersemester 2022}\\[0.25em]
		\large{Abgabetermin \today}
	\end{center}

	\vfill

	\begin{flushright}
		Mara Schulke \\
		Matrikel-Nr. 20215853
	\end{flushright}
\end{titlepage}

\newpage

\section*{Einsendeaufgabe 1}
\stepcounter{section}

\subsection{Induktionsbeweis}

\begin{align*}
	Induktionsvoraussetzung & = \sum_{i=1}^{n}3^{i - 1} = \frac{3^n-1}{2}       \\
	Induktionsbehauptung    & = \sum_{i=1}^{n+1}3^{i - 1} = \frac{3^{n+1}-1}{2} \\
	Induktionsanfang        & = \sum_{i=1}^{1}3^0 = \frac{3^1 - 1}{2} = 1       \\
	Induktionsschritt       & = \sum_{i=1}^{n+1}3^{i - 1}                       \\
							& = \sum_{i=1}^{n}3^{i - 1} + 3^n                   \\
							& = \frac{3^n - 1}{2} + \frac{2 * 3^n}{2}           \\
							& = \frac{3 * 3^n - 1}{2}                           \\
							& = \frac{3^{n + 1} - 1}{2}                         \\
\end{align*}

\subsection{Euklidischer Algorithmus}

\begin{table}[h]
	\begin{center}
		\begin{tabular}{c|c|c}
			a   & b   & r\\
			\hline
			374 & 284 & 90\\
			284 & 90  & 14\\
			90  & 14  & 6\\
			14  & 6   & \textbf{2}\\
			6   & \textbf{2} & 0\\
		\end{tabular}
		\vspace{1em}\\
		gcd(374, 284) = 2
	\end{center}
\end{table}

\end{document}
