\documentclass{article}

\usepackage[style=apa]{biblatex}
\usepackage[a4paper, left=1.5cm, right=1.5cm]{geometry}
\usepackage[ngerman]{babel}
\usepackage[utf8]{inputenc}
\usepackage[T1]{fontenc}
\usepackage{hyperref}
\usepackage{xcolor}
\usepackage{csquotes}
\usepackage{doc}

\hypersetup{colorlinks=true, allcolors=black}

\bibliography{references}
\pagestyle{empty}

\begin{document}

Sichere Kommunikation ist eine Grundvoraussetzung im Smart Home. Mesh Netzwerke
bieten viele Vorteile, wie eine hohe Ausfallsicherheit und leichte
Erweiterbarkeit. Bluetooth Mesh und Thread sind zwei sehr unterschiedliche Low
Power Mesh-Protokolle mit hoher Relevanz für die IoT-Industrie
\parencite{ThreadMeshVsOtherWirelessIEEE, HardwareSecurity, BLEMeshEmbeddedComputingCom}.

Im Gegensatz zu Thread baut Bluetooth Mesh auf dem weit verbreiteten Protokolle
Bluetooth Low Energy auf. Thread verwendet, ähnlich wie sein Vorgänger Zigbee
IP, IEEE 802.15.4 für die drahtlose Datenübertragung \parencite{ThreadSpec, BluetoothSpec}.
Diese Arbeit grenzt sich von vorangegangenen Arbeiten durch den
sicherheitsbezogenen Vergleich der beiden Protokolle ab. Wir präsentieren die
Ergebnisse unserer Literaturanalyse von Sicherheitsuntersuchungen und
Spezifikationen.

\textbf{On-Mesh Commissioning:} Der Commissioner kann dem Joiner festgelegte
Credentials bereitstellen um dem Netzwerk beizutreten. Er kann diese
Credentials auf die vom Hersteller vergebene EUI64 beschränken
\parencite{ThreadSpec}.

\textbf{External Commissioning:} Der authentifizierte Commissioner stellt die
Identität des Joiners sicher. Anschließend weist er den Border-Router an, dem
Joiner alle Informationen zum Beitritt des Netzwerks bereitzustellen
\parencite{ThreadSpec}.

Thread ist ein offenes Protokoll, da es im Vergleich zu Bluetooth Mesh, kaum
einen Einfluss auf die Anwendungsebene des Netzwerks hat.

Im Jahr 2020 wurde eine Sicherheitsanalyse von Thread anhand der 10
relevantesten Sicherheitsbedenken der OWASP für IoT-Netzwerke durchgeführt
\parencite{ThreadSecurityCSIAC}. Diese Analyse zeigt, dass Thread den Fokus
primär auf die Verschlüsselung und die Sicherung der Übertragung legt, aber
Risiken durch die fehlende Spezifikation der Anwendungsebene entstehen
könnten. Hersteller müssen selbst für die Sicherheit auf der Anwendungsebene
sorgen. Das macht es möglich, bereits gut erforschte
Verschlüsselungsprotokolle, wie z. B. DTLS, zu verwenden und diese im
Bedarfsfall auszu-tauschen. Diese Freiheit birgt aber auch Risiken, da sich der
Nutzer auf die Sicher-heitsmaßnahmen des Hersteller verlassen muss.


Dinu und Kizhvatov (\citeyear{ThreadEMAttack}) haben herausgefunden, dass durch
eine Verkettung von Schwachstellen und einer differenziellen
elektromagnetischen Analyse Rückschlüsse auf den Datenverkehr bei Thread zu
ziehen ist, und Netzwerkschlüssel abgegriffen werden können. Dies lohnt sich
allerdings aufgrund des hohen Aufwands i.d.R. bei Smart-Home-Netzwerken nicht,
sondern stellt eher eine Gefahr für kommerzielle Anwendungen dar
\parencite{ThreadEMAttack}.


In den letzten Jahren wurden mehrere, teils kritische, Sicherheitslücken in der
Bluetooth Spezifikation entdeckt \parencite{BluetoothIssues}.
Diese beeinträchtigen zum Großteil die Sicherheit von Direktverbindungen - also
Bluetooth BR/EDR und Bluetooth Low Energy (BLE). Zu den schwerwiegendsten
Schwachstellen in der Bluetooth Core Spezifikation zählen die sogenannten
“Bluetooth Impersonation Attacks”. Diese macht es möglich, dass sich ein
Angreifer als bereits authentifiziertes Gerät ausgibt und somit die komplette
Authentifizierung bei einem Verbindungsaufbau umgehen kann. Dies ermöglicht
Man-In-The-Middle-Attacken \parencite{BluetoothLowEnergyAttackOxford}, durch
die Daten ausgelesen, Daten manipuliert oder gefälschte Nachrichten in Umlauf
gebracht werden können.

Bluetooth Mesh verwendet als Übertragungsprotokoll BLE und verwendet, anders
als beim BLE-Pairing, elliptische 256-Bit-Kurven und
Out-of-Band-Authentifizierung um das Hinzufügen von Netzwerkknoten abzusichern
\parencite{BluetoothSpec}.
Allerdings wurden diesbezüglich im Jahr 2020 die Sicherheitswarnungen
CVE-2020-26556, CVE–2020-26557, CVE-2020-26559, CVE-2020-26560 veröffentlicht
\parencite{BluetoothIssues}.

Die aktuell bekannten Sicherheitslücken von Bluetooth Mesh betreffen das
Hinzufügen von neuen Netzwerkknoten \parencite{BluetoothIssues} und vermindern
somit die Integrität des Netzwerks. Bislang wurden bei Thread, trotz
ausgiebiger Untersuchungen \parencite{ThreadSecurityCSIAC, ThreadSecurityEmbeddedCom},
noch keine vergleichbaren Schwachstellen gefunden. Auf der anderen Seite tritt
Thread die Verantwortung zu Absicherung der Anwendungsschicht an den Hersteller
ab, ist somit zwar flexibler, könnte aber mögliche Sicherheitslücken durch
nicht abgesicherte Geräte öffnen. Eine eindeutige Empfehlung für oder gegen
eines der beiden Protokolle nur auf Basis der Sicherheit kann nicht klar
gegeben werden, da beide Protokolle unterschiedliche Vorteile und
Sicherheitsrisiken mit sich bringen. Eine Entscheidung muss vom
Implementierung-Kontext abhängig gemacht werden.

\printbibliography[heading=none]

\end{document}
