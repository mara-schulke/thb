\documentclass{article}

\usepackage[a4paper]{geometry}
\usepackage[ngerman]{babel}
\usepackage[utf8]{inputenc}
\usepackage[T1]{fontenc}
\usepackage{amsmath}
\usepackage{listings}
\usepackage{color}

\lstset{
	language=Python,
	aboveskip=3mm,
	belowskip=3mm,
	showstringspaces=false,
	columns=flexible,
	basicstyle={\small\ttfamily},
	numbers=none,
	breaklines=true,
	breakatwhitespace=true,
	tabsize=4
}

\renewcommand\thesubsection{\alph{subsection})}

\begin{document}

\begin{titlepage}
	\begin{flushleft}
		TH Brandenburg \\
		Online Studiengang Medieninformatik \\
		Fachbereich Informatik und Medien \\
		Algorithmen und Datenstrukturen \\
		Prof. Dr. rer. nat. Ulrich Baum
	\end{flushleft}

	\vfill

	\begin{center}
		\Large{Einsendeaufgabe 2}\\[0.5em]
		\large{Sommersemester 2021}\\[0.25em]
		\large{Abgabetermin 24.05.2021}
	\end{center}

	\vfill

	\begin{flushright}
		Mara Schulke \\
		Matrikel-Nr. 20215853
	\end{flushright}
\end{titlepage}

\newpage

\section{Finde die falsche Münze}

\subsection{Man kann das Problem lösen, indem man die erste Münze nacheinander
mit allen anderen vergleicht. Geben Sie den Worst-Case und Best-Case Aufwand
dafür an.}

Falls der Best-Case entritt (die falsche Münze liegt an einer der ersten beiden
Positionen) haben wir nur einen einzigen Vergleich dementsprechen $O(1)$. Der
Worst- Case tritt dementsprechen im umgekehrten Fall ein, also wenn die falsche
Münze ganz am Ende der Folge liegt - dann benötigen wir $n - 1$ Vergleiche und
liegen somit in $O(n)$.

\subsection{Entwerfen und beschreiben Sie umgangssprachlich einen rekursiven
Divide-And-Conquer Algorithmus, der das Problem für große $n$ deutlich
schneller löst.}

Am besten geht man hier mit einer modifizierten Version der binären Suche vor.
Da $n$ immer glatt durch 2 teilbar ist, haben wir immer eine identische Menge
an Münzen auf der linken und auf der rechten Seite, wenn wir die Ausgangsmenge
in der exakten Mitte teilen. Nun kann man diese beiden Mengen miteinander
vergleichen. Falls die Anzahl der Münzen in jeder Hälfte größer als 1 ist, muss
dieser Vorgang wiederholt werden, bis nurnoch 2 Münzen miteinander verglichen
werden, dann ist die leichtere der beiden die falsche Münze.

\subsection{Analysieren Sie den Worst-Case und Best-Case Aufwand dafür. Zählen
Sie die Anzahl der Wiegevorgänge dabei genau (nicht asymptotisch)}

Nun unterscheidet sich der o.g. Algorithmus von der normalen binären Suche
dadurch, dass wir keinen Vergleich mit einem einzelnen Element haben, mit dem
wir vorzeitig abbrechen könnten. Wir müssen also jedes mal, selbst wenn die
falsche Münze in der Mitte liegt, die Hälften immer ganz abarbeiten was bei 2k
Elementen zu genau $log_2 n$ oder anders ausgedrückt $k$ Vergleichen führt.

\section{Modifizierter Mergesort}

\subsection{Beschreiben Sie die Arbeitsweise von $merge4$ umgangssprachlich.
Bestimmen Sie die worst-case-Anzahl von Vergleichen von Vergleichen für
$merge4$ auf 4 gleich großen sortierten Folgen von je $m$ Elementen. Zählen Sie
die Vergleiche genau (keine O-Notation).}

Beim normalen $merge$ werden zwei in sich sortierten Listen zu einer großen
sortierten Liste zusammen gefügt, in dem immer das kleinere der beiden Elemente
der beiden Teillisten zu erst in die neue Liste geschrieben wird, dieser
Vorgang kostet einen Vergleich.

Nun müssen bei $merge4$ nicht 2 sondern 4 Listen verglichen werden, es muss
also bei jedem Vorgang das minimum von 4 Elementen identifiziert werden, das
kostet 3 Vergleiche. Sobald die Anzahl aller Elemente der Teillisten 4 ist
brauchen wir nur $3!$ Vergleiche um die restlichen Elemente einzusortieren.

Also anders ausgedrückt braucht $merge4$ im Worst-Case $((m - 1) * 4 * 3) + 3!$
Vergleiche.  Der Worst-Case tritt ein, wenn die Elemente gleichmäßig auf die
Teillisten verteilt ist, also bei $m = 2$ die erste Liste $[1, 5]$, die zweite
Liste $[2, 6]$, die dritte Liste $[3, 7]$ und die vierte Liste $[4, 8]$ ist.
Diese Aufteilung verhindert das vorzeitige Abarbeiten einer einzelenen Liste
wodurch Vergleiche eingespart werden könnten.

\subsection{T(n) sei die worst-case Anzahl von Vergleichen des
Sortierverfahrens auf einer Inputfolge der Länge $n$. Nehmen Sie an, dass $n$
ein Vielfaches von 4 ist.  Stellen Sie die Rekursionsgleichung für T(n) auf.}

\begin{align*}
	         T(4) & = 0                                        \\
	         T(n) & = \, 4T(\frac{n}{4}) + merge4(\frac{n}{4}) \\
	\\
	    merge4(m) & = ((m - 1) * 4 * 3) + 3!
\end{align*}

\subsection{Lösen Sie die Rekursionsgleichung asymptotisch mit dem
Master-Theorem}

\begin{align*}
	         T(n) & = \, aT(\frac{n}{b}) + merge4(\frac{n}{b}) \\
		merge4(m) & \in \Theta(m^d)                            \\
				  & \Rightarrow a = 4                          \\
				  & \Rightarrow b = 4                          \\
				  & \Rightarrow c = 1                          \\
				  & \Rightarrow d = \log_b a                   \\
				  & \Rightarrow T(n) \in \Theta(n \log_2 n)
\end{align*}


\subsection{Wird diese Variante in der Praxis schneller sein als der
Standard-Mergesort?}

Nein. Der Aufwand der einzelenen Merge-Vorgänge ist deutlich höher, da viel mehr
Elemente gleichzeitig betrachtet werden müssen. Grob betrachtet ist der Aufwand
für $merge2$ für 2 Elemente der Länge $m$ $= 2m - 1$, der Aufwand für $merge4$
war ja bereits oben als $((m - 1) * 4 * 3) + 3!$ definiert. Also bei 16
Elementen wäre der Aufwand von $merge2(8) = 2 * 8 - 1 = 15$ und der Aufwand von
$merge4(4) = ((4 - 1) * 4 * 3) + 3! = 36 + 6 = 42$.  Die größten Bestandteile
der Laufzeit von $merge2$ bzw. $merge4$ sind bei $merge2 = 2m$ und bei $merge4
= 12(m - 1)$ – somit wird $merge4$ in der Praxis immer langsamer sein.

\newpage

\section{Umdrehen von Listen}

\begin{lstlisting}
def reverse(head):
    first = copy(head)

    while first.next is not Null:
        old_head = copy(head)
        head = first.next
        first.next = head.next
        head.next = old_head

    return head
\end{lstlisting}


\section{Datenstrom-Analyse}
Sie sollen die Umsätze eines Web-Shops kontinutierlich überwachen und erhalten
zu jedem Verkauf den Umsatz als Zahl übermittelt. Ihr Input ist also ein
beliebig langer fortlaufender Strom von Zahlen. Weiterhin wird vom Anwender
beim Start des Algorithmus eine natürliche Zahl $k > 1$ gewählt. $k$ bleibt
während der Laufzeit unverändert. Das zu lösende Problem besteht aus zwei
Teilproblemen:

\begin{enumerate}
	\item{Jede neu eingehende Zahl einzeln geeignet verarbeiten,
		damit Sie die Anfragen in 2. beantworten können.}
	\item{Jederzeit auf Anfrage die $k$ kleinsten alle rbisher empfangenen
		Umsatzzahlen irgendeiner Reihenfolge auszugeben. (Sie können davon
		ausgehen, dass Anfragen erst gestellt werden, wenn schon mindestens $k$
		Zahlen eingegangen sind.)}
\end{enumerate}

\subsection{Entwerfen und beschreiben Sie umgangssprachlich einen Algorithmus,
der die beiden Teilprobleme unter Verwendung einer Prioritätswarteschlange
effizient löst. Die Laufzeit Ihres Algorithmus darf nur von $k$ abhängen,
nicht aber von der Anzahl der empfangenen Umsatzzahlen!  Brauchen Sie dafür
eine Maximum- oder Minimum-Prioritätswarteschlange?}

Es muss lediglich eine einzige Maximum-Prioritätswarteschlange der Länge $k$
gepflegt werden.  Sind noch weniger als $k$ Zahlen in der
Prioritätswarteschlange, kann die Zahl in jedem Fall einfach über $insert()$
eingefügt werden.

Ansonsten muss die eintreffende Zahl mit der jeweils höchsten Zahl bzw. dem
Root der Prioritätswarteschlange verglichen werden. Wenn die neue Zahl kleiner
als die alte Root ist, qualifiziert sich die neue dafür in den $k$ kleinsten
der bisher eingetroffenen Zahlen zu sein. Wenn die eintreffende Zahl
qualifiziert ist, kann sie über $insert()$ eingefügt werden und anschließend
kann die größte Zahl über $removeMax()$ entfernt werden.

Sobald eine Anfrage eintrifft, die $k$ kleinsten Zahlen auszugeben, gibt es 2
Möglichkeiten.  Entweder es wird eine Referenz zur Prioritätswarteschlange
zurückgegeben, was zwar sehr effizient ist, aber unter Umständen ungewünschte
Seiteneffekte nach sich ziehen könnte, falls jemand von ''außen'' in die
Prioritätswarteschlange schreibt. Alternativ, könnte eine ineffizientere, aber
sichere Variante sein, eine Kopie der Inhalte der Prioritätswarteschlange zu
erstellen und diese zurückzugegeben - so können ungewünschte Seiteneffekte
vermieden werden, außerdem würden sich die Daten des Abfragenden nach der
Abfrage nicht mehr ändern. Würde die Referenz zurückgegeben werden, müsste
quasi nur ein einziges Mal eine Anfrage gestellt werden, da jedes Mal über die
Referenz zugegriffen werden könnte.

Im Zuge der Sicherheit würde ich zu der langsameren aber stabilieren Variante -
der Kopie - raten.

\subsection{Analysieren Sie den asymptotischen Aufwand Ihres Algorithmus für
Teilproblem 1 und 2 in Abhängigkeit von $k$.}

Der Vorgang einen Umsatz zu empfangen, liegt im BestCase in $\Omega(1)$ (falls
die Zahl größer als das alte Maximum ist). Im WorstCase muss der neue Wert
eingefügt und der alte Wert entfernt werden. Ein $insert()$ in eine
Prioritätswarteschlange kostet $log_2k$ und ein $removeMax()$ kostet $2log_2k$.
Somit ist der Gesamtaufwand im WorstCase ca. $3log_2k$ und liegt daher in
$O(log_2k)$. Der AverageCase dürfte in der Praxis weit unter $log_2k$ liegen
(natürlich total davon abhängig davon wie viele Zahlen aus welchem
Zahlenbereich), da bei unendlich vielen Zahlen die Wahrscheinlichkeit zunehmend
abnimmt, dass eine Zahl erfasst wird, die kleiner als die höchste der
bisherigen Zahlen ist.

Die Ausgabe der $k$ kleinsten Zahlen bei einer Anfrage zu tätigen, muss die $k$
Zahlen lange Prioritätswarteschlange kopiert werden - dies liegt in
$\Theta(k)$.

Um nun eine Zusammenfassende Auskunft über den tatsächlichen AverageCase zu
treffen fehlen Eckdaten, wie der (bisherige) maximale Umsatz, häufigkeit der
Abfragen und Wahrscheinlichkeit bestimmter Umsatzgrößen.

\end{document}

