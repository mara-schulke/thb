\documentclass{article}

\usepackage[a4paper]{geometry}
\usepackage[ngerman]{babel}
\usepackage[utf8]{inputenc}
\usepackage[T1]{fontenc}

\begin{document}

\begin{titlepage}
	\begin{flushleft}
		TH Brandenburg \\
		Online Studiengang Medieninformatik \\
		Fachbereich Informatik \\
		Mensch-Computer-Interaktion
	\end{flushleft}

	\vfill

	\begin{center}
		\Large{Einsendeaufgabe 1: Personas und Storyboards}\\[0.5em]
		\large{Sommersemester 2021}\\[0.25em]
		\large{Abgabetermin 18.04.2021}
	\end{center}

	\vfill

	\begin{flushright}
		Maximilian Schulke \\
		Matrikel-Nr. 20215853
	\end{flushright}
\end{titlepage}

\tableofcontents

\vfill

\section{Aufgabenstellung}

Folgende Aufgabenstellung wurde im Moodle-Kurs bekannt gegeben:

\begin{quote}
	Sie haben die Aufgabe eine Smartphone-App zu konzipieren, die OSMI-Studierende beim Studium unterstützt.
	Die App soll insbesondere die Kommunikation, die gegenseitige Unterstützung und das gemeinsame Bearbeiten
	von Aufgaben und Projekten unterstützen. Denken Sie auch darüber nach, wie die App in der aktuellen
	SARS-CoV-2-Situation helfen kann (also beispielsweise auch Studierende, die normalerweise in Präsenz studieren).
	\\[1em]
	(a) Erstellen Sie zwei Personas der (potentiellen) Zielgruppe.

	(b) Erstellen Sie für jede Persona zwei Storyboards (also insgesamt vier) zur Nutzung der App. Storyboards
	werden in der Regel von Hand gezeichnet (auf Schönheit kommt es in diesem Zusammenhang nicht an). Scannen
	oder fotografieren Sie die Skizzen und beschreiben Sie ggf. kurz wie die Skizzen verstanden werden sollen.
	\\[1em]
	Anmerkungen:

	Abzugeben ist ein PDF-Dokument, das Ihre Ausführungen enthält. Bitte beachten Sie dazu die Hinweise zu den
	Einsendeaufgaben (siehe oben). Der zeitliche Umfang dieser Einsendeaufgabe wird auf 6 Stunden geschätzt.
	Ihre Ausarbeitung sollte ca. 7-9 Seiten (A4) umfassen (eine Seite für Titelblatt inkl. Aufgabenstellung,
	je eine Seite pro Persona, und je eine pro Storyboard sowie ggf. zusätzliche Seiten für die Erläuterungen).
	Die Lösung, die Sie zur Deadline abgeben, sollte eine aus Ihrer Sicht endgültige Lösung sein. Falls Sie
	Fragen zur Aufgabenstellung haben, stellen Sie diese bitte im Vorfeld!
\end{quote}

\newpage

\section{Persona: Mark}

\begin{tabularx}{0.8\textwidth} { 
	| >{\raggedright\arraybackslash}X 
	| >{\centering\arraybackslash}X 
	| >{\raggedleft\arraybackslash}X | }
	\hline
	item 11 & item 12 & item 13 \\
	\hline
	item 21  & item 22  & item 23  \\
	\hline
\end{tabularx}

Name

"quote"

OSMI Studenten, die berufsbegleitend studieren
Alter
Beruf höchstem Bildungsabschluss (Realschulabschluss/ Abitur/ Ausbildung/Studium)
Familienstand,
Kinder?
Motivation für Entscheidung für OSMI?
Hobbies
Wohnort (könnte z.B. auch auf dem Land wohnen oder halt in Berlin)
OSMI-Student seit wann?
wie verbringe ich den Tag?
Konsumverhalten?
Bedürfnisse
Wünsche
Werte
Ängste
Introvertiert
Extrovertiert

\newpage

\section{Persona: XYZ}

\newpage

\section{Storyboard 1: Mark}

\newpage

\section{Storyboard 2: Mark}

\newpage

\section{Storyboard 3: XYZ}

\newpage

\section{Storyboard 4: XYZ}

\newpage


\end{document}