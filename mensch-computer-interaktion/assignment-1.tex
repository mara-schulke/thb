\documentclass{article}

\usepackage[a4paper]{geometry}
\usepackage[ngerman]{babel}
\usepackage[utf8]{inputenc}
\usepackage[T1]{fontenc}

\begin{document}

\begin{titlepage}
	\begin{flushleft}
		TH Brandenburg \\
		Online Studiengang Medieninformatik \\
		Fachbereich Informatik \\
		Mensch-Computer-Interaktion
	\end{flushleft}

	\vfill

	\begin{center}
		\Large{Einsendeaufgabe 1: Personas und Storyboards}\\[0.5em]
		\large{Sommersemester 2021}\\[0.25em]
		\large{Abgabetermin 18.04.2021}
	\end{center}

	\vfill

	\begin{flushright}
		Maximilian Schulke \\
		Matrikel-Nr. 20215853
	\end{flushright}
\end{titlepage}

\tableofcontents

\vfill

\section{Aufgabenstellung}

Folgende Aufgabenstellung wurde im Moodle-Kurs bekannt gegeben:

\begin{quote}
	Sie haben die Aufgabe eine Smartphone-App zu konzipieren, die OSMI-Studierende beim Studium unterstützt.
	Die App soll insbesondere die Kommunikation, die gegenseitige Unterstützung und das gemeinsame Bearbeiten
	von Aufgaben und Projekten unterstützen. Denken Sie auch darüber nach, wie die App in der aktuellen
	SARS-CoV-2-Situation helfen kann (also beispielsweise auch Studierende, die normalerweise in Präsenz studieren).
	\\[1em]
	(a) Erstellen Sie zwei Personas der (potentiellen) Zielgruppe.

	(b) Erstellen Sie für jede Persona zwei Storyboards (also insgesamt vier) zur Nutzung der App. Storyboards
	werden in der Regel von Hand gezeichnet (auf Schönheit kommt es in diesem Zusammenhang nicht an). Scannen
	oder fotografieren Sie die Skizzen und beschreiben Sie ggf. kurz wie die Skizzen verstanden werden sollen.
	\\[1em]
	Anmerkungen:

	Abzugeben ist ein PDF-Dokument, das Ihre Ausführungen enthält. Bitte beachten Sie dazu die Hinweise zu den
	Einsendeaufgaben (siehe oben). Der zeitliche Umfang dieser Einsendeaufgabe wird auf 6 Stunden geschätzt.
	Ihre Ausarbeitung sollte ca. 7-9 Seiten (A4) umfassen (eine Seite für Titelblatt inkl. Aufgabenstellung,
	je eine Seite pro Persona, und je eine pro Storyboard sowie ggf. zusätzliche Seiten für die Erläuterungen).
	Die Lösung, die Sie zur Deadline abgeben, sollte eine aus Ihrer Sicht endgültige Lösung sein. Falls Sie
	Fragen zur Aufgabenstellung haben, stellen Sie diese bitte im Vorfeld!
\end{quote}

\newpage

\section{Persona: Mark}

Name					Marc Zahn
Alter					28
Beruf					Hochschul-Verwaltungsangestellter
Bildungsabschluss		Abitur
Familienstand			Verheiratet
Kinder					Ja
Motto					""
Motivation für OSMI		Möchte nun Umlernen, Kontakt zum Studium durch Job an der Hochschule
Hobbies					Alte Filme sammeln, Kochen, Wandern
Wohnort					Brandenburg a.d.H.
OSMI-Student seit		WS 2019
Alltag?					Kinder Aufwecken, Frühstück für Familie machen, Arbeiten, Studieren
Bedürfnisse				Erfolg, Soziale Anerkennung, Sicherheit für Familie
Wünsche					Neuer Job in Wunschbereich
Werte					Treue, Zuverlässigkeit, Ehrlichkeit
Ängste					Angst zu Versagen, Unsichere Zukunft
Persönlichkeit			Introvertiert

\newpage

\section{Persona: XYZ}

Name					Rosa Beetz
Alter					23
Beruf					Screen Desginerin
Bildungsabschluss		Ausbildung
Familienstand			Ledig
Kinder					Nein
Motto					""
Motivation für OSMI		Möchte sich in Richtung Informatik weiterentwickeln und sich selbst etwas Beweisen
Hobbies					Netflix, Feiern, Kaffee
Wohnort					Berlin Wedding
OSMI-Student seit		SS 2021
Alltag					Arbeit, Freunde treffen \& Netflix
Bedürfnisse				Sozialer Kontakt, Anerkennung
Wünsche					Mehr Geld, Anerkennung im Job
Werte					Freundlichkeit, Ehrgeiz
Ängste					Angst zu Versagen
Persönlichkeit			Extrovertiert

\newpage

\section{Storyboard 1: Mark}

Als OSMI Student möchte ich Fahrgemeinschaften gründen

\newpage

\section{Storyboard 2: Mark}

Als OSMI Student möchte live mit anderen Studenten in einem Whiteboard oder einer PDF zusammenarbeiten

\newpage

\section{Storyboard 3: XYZ}

\newpage

\section{Storyboard 4: XYZ}

Als OSMI Student möchte ich mit anderen Studenten Videotelefonieren

\newpage


\end{document}