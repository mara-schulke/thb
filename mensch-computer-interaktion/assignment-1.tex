\documentclass{article}

\usepackage[a4paper]{geometry}
\usepackage[ngerman]{babel}
\usepackage[utf8]{inputenc}
\usepackage[T1]{fontenc}
\usepackage{tabularx}

\begin{document}

\begin{titlepage}
	\begin{flushleft}
		TH Brandenburg \\
		Online Studiengang Medieninformatik \\
		Fachbereich Informatik und Medien \\
		Mensch-Computer-Interaktion
	\end{flushleft}

	\vfill

	\begin{center}
		\Large{Einsendeaufgabe 1: Personas und Storyboards}\\[0.5em]
		\large{Sommersemester 2021}\\[0.25em]
		\large{Abgabetermin 18.04.2021}
	\end{center}

	\vfill

	\begin{flushright}
		Maximilian Schulke \\
		Matrikel-Nr. 20215853
	\end{flushright}
\end{titlepage}

\tableofcontents

\vfill

\section{Aufgabenstellung}

Folgende Aufgabenstellung wurde im Moodle-Kurs bekannt gegeben:

\begin{quote}
	Sie haben die Aufgabe eine Smartphone-App zu konzipieren, die OSMI-Studierende beim Studium unterstützt.
	Die App soll insbesondere die Kommunikation, die gegenseitige Unterstützung und das gemeinsame Bearbeiten
	von Aufgaben und Projekten unterstützen. Denken Sie auch darüber nach, wie die App in der aktuellen
	SARS-CoV-2-Situation helfen kann (also beispielsweise auch Studierende, die normalerweise in Präsenz studieren).
	\\[1em]
	(a) Erstellen Sie zwei Personas der (potentiellen) Zielgruppe.

	(b) Erstellen Sie für jede Persona zwei Storyboards (also insgesamt vier) zur Nutzung der App. Storyboards
	werden in der Regel von Hand gezeichnet (auf Schönheit kommt es in diesem Zusammenhang nicht an). Scannen
	oder fotografieren Sie die Skizzen und beschreiben Sie ggf. kurz wie die Skizzen verstanden werden sollen.
	\\[1em]
	Anmerkungen:

	Abzugeben ist ein PDF-Dokument, das Ihre Ausführungen enthält. Bitte beachten Sie dazu die Hinweise zu den
	Einsendeaufgaben (siehe oben). Der zeitliche Umfang dieser Einsendeaufgabe wird auf 6 Stunden geschätzt.
	Ihre Ausarbeitung sollte ca. 7-9 Seiten (A4) umfassen (eine Seite für Titelblatt inkl. Aufgabenstellung,
	je eine Seite pro Persona, und je eine pro Storyboard sowie ggf. zusätzliche Seiten für die Erläuterungen).
	Die Lösung, die Sie zur Deadline abgeben, sollte eine aus Ihrer Sicht endgültige Lösung sein. Falls Sie
	Fragen zur Aufgabenstellung haben, stellen Sie diese bitte im Vorfeld!
\end{quote}

\newpage

\section{Persona: Mark Zahn}

\begin{tabularx}{\textwidth}{|l|X|}
	\hline
	Name                & Marc Zahn                                           \\
	\hline
	Alter               & 28                                                  \\
	\hline
	Beruf               & Hochschul-Verwaltungsangestellter                   \\
	\hline
	Bildungsabschluss   & Abitur                                              \\
	\hline
	Familienstand       & Verheiratet                                         \\
	\hline
	Kinder              & Ja                                                  \\
	\hline
	Motto               & ""                                                  \\
	\hline
	Motivation für OSMI & Möchte nun Umlernen, bekam Kontakt zum Studium als
	Umschulungsmöglichkeit, durch Job an der Hochschule                       \\
	\hline
	Hobbies             & Alte Filme sammeln, Kochen, Wandern                 \\
	\hline
	Wohnort             & Brandenburg an der Havel                            \\
	\hline
	OSMI-Student seit   & WS 2019                                             \\
	\hline
	Bedürfnisse         & Erfolg, Soziale Anerkennung, Sicherheit für Familie \\
	\hline
	Wünsche             & Neuer Job in Wunschbereich                          \\
	\hline
	Werte               & Treue, Zuverlässigkeit, Ehrlichkeit                 \\
	\hline
	Ängste              & Angst zu Versagen, Unsichere Zukunft                \\
	\hline
	Persönlichkeit      & Introvertiert                                       \\
	\hline
\end{tabularx}

\newpage

\section{Persona: Rosa Beetz}

\begin{tabularx}{\textwidth}{|l|X|}
	\hline
	Name                & Rosa Beetz                                          \\
	\hline
	Alter               & 23                                                  \\
	\hline
	Beruf               & Screen Desginerin                                   \\
	\hline
	Bildungsabschluss   & Ausbildung                                          \\
	\hline
	Familienstand       & Ledig                                               \\
	\hline
	Kinder              & Nein                                                \\
	\hline
	Motto               & ""                                                  \\
	\hline
	Motivation für OSMI & Möchte sich in Richtung Informatik weiterentwickeln
	und sich selbst etwas Beweisen                                            \\
	\hline
	Hobbies             & Netflix, Feiern, Kaffee                             \\
	\hline
	Wohnort             & Berlin Wedding                                      \\
	\hline
	OSMI-Student seit   & SS 2021                                             \\
	\hline
	Bedürfnisse         & Sozialer Kontakt, Anerkennung                       \\
	\hline
	Wünsche             & Mehr Geld, Anerkennung im Job                       \\
	\hline
	Werte               & Freundlichkeit, Ehrgeiz                             \\
	\hline
	Ängste              & Angst zu Versagen                                   \\
	\hline
	Persönlichkeit      & Extrovertiert                                       \\
	\hline
\end{tabularx}

% Name					Rosa Beetz
% Alter					23
% Beruf					Screen Desginerin
% Bildungsabschluss		Ausbildung
% Familienstand			Ledig
% Kinder					Nein
% Motto					""
% Motivation für OSMI		


\newpage

\section{Storyboard 1: Mark}


\newpage

\section{Storyboard 2: Mark}

Als OSMI Student möchte live mit anderen Studenten in einem Whiteboard oder einer PDF zusammenarbeiten

\newpage

\section{Storyboard 3: Rosa Beetz}

Als OSMI Student möchte ich Fahrgemeinschaften gründen

\newpage

\section{Storyboard 4: Rosa Beetz}

Als OSMI Student möchte ich mit anderen Studenten Videotelefonieren

\newpage


\end{document}