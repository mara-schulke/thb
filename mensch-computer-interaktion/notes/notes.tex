\documentclass{article}

\usepackage[a4paper, margin=1em]{geometry}
\usepackage[dvipsnames]{xcolor}
\usepackage[ngerman]{babel}
\usepackage[utf8]{inputenc}
\usepackage[T1]{fontenc}
\usepackage{multicol}
\usepackage{plex-mono}

\begin{document}

\ttfamily

\begin{multicols}{3}
	\begin{flushleft}
		\begin{tiny}
			\textcolor{LimeGreen}{\textbf{\scriptsize{\# Bestimmung des
				Kontextes}}}

			\textbf{\# Setze Kontext, Nutzungskontext, Gebrauchtstglkt.\ in
				Bezug}

			Der Nutzungskontext ist wichtig zur Ermittlung der
			Gebrauchstauglichkeit.

			\textbf{\# Welche Daten bei Ermittlung des Kontextes
				berücksichtigen?}

			Ob User bei hellem Licht oder in der Dunktelheit das System nutzen,
			Ob User während der Benutzung im Stress ist. Die technische
			Umgebung des Systems. Also alle technischen Hilfs- und
			Arbeitsmittel des Benutzers

			\textbf{\# Unterschied zw. Arbeitsmittel und Kontext}
			Kontext umfasst technische, physische,
			gesellschaftliche Umgebung; Arbeitsmittel
			sind alle Werkzeuge

			\textbf{\# Einführung neuer beruflicher Software verändert
				Arbeitsprozesse}
			Richtig, da Einf.\ einer brfl genutzt Software m.Zielen
			verbunden ist, die normalerws. Auswirkungen auf die
			Arbeitsproz.hat (Einsparungen, Qualiverbesserung, Änderung d.\
			Kommunikationsstrukturen, Controlling o.ä.)

			\textcolor{LimeGreen}{\textbf{\scriptsize{\# Menschengerechte
				Arbeitsgestaltung}}}

			\textbf{\# Begriffe Grundmodell Hacker}
			1.Ausführbar
			2.Schädigungslos
			3.Beeinträchtigungsfrei
			4.Persönlichkeitsfördernd
			6 Humankriterien
			1.Benutzerorientierung
			2.Anforderungsvielfalt
			3.Ganzheitlichkeit und Bedeutsamkeit
			4.Handlungsspielräume
			5.Rückmeldungen
			6.Entwicklungsmöglichkeiten

			\textbf{\# Ganzheitlichkeit und Bedeutsamkeit}
			Bedeutung von Aufgabe kann erkannt und in Gesamtablauf eingeordnet
			werden. Ergebnis / Arbeitsfortschritt kann selbst beurteilt werden
			=> steigert Zufriedenh. Mitarbeiter sollte Sinnhaftigkeit und
			Bedeutsamkeit einer Aufg.\ verstehen.

			\textbf{\# Unterschied Belastung und Beanspruchung?}
			\textit{Belastung:} Äußere Bedingungen und Anforderungen;
			\textit{Beanspruchung:} Reaktion des Individuums auf Belastung
			Überforderung kann zu innerer Anspannung, Angst, Erschöpfung
			führen; Unterforderung zu Langeweile, Unlust und dazu, dass Arbeit
			schlecht gemacht

			\textbf{\# Ziel menschengerechter Arbeitsgestaltung?}
			Gestaltung der ergonomischen Maßnahmen (Arbeitspl.\
			u.Arbeitsmittel) Gestaltung der organisatorischen und sozialen
			Maßnahmen (Arbeitszeiten, Hierarch, Informatnsfls in Firma

			\textcolor{LimeGreen}{\textbf{\scriptsize{\# Grundsätze der
				Dialoggestaltung}}}

			\textbf{\# Definition Dialog}
			Interaktion zw. Benutzer und interaktiven System
			in Form einer Folge von Handlungen des Benutzers (Eingaben) und
			Antworten des interaktiven Systems (Ausgaben), um ein Ziel zu
			erreichen.

			\textbf{\# Definition Benutzungsschnittstelle}

			„Alle Bestandteile eines interaktiven Systems (Software oder
			Hardware), die Informationen und Steuerelemente zur Verfügung
			stellen, die für den Benutzer notwendig sind, um eine bestimmte
			Arbeitsaufgabe mit dem interaktiven System zu erledigen.“

			\textbf{\# Was ist der Nutzen von Gestaltungsgrundsätzen für
				Dialoge?}

			- gut gestalt. Dialoge haben großen Anteil daran, wie gut
			  Gebrauchsttglkt.
			- Die Gestaltungsgrundsätze geben konkrete Kriterien wie Dialoge
			  gebrauchstauglich gestaltet werden.

			\textbf{\# Die 7 Grundsätze der Dialoggestaltung nach
				ISO 9241-110}

			1. Aufgabenangemessenheit (Möglichst einfach gestaltet)
			2. Selbstbeschreibungsfahigkeit (Wo befinde ich mich / was muss ich
			   tun)
			3. Erwartungskonformität (Dialog ist intuitiv nutzbar - beispiel:
			   Überweisungsträger)
			4. Lernförderlichkeit (Tutorials + direkt erkennbar was getan
			   werden soll)
			5. Steuerbarkeit (Nutzer kann Aktionen selber abbrechen/starten)
			6. Fehlertoleranz (Kann mit minimalen Aufwand die meisten Fehler
			   beheben)
			7. Individualisierbarkeit (Nutzer kann die System ändern)

			\textbf{\# 5 Erweit.\ der Grundsätze der Dialoggestaltung
				nach Sarodn.\ und Brau?}
			1. Prozessangemessenheit
			2. System- und Datensicherheit
			3. Wahrnehmungssteuerung
			4. Interkulturelle Aspekte
			5. Joy of use

			\textbf{\# Shneidermans 8 goldene Regeln des
				Dialogdesigns?}

			1. Streben nach Konsistenz
			2. Universelle Gebrauchstauglichkeit
			3. Biete informative Rückmeldungen
			4. In sich geschlossene Dialoge (Dialoge sollten eine klar
			   erkennbare Struktur haben)
			5. Verhindere Fehler
			6. Erlaube einfache Rücksetzungen (Undo)
			7. Unterstütze das eigene Kontrollbedürfnis (Nutzer steuert
			   Software, nicht anders herum)
			8. Reduziere die Belastung des Arbeitsgedächtnisses

			\textbf{\# Welche Shneiderman/ISO Grunds.werden verl wenn
				Softwareanw.autonom korrigiert}

			- Lernförderlichkeit – DIN ISO 9241-110
			- Steuerbarkeit – DIN ISO 9241-110
			- Fehlertoleranz – DIN ISO 9241-110
			- Biete informative Rückmeldungen – Shneiderman
			- Erlaube einfache Rücksetzung – Shneiderman
			- Unterstütze das eigene Kontrollbedürfnis – Shneiderman

			\textbf{\# 3 Arten der Konsistenz / Begriffe
				Erwartungskonformität?}
			- Äußere (OS)
			- Metaphorische (zu real Welt.)
			- Innere (in app)

			\textbf{\# Vorteile von Styleguides in Bez.\ auf die
				Dialoggrundsätze?}

			Einheitliches Design => ``Erwartungskonformität''

			\textbf{\# ``Fehlererkennung'' Beispiel}

			- System validiert die Eingaben (z.B. Überprüfung eines
			  PLZ-Feldes auf korrekte Syntax und Länge)

			\textbf{\# ``Fehlervermeidung'' Beispiel}

			- Kennzeichnung von Pflichtfeldern oder
			- Auswahloptionen verwenden statt Freitexteingaben

			\textbf{\# ``Fehlerkorrektur'' Beispiel}

			- System gibt Korrekturvorschläge (z.B. Rechtschreibung)

			\textcolor{LimeGreen}{\textbf{\scriptsize{\# Interaktionsformen}}}

			\textbf{\# Welches Gesetz beschr. Zeit zur Positionierung in Abh.\
				von Entf.\ und Zielgröße?}
			Fitts’ Law

			\textbf{\# Womit die Interaktionszeiten für den Wechsel von Maus
				und Tastatur abschätzen?}
			GOMS

			\textbf{\# Gesetz das sagt, dass Zeiten f.Erkennung/Auswahl
				abhängig von Anz. Menüpkte?}
			Hicks’ Law

			\textbf{\# Welche Interaktionsformen für Touch-Systeme sehr gut?}

			- Interakt.\ via Icons oder via Kontextmenü
			- Modale Dialoge

			\textbf{\# Worum geht es im GOMS Modell?}

			- durschnittliche Zeiten für typische Interaktionen
			- Anhand dieser Zeiten kann komplette Interaktion Ausgewertet

			\textbf{\# Nennen Sie mögl Interaktionsformen mit Tastatur}

			- Kommandosys.
			- FN Keys
			- Shrtcts
			- Cursorblock \& Pfeiltstn

			\textbf{\# Was sind Vor- und Nachteile von Kommandosystemen}

			Vorteile:
			- Nur Tastatur
			- History
			- Experten
			- Scriptable
			- Automatisierbar
			Nachteile:
			- Hoher Lernaufwand
			- System führt die Befehle ohne vis Rückmeldung aus ``Trial and
			  Error''
			- Tippfehler sind leicht möglich.

			\textbf{\# Erläutern Sie die Vor und Nachteile von Funktionstasten}

			Vorteile:
			- Kann Interaktionszeiten stark reduzieren
			Nachteile:
			- Eventuell ungewohnt wenn die Belegung nicht individualisiert
			  werden kann

			\textbf{\# Nennen Sie Vor- und Nachteile von Shortcuts}

			Vorteile:
			- Geschwindigkeit
			- Gut für Blinde und Sehbehinderte
			Nachteile:
			- Verwirrend für Anfänger
			- ggf.\ unergonomisch bei schlechter Wahl der Kombi

			\textbf{\# Vor- und Nachteile von Cursorblock + Pfeiltasten}

			Vorteile:
			- Schnelle Navigation durch Oberfächen jeglicher Art
			- Deutlich effizienter als wechsel zur Maus
			Nachteile:
			- Software muss Tasten unterstützen

			\textbf{\# Mögliche Interaktionsarten mit Zeigegeräten}

			- Menüs
			- Ribbons
			- Kontextmenüs
			- Mausgesten
			- Dialogfenstern mit Erweiterung

			\textbf{\# Wofür steht WIMP?}

			- Windows
			- Icons
			- Menus
			- Pointing Device

			\textbf{\# Was sind Vor- und Nachteile von WIMP Interfaces?}

			Vorteile:
			- sehr hohe Lernförderlichkeit
			- Objekte und Aktionen sind direkt erkennbar
			- direkte Reaktion auf Aktionen des Nutzers
			- Handlungen sind nah zur realen Welt
			  (statt Kommandos - Beispiel: Datei verschieben)
			Nachteile:
			- Langsam und ineffizient im Vergleich zu anderen
			  Interaktionsformen
			- ggf.\ hohe Frequenz der Wechsel zwischen Tastatur und Maus
			- bei komplexer Software müssen Menüs und Icons müssen gesucht
			  werden

			\textbf{\# Wann bieten sich Touch-Anwendungen an?}

			kurze Interaktionsdauer

			\textcolor{LimeGreen}{\textbf{\scriptsize{\# Interaktionsdesign}}}

			\textbf{\# Nenne charakteristische Eigensch.\ von dargestellten
				Infos (nach ISO 9241-12)}

			- Klarheit
			- Unterscheidbarkeit
			- Kompaktheit
			- Konsistenz
			- Erkennbarkeit
			- Lesbarkeit
			- Verständlichkeit

			\textbf{\# Welche Interaktionselemente gibt es?}

			1. Radiobuttons und Checkboxen
			2. Dropdown- und Auswahllisten
			3. Eingabefelder
			4. Schaltflächen

			\textbf{\# Was sind Masken? Was sind Strukturblöcke?}

			- Eingabemaske (=Maske) stammt aus Zeit in der
			  Computer grüne Bildschirme nur 25 Zeilen je 80 Chars anzeigen
			  konnten
			- Strukturblock = Gruppierung von Controls -> abgegrenzte
			  Informations- oder Interaktionsmöglichkeit 

			\textbf{\# Fluchtlinien und wie sollte verwenden?}

			- FL = die Linien zu Kanten und Seiten von Interaktionsel
			- sollten auf ein Mindestmaß reduziert werden

			\textbf{\# Was bei Abständen zwisch Interaktionsel beachten?}

			- Gesetz der Nähe für vertikale und horiz Abstände
			- Abstände zu beschreibenden Texten berücksichtigen

			\textbf{\# Was ist bei der Gruppierung von Informationen zu
				beachten?}

			- Fachlich zusammenhängende Interaktionselemente sollten
			  entsprechend der erwarteten Reihenfolge gruppiert werden

			\textbf{\# Wie sind Styleguides aufgebaut?}

			1. Einleitung: Version, Datum und Autor, Zweck + Einsatzgeb,
			   Zielpublikum
			2. Konzept: Produktvision, UX-Ziele, Accesblty,
			   Designprnzpien
			3. Interaktnsmuster
			4. Struktur: Navigtnkonzept, Seitenlayout, Raster
			5. Visuelle Gestaltung: Col, Font, Grafiken, Interaktionsel in
			   exaktem Design
			6. Kommunikationsstil: Stil Texte, Wortwahl, Stil Bilder,
			   Einsatz von Audio, Video
			7. Weiteres: Beispiele, Werkzeuge

			\textbf{\# 2 Maßnahmen um ein System gut bedienbar für
				Farbfehlsichtige zu machen}

			- Nicht nur Farben zur Kodierung von Hinweisen/Information 
			  -> Symbole/Schraffuren/Kontrast
			- UIs mit Usability-Tests mit Farbfehlsichtigen testen

			\textbf{\# Nenne 4 Usability-Aspekte die für Senioren zu
				berücksichtigen sind + begründen}

			- Große Schrift und Schriftgröße auf einfache Weise noch stärker
			  vergrößerbar, da die Sehfähigkeit oft eingeschränkt ist
			- Große Interaktionsflächen (Buttons) und große Abstände zwischen
			  den Interaktionsflächen, da die motorischen Fähigkeiten oft
			  eingeschränkt sind
			- Einfache Wortwahl, einfache Sätze
			- einfache Navigation (wenige Menüpunkte), da kognitive Leistungen
			  eingeschr.\ sein können

			\textcolor{LimeGreen}{\textbf{\scriptsize{\# Usability
				Engineering}}}

			\textbf{\# Bedeutung Usability Engineering?}

			- beschreibt diesen Prozess, wie parallel zum Software Engineering
			  auch die Gebrauchstauglichkeit berücksichtigt werden kann
			- ``Engineering'' = strukturierte, methodische, prozess- und
			  phasenorientierte Vorgehensweise

			\textbf{\# 7 Schrt. Phasenmod nach Sarodnick und Brau}

			1. Analyse der Arbeit und des Arbeitsumfeldes
			2. Analyse der Benutzergruppen
	 		3. Bestimmung von Anforderungen
			4. Entscheidung über Funktionalität und Ableitung eines Handlungs-
			   und Bedienkonzeptes
			5. Entwicklungsbegleitende Evaluation und Verbesserung des Systems
			6. Einführung und Schulung
			7. Weiterentwicklung

			\textbf{\# 4 Phsn Usb. Engin\. nach Sarodnick u. Brau}

			1. Analysephase (Protokoll, Personas und gfx drstlng der Prozesse)
			2. Konzeptphase (visuelles Rohgerüst als Papier-Prototyp(en))
			3. Entwicklungsphase (fertiges System)
			4. Einführungsphase (Usability-Tests, Fragebögen,
			   Interviewergebnisse)

			\textbf{\# Arbeitsschritte der Analysephase}

			- Arbeitsanalyse (Arbeitsanforderungen, -bedingungen, -prozesse)
			- Prozessanalyse
			- Systemanalyse
			- Erhebnung von Nutzeranforderungen

			\textbf{\# Arbeitsschritte der Konzeptphase}

			- Arbeitsgestaltung und Prozessdefinition
			- Entscheidung über Systemfunktionalitäten
			- Konzepterstellung

			\textbf{\# Wo ist Phasenmodell v.Sarodnick/Brau
				m.Vorgehensm.d.Softwaretechnik verknüpft?}

			Beim Unterpunkt „Systemintegration“ verlassen wir im
			Projektmanagement das Phasenmodell des Usability Engineering und
			wechseln in das Vorgehensmodell der Softwaretechnik. Nach den
			Regeln der Softwaretechnik wird das System erstellt und nachdem das
			lauffähige System vorliegt, erfolgt wieder der Wechsel in das
			Phasenmodell des Usability Engineering

			\textbf{\# Mögliche Reaktionen von User auf neues System + Gründe?}

			- gerade Nutzer im betrieblichen Umfeld reagieren auf Neuerungen
			  oftmals nicht positiv
			- viele Nutzer reagieren skeptisch und mit geringer Akzeptanz

			Gründe:
			Unterbrechung der Routine, Neulernen, Entwertung der Qualifikation,
			Statusverlust, Doppelbelastung während der Einführung, Verlust von
			Freiräumen, Angst vor Arbeitsplatzverlust

			\textbf{\# Was kann Akzeptanz der User ggü.\ neu
			eingeführten/weiterentw. System erhöhen?}

			- gute Nutzerbeteiligung (optimal in allen Phasen)

			\textcolor{LimeGreen}{\textbf{\scriptsize{\# Heuristiken}}}

			\textbf{\# Nenne die Kriterien der heuristischen Evaluation (nach
				Nielsen)}

			1. Sichtbarkeit des Systemzustands
			2. Übereinstimmung von Systemzustand und Realwelt
			3. Freiheit der Benutzersteuerung, ``Notausgang''
			4. Konsistenz und Einhaltung von Konventionen und Standards
			5. Fehlerverhinderung
			6. Wiedererkennen vor Erinnerung
			7. Flexibilitt und Nutzungseffektivität (Abkürzungen für geübte
			   Benutzer)
			8. Aesthetisches und minimalistisches Design
			9. Unterstuetzung beim Erkennen, Deuten und Beheben von Fehlern
			10. Hilfe und Dokumentation

			\textbf{\# Definieren sie Heuristik}

			- die Kunst, mit begrenztem Wissen und wenig Zeit zu guten Lösungen
			  zu kommen

			\textbf{\# Definition heuristische Evaluation?}

			- Verfahren zur Problemlösung durch analytische Betrachtung und
			  systematisches Probieren

			\textbf{\# Grober Ablauf einer heuristischen Evaluation?}

			- Usability-Experten bewerten die Software mit anerkannten
			  Gestaltungsgrundsätze und geben konkrete Verbesserungsvorschl.
			- Dialoge werden einzeln betrachtet und jeder Dialog für sich wird
			  gegen die Heuristiken geprüft - Hierbei werden keine Personas
			  oder Use-Cases eingesetzt

			\textbf{\# Was sind typische Heuristiken?}

			- Shneidermans 8 goldene Regeln
			- Usability-Prinzipien von Nielsen
			- Heuristiken nach Sarodnick und Brau

			\textbf{\# Wie haengen die 8 goldenen Regeln mit der heuristischen
				Evaluation zusammen?}

			- die 8 goldenen Regeln sind eine Heuristik Allgemein anerkannte
			  Kriterien, mit denen Dialoge geprüft und verbessert und evaluiert
			  werden können.
			- Wiederholte anlegen gleich. Kriterien (8 Reglen) können für
			  Statistiken genutzt werden => Fließen wieder in Evaluation ein.

			\textbf{\# Erläutern Sie die Begriffe "Formative Evaluation" und
				``Summative Evaluation''}

			Formative Evaluation:
			- Evaluationsmethoden, die während der Entwicklung eines Systems
			  verwendet werden
			- Ergebnisse fließen direkt in den weiteren Entwicklungsprozess ein
			  -> Wie kann bei der Entwicklung das optimale Ergebnis erzielt
			  werden?

			Summutative Evaluation:
			- Evaluationsmethoden, die nach der Entwicklung eines Systems
			  verwendet werden und das System bewerten
			- Summativ = abschließend, zusammenfassend -> Wie gut
			  Gebrauchstauglichkeit? + Umfang erreichter Ziele

			\textbf{\# Was ist der Vorteil von formativer Evaluation?}

			- beantwortet Frage: Wie kann bei der Entwicklung das optimale
			  Ergebnis erzielt werden? -> es wird effizienter entwickelt
			- bezüglich Zeiten und Kosten ist es viel besser während des
			  Entwicklungsprozesses zu evaluieren, als am Ende die fertige
			  Entwicklung zu betrachten

			\textbf{\# Ordnen sie dem Prinzip ``Benutzerbasierte Evaluation''
				Methoden zu}

			- Befragung (Fragebogen, Interview) - Usability-Test - Eye-Tracking
			- Beobachtung - Thinking Aloud

			\textbf{\# Ordnen Sie dem Prinzip Menschzentrierte Gestaltung
				Methoden zu}

			- Informationsanalyse (Datenanalyse, Dokumentensichtung)
			- Befragung (Fragebogen, Interview)
			- Personas
			- Usability-Test
			- Storyboards
			- Zukunftswerkstätten
			- Fokusgruppen

			\textbf{\# Ordnen Sie dem Prinzip ``Theoriebasierte Evaluation''
				Methoden zu}

			- Informationsanalyse (Datenanalyse, Dokumentensichtung)
			- Inspektionsmethoden (Cognitive Walkthrough, Heuristische
			  Evaluation)
			- Personas
			- Styleguide

			\textbf{\# Welche Maßnahmen gehören zu Cognitive Walkthrough?}

			- Testziele festlegen
			- Typische Szenarien bestimmen
			- Zielgruppe definieren (z.B. Personas/Storyboards erstellen)
			- Testdurchführung
			- Ergebnisse bewerten

			\textbf{\# Welche Maßnahmen gehören zu einem Usability Test?}

			- Testziele festlegen
			- Testablauf festlegen (Testkonzept)
			- Auswerteverfahren festlegen
			- Probanden auswählen
			- Probedurchlauf vornehmen
			- Testdurchführung

			\textbf{\# Nennen Sie jeweils 3 Vor- und Nachteile von
				Beobachtungen}

			Vorteile:
			- sehr gut geeignet um das Arbeitsgebiet des Benutzers kennen zu
			  lernen
			- im Rahmen einer Arbeits- und Aufgabenanalyse entsteht ein gutes
			  Bild von der realen Benutzung einer Software
			- besonders für Routineaufgaben geeignet (für die Optimierung
			  v.Arbeitsabläufen)

			Nachteile:
			- reine Beobachtung ist für Evaluation eher ungeeignet, besser
			  Kombination aus Interview und Beobachtung 
			- verdeckte Beobachtungen sind für eine Evaluation eher ungeeignet
			- Mitarbeiter verhalten sich bei angekündigten Beobachtungen
			  zurückhaltender

			\textbf{\# Was versteht man unter menschzentrierter Gestaltung?}

			- die Benutzer werden in den Entwicklungsprozess einbezogen

			\textbf{\# Wie User bei menschzentr.Gestalt.in Prozess der
				Sys.entw.\ einbeziehen?}

			Wie wird beteiligt?:
			- aktive Mitentscheidung (über Funktionsumfänge etc.)
			- Aktive Partizipation: Die Benutzer werden in den frühen
			  Gestaltungsphasen direkt gestaltend tätig (Arbeitsprozesse
			  abbilden, Designentwürfe vorbringen)

			Wann wird beteiligt?:
			- während des gesamten Entwicklungsprozesses oder
			- während eines Teils des Entwicklungsprozesses oder
			- zu ausgewählten Zeitpunkten in Workshops

			Woran wird beteiligt?
			- Beteiligung kann sich auf alle Einzelbereiche der Software
			  (Prozesse, Funktionalitäten, Schnittstellen etc.) beziehen

			\textbf{\# Was haben „Cognitive Walkthrough“ und „Heuristische
				Evaluation“ gemeinsam?}

			- beides Inspektionsmethoden (= Expertenbasierte Methoden = keine
			  Einbeziehung der tatsaechlichen Benutzer) und/oder
			- beide sind relativ schnell durchzufuehren und kostenguenstig

			\textbf{\# App f.Projektzeiten auf Gestaltunssgr\. untersuchen,
				worauf muss man achten?}

			- Aufgabenangemessenheit: kann man Projektzeiten mit wenigen Klicks
			  erfassen
			- Selbstbeschreibungsfähigkeit: Hat jeder App-Screen eine
			  klare Überschrift/ Seitenbenennung
			- Erwartungskonformität: Ist die
			  innere Konsistenz der Buttons bzgl. Anordnung und Aussehen geben
			- Lernförderlichkeit: Ist ein Link zu einem Tutorial/Video
			  vorgesehen
			- Individualisierbarkeit: Können verschiedene Projekte entsprechend
			  der Reihen- folge in der App verschoben werden
			- Steuerbarkeit: Sind alle Interaktionselemente leicht zugänglich
			  angebracht
			- Fehlertoleranz: sind möglichst wenige Input-Felder vorhanden und
			  wurden wo immer möglich stattdessen Auswahlmenüs verwendet

			\textbf{Beispiel-Heuristiken:}

			\textbf{Aufgabenangemessenheit:} kann man Projektzeiten mit
				wenigen Klicks erfassen
			\textbf{Selbstbeschreibungsfähigkeit} Hat jeder App-Screen eine
				klare Überschrift/Seitenbenennung
			\textbf{Erwartungskonformität} Ist die innere Konsistenz der
				Buttons bzgl. Anordnung und Aussehen geben
			\textbf{Lernförderlichkeit} Ist ein Link zu einem Tutorial/Video
				vorgesehen
			\textbf{Individualisierbarkeit} Können verschiedene Projekte
				entsprechend der Reihenfolge in der App verschoben werden
			\textbf{Steuerbarkeit} Nutzer kann aktionen abbrechen / undo /
				multistep formular (zwischen schritten springen)
			\textbf{Fehlertoleranz} Formular geht nicht kaputt bei falscher
				eingabe / Falsche eingaben werden verhindert

			\textbf{6 Usability-Maßnahmen:}

			\textbf{1. Beobachtung, Interviews} (mit Zielgruppe)
			\textbf{Ergebnis:} Nutzungskontextes, derzeitige Nutzung,
				Stärken / Schwächen.
			\textbf{2. Usability-Tests} (Eye-Tracking / lautes Denken zu
				wichtigen Aufgaben)
			\textbf{Ergebnis:} Usability-Probleme
			\textbf{3. Kreativtechniken} (Fokusgruppen / Card-Sorting zur
				Ideenfindung / prüfen von Konzeptideen)
			\textbf{Ergebnis:} Neue Ideen / Absichern
			\textbf{4. Papier-Prototypen} (möglichst viele verschiedene)
			\textbf{Ergebnis:} Vergleich Ideen / Konzepte -> Optimales
				Konzept
			\textbf{5. Funktionalen Prototypen + Cognitive Walkthrough}
			\textbf{Ergebnis:} Interaktionspfade + weitere Optimierung
			\textbf{6. Test \& Beobachtung} beim Umgang mit funkt.
				Prototypen
			\textbf{Ergebnis:} Weitere Optimierungen

			\textcolor{LimeGreen}{\textbf{\scriptsize{\# 6-Ebenen-Modell}}}

			Es untersucht das Planen und Ausführen einer Aktion auf
			verschiedenen Ebenen, von der globalen Betrachtung (Intention =
			Tätigkeiten und Bewertung) bis hin zu der Motorik. Damit
			ermöglicht das Modell sehr gut die Analyse der GUI von der
			planenden Idee über die pragmatische, semantische, syntaktische
			und lexikalische Ebene bis zur motorischen Handlung. Es kann
			damit sehr gut erkannt werden, ob eine GUI auf allen Denk- und
			Handlungsebenen optimiert wurde Bsp: Zur Überprüfung der
			lexikalischen Ebene muss in einem Druckdialog jedes Eingabefeld
			auf alle möglichen Zeichen hin untersucht werden

			\textcolor{LimeGreen}{\textbf{\scriptsize{\# Beispiele}}}

			\textbf{\# Usability-Maßn.\ in Reihenfolge, um Website d.\
				Hochschule bez.Usability zu verbessern}

			1. Beobachtung u.Interviews m.d.Zielgruppe (Interessenten,
			Studierende, Angestellte) => \textit{Verstehen d.Nutzungskontextes
			u.d.derzeitigen Nutz- ung sowie Ermittlung v.Stärken u.Schwächen.}

			2. Durchführung v.Usability-Tests (auch mittels Eye-
			Tracking od.„lautem Denken“) zu wichtigen Aufg.\
			wie z.B. d.Immatrikulation,dem Auffinden v.Adress-
			daten, Ansprechpartner in d.Fachbereichen,Öffnungs-
			zeiten d.Bibliothek etc.). => \textit{Erm.d.Usability-Probleme.}

			3.(Kreativtechniken) Durchf.d.Methode „Fokusgr.“
			od.„Zukunftswerkstatt“od.„Card-Sorting“ zur Ideen-
			findung bzw.zur Überpr.v.Konzeptideen => \textit{Absicherung
			Bish.Ideen u.Findung neuer Ideen}

			4. Erstellen v.(möglichst vielen alternativen) Papier-Prototypen =>
			\textit{Vergl.\ untersch.Konz.z.Erm.eines optimalen Informations-
			und Bedienkonzepts.}

			5.Erstellung eines funktionalen Prototypen u.Durchf.eines Cognitive
			Walkthrough => \textit{Erm.d.Interaktionspfade aus Sicht
			d.Zielgruppe z.\ weiteren Optimierung d.Prototypen}

			6. Test + Beob.\ v.Probanden beim Umgang m.d.funktionalen Protot.
			(ggf.Nutzung v.Eye-Tracking) => \textit{Optimierung d.Protot.}

			Mara Schulke, 20215853
		\end{tiny}
	\end{flushleft}
\end{multicols}

\end{document}
