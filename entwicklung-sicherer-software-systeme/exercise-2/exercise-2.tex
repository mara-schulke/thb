\documentclass{article}

\usepackage[a4paper]{geometry}
\usepackage[ngerman]{babel}
\usepackage[utf8]{inputenc}
\usepackage[T1]{fontenc}
\usepackage{graphicx}
\usepackage{fancyhdr}
\usepackage{xcolor}
\usepackage{float}
\usepackage{tabularx}
\usepackage{pdflscape}

\graphicspath{{./images/}}

\pagestyle{fancyplain}
\fancyhf{}
\lhead{\fancyplain{}{Mara Schulke} }
\rhead{\fancyplain{}{\today}}
\cfoot{\fancyplain{}{\thepage}}

\begin{document}

\begin{titlepage}
	\begin{flushleft}
		TH Brandenburg \\
		Studiengang IT Sicherheit \\
		Fachbereich Informatik und Medien \\
		Entwicklung Sicherer Softwaresysteme \\
		Prof. Dr.-Ing. Martin Schafföner
	\end{flushleft}

	\vfill

	\begin{center}
		\Large{Einsendeaufgabe: Nichtfunktionale Sicherheitsanforderungen}\\[0.5em]
		\large{Sommersemester 2024}\\[0.25em]
		\large{Abgabetermin \today}
	\end{center}

	\vfill

	\begin{flushright}
		Mara Schulke \\
		Matrikel-Nr. 20215853
	\end{flushright}
\end{titlepage}

\tableofcontents

\listoffigures

\section{Einführung}

In der vorherigen Einsendeaufgabe wurde die Online-Meeting-Plattform Google Meet 
analysiert und ausgewertet. Die Sicherheitsbegutachtung von Google Meet wird im Rahmen 
dieser Aufgabe fortgeführt und um Sicherheits-Akzeptanzkriterien bzw. 
sicherheitsspezifische Anforderungen erweitert. Außerdem werden die verschiedenen 
Vorgehensweisen verglichen und es werden Testszenarien definiert, mittels derer die 
Sicherheits-Akzeptanzkriterien verifiziert werden können.

\section{Funktionale Anforderungen}

Da die funktionalen Anforderungen im Rahmen der letzten Einsendeaufgabe noch nicht konkret 
benannt worden sind, ist der erste Schritt um sinnvolle Sicherheitsanforderungen zu 
definieren die Benennung der funktionalen Anforderungen an Google Meet. Im konkreten 
(angenommen, es wird nur der grundlegende Umfang einer Meeting-Plattform abgedeckt) 
handelt es sich dabei um die folgenden Anforderungen:

\begin{itemize}
	\item \textbf{Benutzerauthentifizierung:} Google Meet muss es Benutzern ermöglichen, sich zu registrieren, anzumelden und abzumelden.
	\item \textbf{Meeting-Planung:} Google Meet muss es den Benutzern ermöglichen, Besprechungen mit bestimmten Teilnehmern zu planen.
	\item \textbf{Video- und Audio-Streaming}: Google Meet muss Video- und Audiostreaming in Echtzeit unterstützen.
	\item \textbf{Chat-Funktionalität:} Google Meet muss Textchats in Echtzeit während der Sitzungen unterstützen.
	\item \textbf{File-Sharing:} Google Meet muss es den Nutzern ermöglichen, während einer Besprechung Dateien gemeinsam zu nutzen bzw. zu teilen.
	\item \textbf{Bildschirmfreigabe:} Google Meet muss es den Nutzern ermöglichen, ihren Bildschirm während einer Besprechung gemeinsam zu nutzen.
\end{itemize}

\section{Nichtfunktionale Sicherheitsanforderungen}

Die nichtfunktionalen Sicherheitsanforderungen an Google Meet bauen auf den funktionalen 
Anforderungen auf und decken die zugrundeliegenden Sicherheitsmechanismen ab. Sie benennen 
die konkreten Aspekte die Software-Entwickler bei der Umsetzung der obigen Anforderungen 
in einem System berücksichtigen müssen. Bei der Formulierung von nichtfunktionalen 
Sicherheitsanforderungen stehen einige Möglichkeiten zur Verfügung:

\subsection{Verschiedene Vorgehensweisen}

\subsubsection{Evil User Stories}

Evil User Stories sind ein Mittel, um potenzielle Sicherheitsbedrohungen zu 
identifizieren, indem man aus der Perspektive eines böswilligen Nutzers (bzw. Angreifers) 
denkt. Diese Methode hilft Software-Entwicklern, Schwachstellen im System zu erkennen und 
entsprechende Gegenmaßnahmen zu planen.
Diese folgen in der Regel der Struktur:

\begin{quote}
``Als [böswilliger Akteur] möchte ich [schädliche Aktion], um [negativen Effekt] zu erreichen.''	
\end{quote}

\begin{flushleft}
Ein Beispiel für diese Vorgehensweise könnte sein:
\end{flushleft}

\begin{quote}
``Als Angreifer möchte ich unverschlüsselte Daten abfangen, um vertrauliche Informationen zu stehlen.''
\end{quote}

\subsubsection{Sicherheits-Akzeptanzkriterien}

Sicherheits-Akzeptanzkriterien sind spezifische, messbare Bedingungen, die erfüllt sein 
müssen, damit ein Sicherheitsfeature als erfolgreich implementiert gilt. Diese Kriterien 
helfen dabei, sicherzustellen, dass die Sicherheitsanforderungen korrekt und vollständig 
umgesetzt wurden.

\begin{flushleft}
Ein Beispiel für diese Vorgehensweise könnte sein:
\end{flushleft}

\begin{quote}
``Alle Datenübertragungen müssen mit mindestens mit AES-256 verschlüsselt sein.''
\end{quote}

\subsubsection{Sicherheitsspezifische Anforderungen}

Sicherheitsspezifische Anforderungen sind detaillierte, konkrete Anforderungen, die 
Sicherheitsmechanismen und -verfahren beschreiben, um das System vor Bedrohungen zu 
schützen. Diese Anforderungen sind spezifisch und oft technischer Natur.

\begin{flushleft}
Ein Beispiel für diese Vorgehensweise könnte sein:
\end{flushleft}

\begin{quote}
``Implementiere HMAC für alle Nachrichten, um ihre Integrität sicherzustellen''
\end{quote}

\newpage

\begin{landscape}
	
\subsection{Auflistung der nichtfunktionale Sicherheitsanforderungen}

\begin{tabularx}{20cm}{|l|X|X|}
    \hline
    & \vspace{0em} \textbf{Sicherheits-Akzeptanzkriterien} \vspace{0.5em}
    & \vspace{0em}\textbf{Sicherheitsspez. Anforderungen} \vspace{0.5em} \\
    \hline
        \textbf{Authentifizierung}
        & Die Verfahren zur Benutzerauthentifizierung müssen einen Penetrationstest 
          bestehen, um die Robustheit der Multi-Faktor-Authentifizierung zu überprüfen.
        & Alle Authentifizierungen müssen durch eine Multi-Faktor-Authentifizierung 
    	  geschützt sein.\vspace{0.5em}
    	
    	  Passwörter müssen mit einer dem Industriestandard entsprechenden Verschlüsselung 
    	  (z. B. bcrypt) gespeichert werden.
        \\
    \hline
        \textbf{Datenübertragung}
        & Alle Datenübertragungen müssen mit einem Netzwerkprotokoll-Analysator validiert 
          werden, um sicherzustellen, dass die Verschlüsselungsstandards eingehalten 
          werden.
        & Alle zwischen den Clients und dem Server übertragenen Daten
          müssen mit TLS 1.2 oder höher verschlüsselt werden.
        \\
    \hline
        \textbf{Zugangskontrolle}
        & Die Zugriffskontrollmechanismen müssen anhand rollenbasierter Zugriffsszenarien 
          getestet werden, um eine ordnungsgemäße Durchsetzung sicherzustellen.
          \vspace{0.5em}

          Unerlaubte Zugriffsversuche müssen protokolliert und während eines 
          Audits überprüft werden.
        & Die Nutzer dürfen nur auf Sitzungen zugreifen, zu denen sie ausdrücklich 
          eingeladen wurden.\vspace{0.5em}
          
		  Es muss eine rollenbasierte Zugriffskontrolle (RBAC) implementiert werden, um 
		  den Zugriff auf Verwaltungsfunktionen zu beschränken.
        \\
    \hline
        \textbf{Datenschutz}
        & Die Verschlüsselung von Besprechungsaufzeichnungen muss durch unbefugte 
          Zugriffsversuche überprüft werden.
        & Sitzungsaufzeichnungen müssen verschlüsselt und sicher gespeichert werden und 
          nur autorisierten Benutzern zugänglich sein.
        \\
    \hline
        \textbf{Protokollierung}
        & Die Protokolle müssen monatlich überprüft werden, um sicherzustellen, dass sie 
          detailliert und gegen Manipulationen geschützt sind.\vspace{0.5em}

		  Automatisierte Tools müssen die Protokolle auf verdächtige Aktivitäten 
		  überwachen, wobei Warnungen innerhalb von 24 Stunden überprüft werden müssen.
        & Jeder Zugriff auf sensible Informationen und jede Aktion muss protokolliert und 
          auf verdächtige Aktivitäten überwacht werden.\vspace{0.5em}

		  Die Protokolle sind vor Manipulationen zu schützen und mindestens ein Jahr lang 
		  sicher aufzubewahren.
        \\
    \hline
        \textbf{Notfallmanagement}
        & Vierteljährlich muss eine Übung zur Reaktion auf Zwischenfälle durchgeführt 
          werden, über die ein Bericht erstellt und auf ihre Wirksamkeit überprüft wird.
        & Google Meet muss über einen Plan zur Reaktion auf Zwischenfälle verfügen, der 
          Verfahren zur Erkennung, Meldung und Behebung von Sicherheitsvorfällen enthält.
        \\
    \hline
\end{tabularx}

\end{landscape}

\newpage

\section{Vergleich und Präferenz}

\textit{Sicherheitsspezifische Anforderungen} bieten Software-Entwicklern klare und 
umsetzbare Schritte zur Implementierung von einzelnen Sicherheitsmaßnahmen. Sie sind 
spezifisch und können leicht in die Entwicklung integriert werden, allerdings kann es 
ihnen an Kontext fehlen und sie können manchmal zu präskriptiv sein, was weniger Raum für 
Flexibilität lässt.

\textit{Sicherheits-Akzeptanzkriterien} bieten den Vorteil, dass sie sich auf die 
Ergebnisse und überprüfbaren Bedingungen konzentrieren. Sie gewährleisten, dass die 
implementierten Maßnahmen wirksam sind, anstatt eine konkrete Implementierung 
vorzuschreiben. Außerdem bieten sie eine Möglichkeit, bzw. ein klares Szenario, diese 
Maßnahmen zu testen.

\vspace{1em}

Ich persönlich bevorzuge den Einsatz von Sicherheits-Akzeptanzkriterien, weil sie einen 
klaren Rahmen für die Überprüfung der Wirksamkeit von Sicherheitsmaßnahmen bieten. Dieser 
Ansatz stellt sicher, dass die Sicherheit nicht nur implementiert, sondern auch getestet 
und validiert wird, wodurch die Sicherheit des Systems besser gewährleistet werden kann 
und das System allgemein robuster wird.

\section{Testszenarien}

Mithilfe von verschiedenen Testszenarien lässt sich eine bessere Sicherheit für das 
Gesamtsystem sicherstellen. Die Verwendung von Sicherheits-Akzeptanzkriterien ist hier 
sehr dienlich, da diese eine konkrete Ableitung eines Testszenarios zulassen.

Um diese Ausarbeitung in einem angemessenen Umfang zu halten habe ich untenstehend zwei 
Szenarien je definierter nichtfunktionaler Sicherheitsanforderung aus dem zweiten 
Abschnitt erarbeitet:

\subsubsection*{Authentifizierung}

\begin{quote}
Nach mehrfachen Versuchen, sich mit falschen MFA-Codes zu authentifizieren muss 
sichergestellt werden, dass die Mechanismen zur Kontosperrung ausgelöst wurden.
\end{quote}

\begin{quote}
Versuch die gespeicherten Passwörter zu knacken, um die Stärke der Verschlüsselung zu 
überprüfen.
\end{quote}

\subsubsection*{Sicherheit der Datenübertragung}

\begin{quote}
Unter Verwendung eines Netzwerk-Analysetools wie z.B. Wireshark – das 
Datenübertragungen abfangen und überprüfen kann – muss verifiziert werden, ob alle Daten 
verschlüsselt sind.	
\end{quote}

\begin{quote}
Versuch die TLS-Version herabzustufen und anschließende Überprüfung, ob das System die 
Verbindung ablehnt.
\end{quote}

\subsubsection*{Zugangskontrolle}

\begin{quote}
Versuchen, auf ein Meeting ohne Einladungslink zuzugreifen. Es muss sichergestellt 
werden, dass der Zugang verweigert wird.
\end{quote}

\begin{quote}
Durchführung eines Tests mit verschiedene Benutzerrollen, um sicherzustellen, dass die 
Berechtigungen korrekt durchgesetzt werden.
\end{quote}

\subsubsection*{Datenschutz}

\begin{quote}
Versuch, unbefugten Zugriff auf verschlüsselte Besprechungsaufzeichnungen zu 
erhalten mit anschließender Übprüfung, dass der Zugriff verweigert wird.
\end{quote}

\subsubsection*{Protokollierung und Überwachung}

\begin{quote}
Generierung von Sicherheitsereignissen (z.B. durch fehlgeschlagene Anmeldeversuche, 
Datei-Uploads usw.) mit anschließender Überprüfung, ob diese korrekt protokolliert 
werden.
\end{quote}

\begin{quote}
Test des Überwachungssystems, indem verdächtige Aktivitäten simuliert werden mit 
anschließender Überprüfung, dass Warnungen generiert und darauf reagiert wird.
\end{quote}

\subsubsection*{Notfallmanagement}

\begin{quote}
Durchführung eines simulierten Sicherheitsverstoßes und anschließende Befolgung des 
Reaktionsplans, um dessen Wirksamkeit sicherzustellen.
\end{quote}

\begin{quote}
Überprüfung des Berichts über die Reaktion auf einen Vorfall, um sicherzustellen, dass 
alle Schritte befolgt wurden, und um Bereiche mit Verbesserungsbedarf zu ermitteln.
\end{quote}

\end{document}
