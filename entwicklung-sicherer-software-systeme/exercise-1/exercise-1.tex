\documentclass{article}

\usepackage[a4paper]{geometry}
\usepackage[ngerman]{babel}
\usepackage[utf8]{inputenc}
\usepackage[T1]{fontenc}
\usepackage{graphicx}
\usepackage{fancyhdr}
\usepackage{xcolor}
\usepackage{float}

\graphicspath{{./images/}}

\pagestyle{fancyplain}
\fancyhf{}
\lhead{\fancyplain{}{Mara Schulke} }
\rhead{\fancyplain{}{\today}}
\cfoot{\fancyplain{}{\thepage}}

\begin{document}

\begin{titlepage}
	\begin{flushleft}
		TH Brandenburg \\
		Studiengang IT Sicherheit \\
		Fachbereich Informatik und Medien \\
		Entwicklung Sicherer Softwaresysteme \\
		Prof. Dr.-Ing. Martin Schafföner
	\end{flushleft}

	\vfill

	\begin{center}
		\Large{Einsendeaufgabe: Threat Analysis}\\[0.5em]
		\large{Sommersemester 2024}\\[0.25em]
		\large{Abgabetermin \today}
	\end{center}

	\vfill

	\begin{flushright}
		Mara Schulke \\
		Matrikel-Nr. 20215853
	\end{flushright}
\end{titlepage}

\begin{abstract}
	In dieser Einsendeaufgabe wird die Online-Meeting-Plattform Google Meet der Firma 
	Google hinsichtlich ihrer softwareseitigen Risiken und Gefahren analysiert.
	Es wird die Vorgehensweise der Gefahrenanalyse, die entdeckten Risiken und Gefahren 
	zusammengefasst und eine Priorisierung mit Handlungsempfehlung daraus abgeleitet. Des 
	weiteren wird auf die Methodik der Gefahrenanalyse eingegangen und von alternativen 
	Vorgehensweisen abgegrenzt.
\end{abstract}

\tableofcontents

\listoffigures

\newpage

\section{Projektauswahl: Google Meet}

Die Auswahl des zu analysierenden Projektes viel auf die Onlne-Meeting-Plattform Google 
Meet da diese mehrere Punkte abdeckt:

\begin{enumerate}
	\item Ihre Funktionsweise- und -umfang ist ihm Rahmen einer Kurzanalyse greifbar
	\item Die Anwendungsarchitektur lässt sich aus vergleichbaren Echtzeit-Meeting Anwendungen ableiten und vereinfachen
	\item Sie hat einen weitreichenden Bekanntheitsgrad
\end{enumerate}

Vereinfacht lässt sich Google Meet in einer Client-Server Architektur darstellen, auch 
wenn an dieser Stelle angemerkt werden muss, dass durch die Menge an Nutzern, die Vielzahl 
an Integrationen und die Stabilität der Software davon auszugehen ist, dass es sich nicht 
um eine naive Architektur der Server bzw. Server-Infrastruktur handelt. 

Im Rahmen dieser Einsendeaufgabe wird die Annahme getroffen, dass Google Meet der folgenden Architektur entlang aufgebaut ist:

%\begin{figure}[H]
%	\includegraphics[width=0.75\textwidth]{images/01-architecture.png}
%	\centering
%	\caption{Anwendungsarchitektur Google Meet}
%\end{figure}

Dies ist notwendig um in einem zeitlich angemessenen Rahmen in der Lage zu sein eine 
beispielhafte Gefahrenanalyse durchführen zu können.

\section{Auswahl der Gefahrenanalyse Software}

Um eine automatisierte Gefahrenanalyse durchzuführen stehen mittlerweile (Stand Mai 2024), 
eine Vielzahl an verschiedenen Tools zur Verfügung. Nachfolgend sind einige beliebtesten 
bzw. bekanntesten, kostenlosen \textit{Threat-Modelling}-Programme aufgelistet:

% [1]

\subsection*{Kostenlose Software}

Da diese Einsendeaufgabe von geringem Umfang ist und keine weiteren Anwendungsfälle für 
eine ausgereifte Threat-Modelling Software absehbar sind beschränke ich den Vergleich 
verschiedener Software-Lösungen auf kostenlose bzw. Open-Source Programme.

\subsubsection*{MTMT – Microsoft Threat Modeling Tool}

MTMT ist das Threat Modeling Tool der Firma Microsoft. Es ist ein weit verbreitetes 
Werkzeug zur Modellierung von Software und der Automatisierten Analyse dieser Modelle. Da 
der Anbieter Microsoft ist, besteht leider die Anforderung MTMT auf einem Windows-System 
auszuführen.

Das MTMT ist weiterhin ein sehr umfangreiches Werkzeug, das ebenfalls im professionellen 
Kontext Einsatz findet – dies resultiert in einer komplexen Oberfläche mit 
vielen Möglichkeiten und ausgereiften Konzepten.

Leider musste ich durch die Gegebenheit, kein Windows-System zu betreiben, die Analyse 
des MTMT hier beenden, da es mir leider nicht möglich war eine ausführbare Version für 
Unix-Systeme zu finden ohne Virtualisierung zu verwenden.

\subsubsection*{OWASP ThreatDragon}

\subsubsection*{AWS ThreatComposer}

\subsubsection*{Threagile}

Threagile ist kostenlose Open-Source Software zur Gefahrenanalyse ohne eine 
Benutzeroberfläche. Dies ist eine Eigenheit im Vergleich zu den anderen vorgestellten 
Programmen die hauptsächlich über das Desktop- oder Web-Interface zu bedienen sind.
Threagile lässt sich über eine .yaml Datei bedienen die einem vorgegebenen Schema 
entsprechen muss. So lassen sich im Quelltext Komponenten (und ihre Beziehungen), 
Datenbanken, Verschlüsselungen, Datenflüsse, verwendete Technologien und vieles mehr 
Dokumentieren.

Der klare Vorteil von Threagile liegt in der Möglichkeit diese YAML-Datei bzw. die 
analysierte Architektur ohne weiteres in git oder anderen VCS (kurz für \textit{Version 
Control Systems}) zu versionieren. Bei anderer proprietär Software ist dies zwar 
grundsätzlich Möglich aber da VCS in der Regel auf Klartext-Dateien ausgelegt sind sind 
die Möglichkeiten begrenzt bzw. parallel im Team an einer Datei zu arbeiten.

Des Weiteren ist es einfach mittels Threagile eine Gefahrenanalyse auf der YAML-Datei 
auszuführen und der generierte Report hat eine hohe Qualität (optisch, strukturell und 
inhaltlich).

\subsection{Begründung der Auswahl}

Im Rahmen dieser Einsendeaufgabe habe ich mich unter den oben dargestellten Abgrenzungen 
der verschiedenen Softwarelösungen für die Nutzung von Threagile entschieden. Threagile 
ist einfach zu bedienen, reproduzierbar (mittels Docker), eine nützliche Resource für 
zukünftige berufliche Situationen.

\section{Ergebnis der automatisierten Gefahrenanalyse}

\section{Detaillierte Gefahrenanalyse}

Analyze some of the threats in more detail.

\section{Priorisierung der Gefahren}

\section{Zusammenfassung}

% [1] https://www.iriusrisk.com/resources-blog/recommended-threat-modeling-tools


\end{document}
