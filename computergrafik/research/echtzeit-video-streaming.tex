\documentclass[journal]{IEEEtran}

\usepackage[a4paper, left=2cm, right=2cm, top=2cm, bottom=3cm]{geometry}
\usepackage[ngerman]{babel}
\usepackage[utf8]{inputenc}
\usepackage[T1]{fontenc}
\usepackage{hyperref}
\usepackage{lipsum}
\usepackage{xcolor}

\hypersetup{colorlinks=true, allcolors=black}

\title{Ansätze zum Echtzeit-Video-Streaming im Web}
\author{
	\IEEEauthorblockN{Maximilian Schulke \textit{(Matrikel-Nr. 20215853)}}\\
	\IEEEauthorblockA{
		Technische Hochschule Brandenburg \\
		B.Sc. Medieninformatik \\
		Computergrafik
	}
}


\begin{document}

\markboth{Hausarbeit Computergrafik – Maximilian Schulke}{}
\IEEEspecialpapernotice{
	betreut durch Prof.\ Dr.\ rer.\ nat.\ Reiner Creutzburg\\
	Wintersemester 2021\\
	Abgabetermin \today
}

\maketitle

\begin{abstract}
	\lipsum[1-2][2-3]
	\lipsum[1-2][2-3]
	Das abstract schreibe ich zu letzt!
\end{abstract}

\tableofcontents

\section{Einleitung / Motivation}
\IEEEPARstart{M}{\MakeLowercase{it}} der zunehmenden Vernetzung der Arbeitswelt
in den letzten Jahren \textcolor{red}{quelle suchen corona zoom} werden auch
digitale Meeting-Systeme immer relevanter und müssen immer mehr Nutzer in
Echtzeit mit einander verbinden um einen reibungslosen Arbeitsalltag zu
gewährleisten. Dies impliziert natürlich auch dass eine \textcolor{red}{quelle
suchen ab wann gute Qualität} Verbindungsqualität gegeben sein muss, damit die
Systeme nutzbar bleiben.

Aber wie können wir skalierbare Meeting-Systeme realisieren ohne große
Datenmengen über einen Streaming-Server zu schicken der diese an alle anderen
broadcasted? Die Entwicklung der letzten Jahre deuten immer mehr darauf hin,
dass \textit{Peer-To-Peer} basierte Lösungensansätze aufgrund der besseren
Performance und Skalierbarkeit, in der Regel die bessere Wahl darstellen
\textcolor{red}{quelle suchen}. Natürlich spielen zur Auswahl der Architektur
noch weitere Parameter eine wichtige Rolle (z. B. die maximale Bandbreite und
Rechenleistung der Endgeräte), aber mit immer großer werdenden
Heimnetz-Leitungen und zunehmender Rechenleistung der Endgeräte stellt dies
meistens kein Problem mehr da. \textcolor{red}{quelle suchen}.

Die Problematik der Echtzeit-Kommunikation im Web beschäftigt auch
das \textit{W3C} seit 2011 im Zuge der Standardisierung des seit diesem Jahr
zum Web-Standard erklärten Protokoll \textit{WebRTC}. \textcolor{red}{quelle
suchen}.

Es ist also (immer noch) eine sehr aktuelle Thematik in der Informatik
Echtzeit- oder \textcolor{blue}{Nahe-Zu-Echtzeit-Kommunikation} zuverlässig zu
bewältigen. Die Aufgabe dieser Arbeit soll sein, einen Überblick über den
Stand der Architekturmuster, Protokolle und möglicher Problematiken bei der
Implementierung von eigenen Echtzeit-Video-Streaming-Diensten geben,
\textcolor{orange}{diese anhand eines Experiments durchsprechen und auswerten
usw.}

\section{Historie der Echzeit-Übertragung}

\subsection{1996: RTP (RFC 18XX)}
\subsection{2003: RTP (RFC 35XX)}
\subsection{2005: RTMP von Adobe}
\subsection{20XX:\ SRTP (RFC XXXX)}
\subsection{2010: WebRTC (RFC XXXX)}

Hat mit 1996 RTP und dann 2005 mit RTMP angefangen, wann kam SRTP?\ JSEP
Global IP Solutions \& Google WebRTC kam 2010, Microsoft wollte CU-RTC

> Datenströme sind verpflichtend zu verschlüsseln. Dazu werden Verbindungen über DTLS verschlüsselt und Audio- und Videokommunikation zusätzlich durch SRTP geschützt.[16]


Die ersten \textit{Requests for Comments} oder auch \textit{RFC} zu
den grundlegenden Protokollen \textit{RTP} und
\textit{\textcolor{blue}{Protokoll X}}, auf denen die heutigen Protokolle
weitestgehend aufbauen, wurden bereits in den späten 1990er Jahren
veröffentlicht und seit dem immer weiterentwickelt. \textcolor{red}{quelle
anhängen RFC35XX}.

\lipsum[1-3][3-30]


Das ist eine referenzierte aussage\cite{WebRTC}

\section{Architekturmuster}
\lipsum[1-4][1-10]

\subsection{Peer-To-Peer}
\lipsum[1-2][2-3]
\subsection{Relay}
\lipsum[1-3][3-30]

\section{Protokolle}
\lipsum[1-2][2-3]
\subsection{WebRTC}
\lipsum[1-4][1-10]
\subsection{RTP}
\lipsum[1-3][3-30]
\subsection{RTCP}
\lipsum[1-4][1-10]
\subsection{RTSP}
\lipsum[1-3][3-30]
\subsection{SID}
\lipsum[1-2][2-3]
\subsection{STUN}
\lipsum[1-4][1-10]
\subsection{TURN}
\lipsum[1-2][2-3]

\section{Implementierung eines Live-Streams}
\lipsum[1-2][2-3]
\section{Benchmarks}
\lipsum[1-4][1-10]
\section{Auswertung}
\lipsum[1-2][2-3]

\section{Nächste Schritte}

\lipsum

\large{IPV6 only => NAT holepunching entfällt}

\section{Abbildungsverzeichnis}
abc
def

\bibliographystyle{IEEEtran}
\bibliography{IEEEabrv,references}

\end{document}
