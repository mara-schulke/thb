\documentclass[journal]{IEEEtran}

\usepackage[a4paper, left=2cm, right=2cm, top=2cm, bottom=3cm]{geometry}
\usepackage[ngerman]{babel}
\usepackage[utf8]{inputenc}
\usepackage[T1]{fontenc}
\usepackage{hyperref}
\usepackage{lipsum}

\hypersetup{colorlinks=true, allcolors=black}

\title{Ansätze zum Echtzeit-Video-Streaming im Web}
\author{
	\IEEEauthorblockN{Maximilian Schulke \textit{(Matrikel-Nr. 20215853)}}\\
	\IEEEauthorblockA{
		Technische Hochschule Brandenburg \\
		B.Sc. Medieninformatik \\
		Computergrafik
	}
}


\begin{document}

\markboth{Hausarbeit Computergrafik – Maximilian Schulke}{}
\IEEEspecialpapernotice{
	betreut durch Prof.\ Dr.\ rer.\ nat.\ Reiner Creutzburg\\
	Wintersemester 2021\\
	Abgabetermin \today
}

\maketitle

\begin{abstract}
	\lipsum[1-2][2-3]
	\lipsum[1-2][2-3]
	Das abstract schreibe ich zu letzt!
\end{abstract}

\tableofcontents

\section{Einleitung / Motivation}
\IEEEPARstart{D}{\MakeLowercase{as}} wird die Einleitung.

\section{Historie der Echzeit-Übertragung}
\lipsum[1-3][3-30]


Das ist eine referenzierte aussage\cite{WebRTC}

\section{Architekturmuster}
\lipsum[1-4][1-10]

\subsection{Peer-To-Peer}
\lipsum[1-2][2-3]
\subsection{Relay}
\lipsum[1-3][3-30]

\section{Protokolle}
\lipsum[1-2][2-3]
\subsection{WebRTC}
\lipsum[1-4][1-10]
\subsection{RTP}
\lipsum[1-3][3-30]
\subsection{RTCP}
\lipsum[1-4][1-10]
\subsection{RTSP}
\lipsum[1-3][3-30]
\subsection{SID}
\lipsum[1-2][2-3]
\subsection{STUN}
\lipsum[1-4][1-10]
\subsection{TURN}
\lipsum[1-2][2-3]

\section{Implementierung eines Live-Streams}
\lipsum[1-2][2-3]
\section{Benchmarks}
\lipsum[1-4][1-10]
\section{Auswertung}
\lipsum[1-2][2-3]

\section{Nächste Schritte}

\lipsum

\section{Abbildungsverzeichnis}
abc
def

\bibliographystyle{IEEEtran}
\bibliography{IEEEabrv,references}

\end{document}
