\documentclass[journal]{IEEEtran}

\usepackage[a4paper, left=2cm, right=2cm, top=2cm, bottom=3cm]{geometry}
\usepackage[ngerman]{babel}
\usepackage[utf8]{inputenc}
\usepackage[T1]{fontenc}
\usepackage{hyperref}
\usepackage{lipsum}
\usepackage{xcolor}

\hypersetup{colorlinks=true, allcolors=black}

\title{Implementierung eines Atuhentifizierungs-Servers mit FIDO2 Support}
\author{
	\IEEEauthorblockN{Mara Schulke \textit{(Matrikel-Nr. 20215853)}}\\
	\IEEEauthorblockA{
		Technische Hochschule Brandenburg \\
		B.Sc. IT Sicherheit \\
		Hardware Sicherheit
	}
}

\begin{document}

\markboth{Hausarbeit Hardware Sicherheit – Mara Schulke}{}
\IEEEspecialpapernotice{
	betreut durch Prof.\ Dr.\ Oliver Stecklina\\
	Sommersemester 2022\\
	Abgabetermin \today
}

\maketitle

\begin{abstract}
	\lipsum[3]
\end{abstract}

\section{Einleitung}
\IEEEPARstart{A}{\MakeLowercase{uthentifizierung}} ist eines der größten
Probleme die durch verteilte Systeme entstehen und auf die es viele mögliche
Antworten gibt. 

\section{Der FIDO2 Standard}

FIDO2 steht für \textit{F}ast \textit{ID}entity \textit{O}nline 2 und ist ein von
der FIDO Alliance entwickleter, offener und lizenzfreier Standard für
Hardware-Token gestützte Authentifizierung. (\textcolor{red}{quote})





\subsection{Welche Probleme löst FIDO2?}

Die Notwendigkeit für einen solchen Standard hat sich in den letzten Jahren
immer stärker gezeigt, da die klassische Knowledge-Based-Authentication
(\textcolor{red}{KBA})
durch zunehmende Rechenleistung und effizientere Angriffe immer unsicherer und
unhandlicher für Nutzer wird. Die minimale Passwortlängen steigen
dementsprechend an und führen zur Wiederverwendung von gleichen Login-Daten für
mehrere Online-Dienste. Bekannte Lösungen sind die Verwendung von sog.
Passwort-Managern um lange und zufällige Passwörter für verschiedenste
Online-Dienste zu verwenden ohne, dass sich Nutzer diese merken müssen. Solche
Passwort-Manager sind zwar eine Lösung für die sichere Aufbewahrung von langen
Passwörtern, können aber nicht die durch \textcolor{red}{KBA} eröffneten
Angriffsvektoren wie z.B. Man-In-The-Middle-Attacken. So bald ein Angreifer
in den Besitz des geheimen Wissens (in diesem Fall das Passwort) gelangt kann
dieser uneingeschränkt und unbegegrenzt oft auf das Zielsystem zugreifen, bis
der Nutzer seine Daten ändert (vorausgesetzt, der Angreifer hat dies noch nicht
getan).

Durch den Wechsel von \textcolor{red}{KBA} auf Zero-Knowledge-Proof basierte
Authentifizierungsmethoden lassen sich ganze Angriffsvektoren ausschließen, da
ein kompromitierter Server oder eine kompromitierte Verbindung niemals das
geheime Wissen des Nutzers einem Angreifer zugänglich machen. Das heißt, dass
sich ein Angreifer im Falle einer kompromitierten Verbindung maximal in die
Sitzung des Nutzers einschleichen könnte, allerdings bei der nächsten
Authentifizierung nicht erneut die Identität des Nutzers beweisen könnte und
somit den Zugriff verlieren würde.

Im Falle von FIDO2 kennt nichtmal der Nutzer selber seine Schlüssel da diese
auf einem Trusted-Plattform-Modul \textcolor{red}{TPM} oder einem externen
Hardware-Token gespeichert wird und der Beweis der Identität durch die Signatur
einer vom Authentifizierungs-Server ausgestellten Challange erfolgt die das TPM
oder der Hardware-Token intern durchführen und dem Nutzer nur die Signatur
zurückgeben. So stellt selbst ein kompromitiertes Nutzersystem nur eine
temporäre Schwachstelle dar.


\subsection{Übersicht von FIDO2 kompatibler Hardware}

\section{Relevante Protokolle: CTAP \& WebAuthn}
\subsection{CTAP}
Client To Authenticator Protocol

\subsection{WebAuthntication}


\section{Implementierung des Authetifizierungs-Servers}

\lipsum

\section{Technische Dokumentation}

\lipsum

\section{Auswertung}

\lipsum

\section{Abbildungsverzeichnis}

\bibliographystyle{IEEEtran}
\bibliography{IEEEabrv,references}

\end{document}
