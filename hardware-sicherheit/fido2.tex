\documentclass[journal]{IEEEtran}

\usepackage[a4paper, left=2cm, right=2cm, top=2cm, bottom=3cm]{geometry}
\usepackage[ngerman]{babel}
\usepackage[utf8]{inputenc}
\usepackage[T1]{fontenc}
\usepackage{hyperref}
\usepackage{lipsum}
\usepackage{xcolor}

\hypersetup{colorlinks=true, allcolors=black}

\title{Implementierung eines Atuhentifizierungs-Servers mit FIDO2 Support}
\author{
	\IEEEauthorblockN{Mara Schulke \textit{(Matrikel-Nr. 20215853)}}\\
	\IEEEauthorblockA{
		Technische Hochschule Brandenburg \\
		B.Sc. IT Sicherheit \\
		Hardware Sicherheit
	}
}

\begin{document}

\markboth{Hausarbeit Hardware Sicherheit – Mara Schulke}{}
\IEEEspecialpapernotice{
	betreut durch Prof.\ Dr.\ Oliver Stecklina\\
	Sommersemester 2022\\
	Abgabetermin \today
}

\maketitle

\begin{abstract}
	\lipsum[3]
\end{abstract}

\section{Einleitung}
\IEEEPARstart{T}{\MakeLowercase{est}}

\section{Der FIDO2 Standard}

FIDO steht für \textit{F}ast \textit{ID}entity \textit{O}nline und ist ein von
der FIDO Alliance entwickleter offener Standard für Hardware-Token gestützte
Authentifizierung. (\textcolor{red}{quote})


\subsection{Welches Problem löst FIDO2?}
\subsection{Übersicht von FIDO2 kompatiblen Authenticatorn}

\section{Relevante Protokolle: CTAP \& WebAuthn}
\subsection{CTAP}
Client To Authenticator Protocol

\subsection{WebAuthntication}


\section{Implementierung des Authetifizierungs-Servers}

\lipsum

\section{Technische Dokumentation}

\lipsum

\section{Auswertung}

\lipsum

\section{Abbildungsverzeichnis}

\bibliographystyle{IEEEtran}
\bibliography{IEEEabrv,references}

\end{document}
