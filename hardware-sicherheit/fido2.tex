\documentclass[journal]{IEEEtran}

\usepackage[a4paper, left=2cm, right=2cm, top=2cm, bottom=3cm]{geometry}
\usepackage[ngerman]{babel}
\usepackage[utf8]{inputenc}
\usepackage[T1]{fontenc}
\usepackage{hyperref}
\usepackage{graphicx}
\usepackage{lipsum}
\usepackage{xcolor}

\graphicspath{ {./images/} }

\hypersetup{colorlinks=true, allcolors=black}

\title{Implementierung eines Atuhentifizierungs-Servers mit FIDO2 Support}
\author{
	\IEEEauthorblockN{Mara Schulke \textit{(Matrikel-Nr. 20215853)}}\\
	\IEEEauthorblockA{
		Technische Hochschule Brandenburg \\
		B.Sc. IT Sicherheit \\
		Hardware Sicherheit
	}
}

\begin{document}

\markboth{Hausarbeit Hardware Sicherheit – Mara Schulke}{}
\IEEEspecialpapernotice{
	betreut durch Prof.\ Dr.\ Oliver Stecklina\\
	Sommersemester 2022\\
	Abgabetermin \today
}

\maketitle

\begin{abstract}
\end{abstract}

\section{Einleitung}
\IEEEPARstart{A}{\MakeLowercase{uthentifizierung}} ist eines der größten
Probleme die durch verteilte Systeme entstehen. Es gibt zahlreiche
Möglichkeiten die Identität einer Gegenseite sicherzustellen allerdings weisen
viele von ihnen Schwachstellen hinsichtlich Man-In-The-Middle-Attacken und
basieren auf der Annahme, dass das System des Nutzers nicht kompromitiert
wurde. \textcolor{red}{tbd}

\section{Der FIDO2 Standard}

FIDO2 steht für \textit{F}ast \textit{ID}entity \textit{O}nline 2 und ist ein von
der FIDO Alliance entwickleter, offener und lizenzfreier Standard für
Hardware-Token gestützte Authentifizierung. (\textcolor{red}{quote})

Ein Hardware-Token (kann auch in Form eines Trusted-Plattform-Moduls oder als
Teil des Betriebssystems implementiert sein) ist ein physischer Speicher für
die FIDO Schlüsselpaare eines Nutzers. Kernmerkmale die FIDO2 von
herkömmlicher asymmetrischer Kryptografie unterscheiden sind beispielsweise die
Isolation der privaten Schlüssel auf dem Hardware-Token, die Notwendigkeit
einer Nutzerinteraktion zum Verwenden eines privaten Schlüssels und die
Generierung von einem Schlüsselpaar pro Online-Dienst.

Durch all diese Eigenschaften werden Schlüsselverluste unwahrscheinlicher und
weniger Sicherheitskritisch, da selbst bei einem hypothetischen
Schlüsselverlustes der Schaden immer auf einen Online-Dienst begrenzt ist.
Die größte Schwachstelle ist allerdings der physische Diebstahl des
Hardware-Tokens, da dessen Besitz ausreicht für Impersonation-Attacken. Als
Absicherung dagegen kann eine biometrische Authentifizierung erfolgen bevor ein
privater Schlüssel verwendet werden kann - wie beispielsweise bei FaceID.\

\textcolor{red}{tbd}

\subsection{Welche Probleme löst FIDO2?}

Die Notwendigkeit für einen solchen Standard hat sich in den letzten Jahren
immer stärker gezeigt, da die klassische Knowledge-Based-Authentication
(\textcolor{red}{KBA})
durch zunehmende Rechenleistung und effizientere Angriffe immer unsicherer und
unhandlicher für Nutzer wird. Die minimale Passwortlängen steigen
dementsprechend an und führen zur Wiederverwendung von gleichen Login-Daten für
mehrere Online-Dienste. Bekannte Lösungen sind die Verwendung von sogenannten
Passwort-Managern um lange und zufällige Passwörter für verschiedenste
Online-Dienste zu verwenden ohne, dass sich Nutzer diese merken müssen. Solche
Passwort-Manager sind zwar eine Lösung für die sichere Aufbewahrung von langen
Passwörtern, können aber nicht die durch \textcolor{red}{KBA} eröffneten
Angriffsvektoren wie z.B. Man-In-The-Middle-Attacken. So bald ein Angreifer
in den Besitz des geheimen Wissens (in diesem Fall das Passwort) gelangt kann
dieser uneingeschränkt und unbegegrenzt oft auf das Zielsystem zugreifen, bis
der Nutzer seine Daten ändert (vorausgesetzt, der Angreifer hat dies noch nicht
getan).

Durch den Wechsel von \textcolor{red}{KBA} auf Zero-Knowledge-Proof basierte
Authentifizierungsmethoden lassen sich ganze Angriffsvektoren ausschließen, da
ein kompromitierter Server oder eine kompromitierte Verbindung niemals das
geheime Wissen des Nutzers einem Angreifer zugänglich machen. Das heißt, dass
sich ein Angreifer im Falle einer kompromitierten Verbindung maximal in die
Sitzung des Nutzers einschleichen könnte, allerdings bei der nächsten
Authentifizierung nicht erneut die Identität des Nutzers beweisen könnte und
somit den Zugriff verlieren würde.

Im Falle von FIDO2 kennt nichtmal der Nutzer selber seine Schlüssel da diese
auf einem Trusted-Plattform-Modul \textcolor{red}{TPM} oder einem externen
Hardware-Token gespeichert wird und der Beweis der Identität durch die Signatur
einer vom Authentifizierungs-Server ausgestellten Challenge erfolgt die das
\textcolor{red}{TPM} oder der Hardware-Token intern durchführen und dem Nutzer
nur die Signatur zurückgeben. So stellt selbst ein kompromitiertes Nutzersystem
nur eine temporäre Schwachstelle dar.


\section{Relevante Protokolle: CTAP \& WebAuthn}

Der FIDO2 Standard umfasst hauptsächlich die beiden Protokolle \textit{CTAP}
und \textit{WebAuthn}. Diese unterteilen den gesammten
Authentifizierungsvorgang in
zwei Bereiche:

\textit{Client zu Authenticator} also die Kommunikation zwischen dem
Nutzersystem und dem Hardware-Token oder TPM und \textit{Client zu Server}, die
Kommunikation zwischen dem Nutzersystem und dem Online-Dienst bei dem sich der
Nutzer authentifizieren möchte.

\begin{figure}[ht]
	\includegraphics[width=0.5\textwidth]{ctap-webauthn-registration.png}
	\centering
	\caption{Unterteilung des Kommunikationsablaufs in CTAP und WebAuthn am
	beispiel einer Schlüssel-Registrierung}\label{fig:ctap-webauthn}
\end{figure}

\subsection{Gemeinsame Begriffsdefinitionen}

\subsubsection{Relying Party}

Ein Online-Dienst der den FIDO2 Standard zur Nutzerauthentifizierung verwendet.

\subsubsection{Authenticator}

Ein externer Hardware-Token oder ein Teil des Betriebssystems (z.B. über ein
TPM implementiert) der FIDO2 Schlüsselpaare verwaltet.

\subsubsection{Credential}

Ein FIDO2 Schlüssel

\subsubsection{CBOR}

Concise Binary Object Representation, ein Binär-Format zur Darstellung von JSON Objekten.

\subsection{CTAP}

Das Client-To-Authenticator-Protocol kurz \textcolor{red}{CTAP} ist Teil
des FIDO2 Standards und beschreibt den Ablauf der Kommunikation zwischen einem
Nutzersystem und dem Hardware-Token beziehungsweise dem Authenticator.

Neben einer Spezifikation für den Transportlayer / die Nachrichtenstruktur
besteht das Protokoll primär aus der sogenannten ``Authenticator API'' - diese
beschreibt Operationen die ein Nutzersystem auf einem Authenticator ausführen kann.

Spezifiziert sind die folgenden 6 Operationen:

\begin{enumerate}
	\item \textit{authenticatorMakeCredential} - Schlüsselpaar für eine
		``Relying Party'' erstellen
	\item \textit{authenticatorGetAssertion} - Signatur einer Challenge
	\item \textit{authenticatorGetNextAssertion} - Nächste Signatur der
		Challenge erhalten bei mehreren Schlüsselpaaren
	\item \textit{authenticatorGetInfo} - Informationen über die Fähigkeiten
		des Authenticators
	\item \textit{authenticatorClientPIN} - Setzt den Authenticator-PIN
	\item \textit{authenticatorReset} - Zurücksetzen auf Werkseinstellungen
\end{enumerate}

\textcolor{red}{CTAP Spec references}

Die im Standard beschriebenen Transportlayer umfassen USB, NFC oder Bluetooth
Low Energy. Für jeden dieser Transportlayer gibt es eigene Mechanismen zum
sicheren Verbindungsaufbau und zur Nachrichtenstruktur.
\textcolor{red}{CTAP Spec references}

Die JSON Objekte werden innerhalb von \textit{CTAP} mittels dem Datenformat
\textit{CBOR} dargestellt um eine effizientere Codierung zu erhalten und
dennoch die kompatiblität mit dem weitverbreiteten Standard für
Datenserialisierung zu garantieren.
\textcolor{red}{CBOR references}

\subsection{WebAuthn}

Ein Server der das WebAuthn Protokoll implementiert enthält im wesentlichen
zwei grundlegende Operationen die jeweils aus mehreren Schritten bestehen.
Diese sind die Registrierung von Sicherheitsschlüsseln (Credentials) und deren
zuordnung zu einem Nutzeraccount im System der Relying Party und die
Austellung beziehungsweise Validierung von WebAuthn Challenges die bei
erfolgreicher Validierung einem Nutzer zugriff auf das System der Relying Party
gibt.

Ein wichtiger Punkt im Zusammenspiel der Spezifikationen und folgich der
Implementierungen der beiden Protokolle \textit{CTAP} und \textit{WebAuthn} ist
die kompatiblität der Datenttypen. So lässt sich die spezifizierte Ausgabe
einer WebAuthn Registrierungs Challenge ohne veränderung mit der
``authenticatorMakeCredential'' Operation an den Authenticator weitergeben.

Dies hat zur Folge dass die Client-Implementierung selbst keine
protokollrelevanten kryptografischen Berechnungen durchführen muss und effektiv
der Server der Relying Party mit dem Authenticator des Nutzers kommuniziert.
Die Client-Implementierung leitet lediglich jeweils die Ausgaben weiter.

\textcolor{red}{weubauthn spec references}

\subsubsection{Registrierung}

Um die Registrierung eines neuen Schlüssels durchzuführen muss die Relying
Party auf dem Server \texttt{PublicKeyCredentialCreationOptions} für den Nutzer
generieren und an den Client zurücksenden. Diese enthalten Informationen wie
zum Beispiel erlaubte kryptografische Algorithmen, Informationen über den
Nutzer und die Relying Party, ausgeschlossene Authenticator etc.

Diese Parameter nimmt der Client entgegen und gibt diese dann anschließend an
die \textit{CTAP}-Operation ``authenticatorMakeCredential'' weiter. Der
Authenticator überprüft anschließend ob er ein Schlüsselpaar erstellen kann
dass den Anforderungen des Servers entspricht und gibt entweder einen Fehler
oder den öffentlichen Schlüssel an den Client zurück. Dieser wird nun vom
Client an den Server weitergeben und die Registrierung wurde erfolgreich
abgeschlossen.

\textcolor{red}{weubauthn spec references}

\subsubsection{Authentifizierung}

Für die Authentifizierung eines Nutzers mittels WebAuthn muss zu erst die
Relying Party eine Challenge für den Client generieren. Diese muss zufällig und
nur einmal gültig sein um Replay-Attacken zu verhindern. Nach dem der Client
seine Challenge erhalten hat kann er diese über die CTAP-Operation
``authenticatorGetAssertion'' signieren lassen und die signierte Challenge
wieder an den Server zurückgeben. Anschließend überprüft dieser ob einer der
für den Nutzer hinterlegten Schlüssel verwendet wurde um die Challenge zu
signieren. War dies der Fall gilt die Identität des Nutzers als verifiziert und
der Server kann dem Nutzer Zugriff auf geschützte Ressourcen geben.

\textcolor{red}{weubauthn spec references}

\vspace{0.25em}
\rule{0.45\textwidth}{0.4pt}
\vspace{0.5em}

Je nach Kommunikationsprotokoll zwischen dem Client und dem Server muss sich
der Server den Status der Registrierung bzw. Authentifizierung über einzelne
Nachrichten hinweg merken. So sind (auf Nachrichtenebene) verbindungslose
Protokolle wie bspw. HTTPS davon betroffen. Eine bidirektionaler Transport
würde dieses Problem umgehen, da sich der Server nur den Status für die Dauer
der Verbindung merken müsste.

\section{Implementierung des Authetifizierungs-Servers}

\subsection{Zielsetzung}

Ziel der Proof-of-Concept Implementierung ist es eine funktionale,
flüchtige Nutzer-Verwaltung die optional FIDO2 Schlüssel als zweiten
Login-Faktor unterstützt. Diese soll eine API anbieten auf der eine
Client-Implementierung aufsetzen kann.

\subsection{Authentifizierungsmechanismus}

Nutzer ohne FIDO2 Schlüssel können sich bei dem Server mittels der Kombination
aus E-Mail und ihrem Passwort authentifizieren und erhalten sofort einen Token
zurück.

Die Nutzer die einen FIDO2 Schlüssel registriert haben erhalten nach
initialer Angabe ihrer E-Mail und ihres Passworts eine Aufforderung ihren FIDO2
Schlüssel zu verwenden um letztendlich auch einen Token zu erhalten.

Der Token ist ein JSON-Web-Token der eine Nutzer-Id enthält und ein
Ablaufdatum. Alle Tokens werden mit dem symmetrischen Algorithmus HMAC
(Hash-based Message Authentication Code) signiert.

Alle geschützten beziehungsweise nutzerbezogenen Ressourcen die dieser Server
vorhält dürfen nur durch die Mitgabe eines Tokens abgerufen werden. Dieser muss
mit dem \texttt{Authorization}-Header gesetzt werden werden.

\subsection{Speicherung der Nutzer-Passwörter}

Da bekanntermaßen in Klartext gespeicherte Passwörter eine verherende
Sicherheitslücke darstellen speichert der Server nur den Bcrypt-Hash der
Passwörter die durch einen sogenannten Salt randomisiert werden.

Sowohl der Schlüssel zur signierung von Tokens als auch der Salt für Bcrypt
müssen in einem realen Anwendungsszenario geheim gehalten werden.

\subsection{Datenstrukturen zur flüchtigen Verwaltung von Nutzern}

Ein Nutzer erhält bei der Registrierung eine eindeutige Id, einen
Verifizierungscode (der in einem realen Anwendungsszenario ggf.\ per E-Mail
versendet werden könnte) und eine leere Liste an FIDO2 Schlüsseln.

Die Nutzer werden in einer HashMap im RAM vorgehalten, eine Datenbankanbindung
des Servers zur Nutzerverwaltung wären bei einem Produktivsystem unabdingbar.

Um die konsistzenz der Nutzerverwaltung innerhalb des RAM sicherzustellen
befindet sich die Nutzer-HashMap in einem Mutex um gleichzeitige ggf.
inkonsistente Schreibvorgänge zu unterbinden.

Bei jeder eingehenden Anfrage wird die Datenstrukturen zur Nutzerverwaltung der
Anfrage zugewiesen, diese hat nun die Garantie dass sie die einzige Anfrage ist
die in diesem Moment schreiben kann. Sollten mehrere Anfragen gleichzeitig
eingehen muss eine warten bis die andere abgearbeitet wurde. Dies ist zwar
absolut inperformant bei großen Anfragemengen aber bei einem
Proof-of-Concept-Server mit sehr wenigen Nutzern vertretbar.

\subsection{Implementierung von WebAuthn}

Da die Erstellung von den durch WebAuthn definierten Datenstrukturen und die
kryptografischen Algorithmen nicht trivial zu implementieren sind gibt es für
gängige Sprachen die eine Anwendung im Web-Kontext finden Bibliotheken die das
WebAuthn-Protokoll abstrahiert bereitstellen.

Für Rust ist eine umfangreicherere dieser Bibliotheken ``webauthn-rs'',
verfügbar auf GitHub unter der Mozilla-Public-License-2.0:

\url{https://github.com/kanidm/webauthn-rs}

Die Bibliothek lässt sich durch Informationen über den Server konfigurieren:
Es wird eine Relying Party Id und eine registrierbare Domain erwartet um die
Identität des Servers für die Schlüsselerstellung wiederverwenden zu können.

Nach erfolgreicher Initialisierung gibt es im wesentlichen 4 relevante
Funktionen mit denen sich der WebAuthn-Flow komplett implementieren lässt:

\subsubsection{start\_securitykey\_registration}
Gibt die Public-Key-Credential-Creation-Options und einen
Schlüssel-Registrierungsstatus für den momentanen Nutzer zurück. Dieser muss
persistiert werden.

\subsubsection{finish\_securitykey\_registration}
Nimmt einen Schlüssel-Registrierungsstatus und den generierten Schlüssel und
validiert die Gültigkeit des generierten Schlüssels.

\subsubsection{start\_securitykey\_authentication}
Gibt eine Challenge und einen Schlüssel-Authetifizierungsstatus für den
momentanen Nutzer zurück. Dieser muss muss persistiert werden.

\subsubsection{finish\_securitykey\_authentication}
Nimmt eine Schlüssel-Authetifizierungsstatus und eine signierte Challenge und
validiert diese gegeninander.

Somit sind Serverseitig alle Voraussetzungen geschaffen um eine API
bereitzustellen die WebAuthn unterstützt. Detailierte Informationen zu den
bereitgestellten Endpunkten finden sich unter Abschnitt~\ref{docs}.

\section{Implementierung der Nutzer-Oberfläche}

\subsection{Zielsetzung}

Ziel der Client-Implementierung ist es beispielhaftes Nutzer-Oberfläche zu
schaffen das die unter Abschnitt~\ref{docs} beschriebene API verwendet um ein
Nutzer-Login mit FIDO2 zu ermöglichen.

\subsection{Konzept der Oberfläche}

Die Oberfläche ist eine sehr kleine JavaScript Anwendung für die API.\ Folglich
orientieren sich die Aktionen die der Nutzer in diesem ausführen kann direkt an
dieser. In der Abbildung~\ref{fig:ui-flow} sind der Ablauf aller möglichen
Nutzeraktionen und die dazugehörigen API Endpunkte abgebildet.

\begin{figure}[ht]
	\includegraphics[width=0.5\textwidth]{ui-flow.png}
	\centering
	\caption{Abfolge möglicher Nutzer-Aktionen}\label{fig:ui-flow}
\end{figure}

\subsection{Anwendungs Architektur}

Da der Umfang der Nutzer-Aktionen sich stark in Grenzen hält gibt es keinen
Grund eines der komplexeren Frameworks einzusetzen die typischerweise für
Web-Oberflächen eingesetzt werden. Die einzige externe Bibliothek ist eine
moderene und stabilere Base64 Implementierung als die Browser-Native
Alternative.

Das Grundkonzept zum rendering der Oberfläche ist einen Anwendungszustand durch
eine pure (also Seiten-Effekt freie) Funktion in HTML zu übersetzen.
Nutzeraktionen können diesen Zustand durch Interaktion verändern und lösen
somit einen neuen Render-Vorgang aus. Diese bzw. ähnliche Architekturen findet
sich in Frameworks wie React oder Elm wieder.

\textcolor{red}{refs}

Der Zustand speichert den Token, Verifizierungsstatus, Ladezustand der
Anwendung und die Liste aller registrierten Schlüssel.

\subsection{Interaktionen}

Die definierten möglichen Interaktionen die der Nutzer tätigen kann sind:

\begin{enumerate}
	\item signup
	\item verify
	\item login
	\item logout
	\item addKey
	\item removeKey
\end{enumerate}

Diese werden durch die entsprechenden Bedienfelder der Oberfläche ausgelöst.


\section{Technische Dokumentation}\label{docs}

Der Server öffnet den TCP Port 8080 und erwartet eine externe TLS
Terminierung. Tokens können dem Server über den \texttt{Authorization} Header
mitgegeben werden und der \texttt{Content-Type} aller Anfragen und Antworten
ist ausschließlich \texttt{application/json}.

\subsection{/auth/signup - Nutzer erstellen}

\begin{itemize}
	\setlength{\leftskip}{1.5cm}
	\setlength{\itemsep}{0pt}
	\item[Methode:] POST
	\item[Token:] -
	\item[Eingabe:] Credentials \{ email, password \}
	\item[Ausgabe:] UserDetails \{ token, verified, keys \}
	\item[Beschreibung:] Erstellt einen unverifizierten Nutzer ohne FIDO2 Keys.
		Gibt einen Verifizierungscode in den Server-Logs aus (könnte in einem
		echten Szenario per E-Mail verschickt werden).
\end{itemize}

\subsection{/auth/verify - Nutzer verifizieren}

\begin{itemize}
	\setlength{\leftskip}{1.5cm}
	\setlength{\itemsep}{0pt}
	\item[Methode:] POST
	\item[Token:] Notwendig
	\item[Eingabe:] Verification \{ code \}
	\item[Ausgabe:] -
	\item[Beschreibung:] Verifiziert einen Nutzer falls der Code mit dem bei
		der Registrierung generierten Code übereinstimmt.
\end{itemize}


\subsection{/auth/login - Nutzer anmelden}

\begin{itemize}
	\setlength{\leftskip}{1.5cm}
	\setlength{\itemsep}{0pt}
	\item[Methode:] POST
	\item[Token:] -
	\item[Eingabe:] Credentials \{ email, password \}
	\item[Ausgabe:] UserDetails \{ token, verified, keys \} | \textcolor{red}{webauthn challenge}
	\item[Beschreibung:] Gibt dem Nutzer entweder seine UserDetails zurück oder
		stellt eine WebAuthn Authetifizierungs-Challenge die der Nutzer
		signiert bei dem Endpunkt \textit{/auth/fido2/login} enreichen muss
		falls ein FIDO2 Schlüssel hinterlegt wurde.
\end{itemize}

\subsection{/auth/fido2/login - Nutzer mit WebAuthn anmelden}

\begin{itemize}
	\setlength{\leftskip}{1.5cm}
	\setlength{\itemsep}{0pt}
	\item[Methode:] POST
	\item[Token:] -
	\item[Eingabe:] \textcolor{red}{challenge}
	\item[Ausgabe:] UserDetails \{ token, verified, keys \}
	\item[Beschreibung:] Validiert die WebAuthn Challenge des Nutzers mit den
		hinterlegten FIDO2 Schlüsseln und gibt bei erfolgreicher Validierung
		dem Nutzer seine UserDetails zurück.
\end{itemize}

\subsection{/auth/fido2/challenges - WebAuthn Registrierungs Challenge}

\begin{itemize}
	\setlength{\leftskip}{1.5cm}
	\setlength{\itemsep}{0pt}
	\item[Methode:] POST
	\item[Token:] Notwendig
	\item[Eingabe:] -
	\item[Ausgabe:] \textcolor{red}{ccr}
	\item[Beschreibung:] Startet einen Registrierungsprozess für einen FIDO2
		Schlüssel. Setzt Serverseitig den ``Key-Registration-State'' eines
		Nutzers und gibt eine Registrierungs-Challenge zurück.
\end{itemize}

\subsection{/auth/fido2/keys - WebAuthn Registrierung abschließen}

\begin{itemize}
	\setlength{\leftskip}{1.5cm}
	\setlength{\itemsep}{0pt}
	\item[Methode:] POST
	\item[Token:] Notwendig
	\item[Eingabe:] \textcolor{red}{PublicKeyCredential}
	\item[Ausgabe:] Key \{ id \}
	\item[Beschreibung:] Nimmt die CTAP Ausgabe der
		``authenticatorMakeCredential'' Operation an und ordnet diesen
		Schlüssel dem Nutzer zu.
\end{itemize}

\subsection{/auth/fido2/keys/:id - WebAuthn Schlüssel entfernen}

\begin{itemize}
	\setlength{\leftskip}{1.5cm}
	\setlength{\itemsep}{0pt}
	\item[Methode:] DELETE
	\item[Token:] Notwendig
	\item[Eingabe:] \textit{/:id}
	\item[Ausgabe:] -
	\item[Beschreibung:] Entfernt einen FIDO2 Schlüssel anhand seiner ID.\@
\end{itemize}

\section{Auswertung}

\subsection{Nutzerbarkeit}

Nutzbarkeit abgesehen von der Verifizierungsmethode, die aus dem Terminal
abgelesen werden muss ist gut.

\subsection{Risiken}

- Fehlender Zwang FIDO2 zu verwenden

\subsection{Verbesserungsvorschläge}

- FIDO2 Key als hauptfaktor, backup codes und long lived access

\listoffigures

\bibliographystyle{IEEEtran}
\bibliography{IEEEabrv,references}

\section*{Anhang}

\subsection*{Quellcode der Implementierung}

\url{https://github.com/mara214/fido2-auth}


\end{document}
