\documentclass{beamer}

\usepackage[ngerman]{babel}
\usepackage[utf8]{inputenc}
\usepackage[T1]{fontenc}

\usecolortheme{seagull}
\usefonttheme{serif}

\title{Sichere Löschung von Festplatten und Datenspeichern}
\author{Mara Schulke\\\tiny{Matrikelnr. 20215853, SS22 B.Sc. IT Security, THB}}

\begin{document}

\begin{frame}
	IT Forensik
	\vspace{1em}
	\titlepage
\end{frame}

\begin{frame}{Übersicht}
	\tableofcontents
\end{frame}

\section{Wieso ist eine sichere Löschung so wichtig?}
\begin{frame}{Wieso ist eine sichere Löschung so wichtig?}
	\begin{itemize}
		\item verhinderung von Datendiebstahl
		\item rechtliche Vorgaben
		\item Hardware wiederverwendbar machen
	\end{itemize}
\end{frame}

\section{Gefahren durch falsche Löschung}
\begin{frame}{Gefahren durch falsche Löschung}
	\begin{itemize}
		\item Häufige Gerätewechsel bei mangelnder Aufbereitung des Speichermediums
			bergen die Gefahr dass ein Angreifer diese nach Verkauf / Entsorgung
			rekonstruieren kann
		\item Selbst normale Endanwender können mittels Software wie z.B.
			EaseUS schlecht gelöschte Speichermedien wiederherstellen 
	\end{itemize}
\end{frame}

\section{Wie funktioniert die Rekonstruktion von Datenseichern?}
\begin{frame}{Wie funktioniert die Rekonstruktion von Datenseichern?}
	\begin{itemize}
		\item Rekonstruktion ist möglich wenn Daten nicht physisch gelöscht
			wurden
			\begin{itemize}
				\item Formatierung entfernt Daten nur oberflächlich da i.d.R.
					nur die Zugriffstabellen angepasst werden und nicht die
					Register geleert werden.
			\end{itemize}
		\item Durch durchsuchen der Register einer Festplatten mit entfernter
			Zugriffstabelle lassen sich softwareseitig Datenrekonstruieren
		\item Bei Beschädigten oder mittels SMR gesicherten Festplatten ist
			eine softwareseitige Rekonstruktion oft nicht mehr möglich, hier
			gibt es allerdings noch die Möglichkeit über physischen Zugriff auf
			die Festplatte Informationen wiederherzustellen (bspw.\ von
			spezialisierten Unternhemen unter Reinraum-Bedingungen)
	\end{itemize}
\end{frame}

\section{Methoden zur sicheren Speicherlöschung}
\begin{frame}{Methoden zur sicheren Speicherlöschung}
	\begin{itemize}
		\item Physische Zerstörung des Datenträgers
		\item Überschreiben der Daten innherlab des Datenträgers
	\end{itemize}
\end{frame}

\section{Methoden zur sicheren Speicherlöschung: A}
\begin{frame}{Methoden zur sicheren Speicherlöschung: A}
	\begin{itemize}
		\item 
	\end{itemize}
\end{frame}

\section{Typsiche Speichertypen}
\begin{frame}{Typsiche Speichertypen}
	\begin{itemize}
		\item HDD (Hard Disk Drive)
		\item SSD (Solid State Drive)
		\item SSHD (Solid State Hybrid Drive)
	\end{itemize}
\end{frame}

\section{Unterschiede zwischen Speichertypen}
\begin{frame}{Unterschiede zwischen Speichertypen}
	\begin{itemize}
		\item 
	\end{itemize}
\end{frame}

\section{Shingled Magnetic Recording und der TRIM-Befehl}
\begin{frame}{Shingled Magnetic Recording und der TRIM-Befehl}
	\begin{itemize}
		\item 
	\end{itemize}
\end{frame}

% https://www.bsi.bund.de/DE/Themen/Verbraucherinnen-und-Verbraucher/Informationen-und-Empfehlungen/Cyber-Sicherheitsempfehlungen/Daten-sichern-verschluesseln-und-loeschen/Daten-endgueltig-loeschen/daten-endgueltig-loeschen_node.html
% https://www.experte.de/it-sicherheit/festplatte-loeschen
% https://www.easeus.de/festplatte-wiederherstellen/formatierte-festplatte-wiederherstellen-freeware.html
% https://de.wikipedia.org/wiki/Datenwiederherstellung

\end{document}
