\documentclass{beamer}

\usepackage[ngerman]{babel}
\usepackage[utf8]{inputenc}
\usepackage[T1]{fontenc}
%\usepackage{graphicx}

\usecolortheme{seagull}
\usefonttheme{serif}


\title{Softwarearchitektur}
\subtitle{5 in 5 Zusammenfassung}
\author{Maximilian Schulke}

\AtBeginSection[]{
	\begin{frame}{Übersicht}
		\tableofcontents[currentsection]
	\end{frame}
}

\begin{document}

\begin{frame}
	Softwaretechnik \hfill SS21
	\titlepage
	Prof. Dr. Martin Schafföner
\end{frame}

\section{Was ist Architektur?}

\begin{frame}{Was ist Architektur?}
	Architektur ist die \dots
	\begin{itemize}
		\item{strukturierte Aufteilung von Software in Teilprobleme}
		\item{systematische Anwendung von Prinzipien wie Patterns, Kapselung und
			Abstraktion}
		\item{Definition der Kommunikation und Verteilung der
			Systemkomponenten}
	\end{itemize}
\end{frame}

\section{Wieso braucht man Architektur?}

\begin{frame}{Wieso braucht man Architektur?}
	\begin{itemize}
		\item{Vermeidung des ``Big Ball of Mud''.}
		\item{Unstrukturierte Anhäufung von Code wird unübersichtlich.}
		\item{Bei steigender Komplexität hilft gute Strukturierung eine
			Übersicht zu behalten.}
		\item{Strukturierter Code ist leichter erweiterbar und wartbarer.}
		\item{Hilft die Software in Teilsysteme zu zerlegen die unabhängig
			voneinander bearbeitet werden können.}
		\item{Korrektheit und Qualität lässt sich leichter sicherstellen.}
	\end{itemize}
\end{frame}

\section{Ziele der Architektur}

\begin{frame}{Ziele der Architektur}
	\begin{itemize}
		\item{Identifikation von potenziellen Risiken}
		\item{Wartbarkeit, Erweiterbarkeit erhöhen}
		\item{Kohäsion erhöhen / Kopplung minimieren}
		\item{Identifikation von Möglichkeiten zur Wiederverwendung von
			Softwarekomponenten}
		\item{\dots}
	\end{itemize}
\end{frame}

\begin{frame}{Ziele der Architektur}
	Grundsätzlich ist eine schwache / lose Kopplung erstrebenswert.
	Es gibt mehrere Möglichkeiten dies zu erreichen, unter anderem:

	\begin{itemize}
		\item{Einsatz von Dependency Injection Frameworks}
		\item{Einsatz von Design Patterns (wie z. B. Factories)}
		\item{Kommunkation über Protokolle (oder z. B. Middleware oder
			Messaging)}
	\end{itemize}
\end{frame}

\section{Architekturebenen}

\begin{frame}{Architekturebenen}
	\begin{itemize}
		\item{Ziel (Gesamtziel der Anwendung)}
		\item{Firma (Firmenspezifische Ziele)}
		\item{Anwendung / System (Systeme innerhalb einer Firma)}
		\item{Softwareschicht (Architektur der Teilsoftware z. B. MVC)}
		\item{Package / Komponentendiagramme (detailierte Ausarbeitung)}
		\item{Klassen (Implementierungsschicht)}
	\end{itemize}
\end{frame}

\section{Standardisierung}

\begin{frame}{Standardisierung}
	Die folgenden Punkte sollten innerhalb einer Architektur standardisiert sein:

	\begin{itemize}
		\item Codestyle
		\item Test und Buildframeworks
		\item CI/CD Pipeline
		\item \dots
	\end{itemize}
\end{frame}

\section{Architekturstile}

\begin{frame}{Architekturstile}
	\begin{itemize}
		\item Kommunizierende Prozesse (z. B. Middleware oder Bussysteme)
		\item Call and Return (z. B. Prozedurale Programmierung)
		\item Datenorientierter Fluss (z. B. Batch Processing / Streams)
		\item Datenzentriert (z. B. Repositories oder Blackboards)
		\item Virtuelle Maschine (z. B. DSL oder Regelbasierte Sytsteme)
		\item Dezentrale Netze (z. B. P2P)
	\end{itemize}
\end{frame}

\section{Prinzipien der Architektur}

\begin{frame}{Prinzipien der Architektur}
	\begin{itemize}
		\item{Inkrementalität} % durch Prototyping und schrittweises Wachstum.}
		\item{Lose Kopplung} % durch Gesetz von Demeter und Vermeidung von Zyklen}
		\item{Hohe Kohäsion} % gute fachliche Gruppierungen überleben auch Technologieshifts (z. B. wie vor langer Zeit von Corba nach JavaEE)}
		\item{Entwurf für Veränderung} % Test was passiert, wenn Komponenten sich ändern oder wegfallen}
		\item{Rückverfolgbarkeit} % Änderungen, Fehler (Exceptions) und Datenflüsse transparent machen}
		\item{Abstraktion} % Interfaces als explizite Schnittstellen, Subclassing, Polymorphie}
		\item{Information Hiding} % Fassaden, Black Box Prinzip, Schichtenbildung}
		\item{Separation of Concerns} % Analyse der Zuständigkeiten}
		\item{Modularität} % Anwendung der Sprachmittel für Gruppierung}
		\item{Selbstdokumentation} % Namensgebung, interne und externe Dokumentation}
	\end{itemize}
\end{frame}

\section{Frameworks}

\begin{frame}{Frameworks}
	Frameworks liefern generische projektübergreifende Lösungen für wiederkehrende Teilprobleme.

	\vspace{0.5em}

	Bekannte Frameworks:

	\begin{itemize}
		\item{JavaEE}
		\item{Spring}
		\item{Django}
		\item{Rails}
		\item{.NET}
		\item{\dots}
	\end{itemize}
\end{frame}

% https://vfhswt.eduloop.de/loop/Enterprise_Design_Patterns_ARC
% https://vfhswt.eduloop.de/loop/Nichtfunktionale_Anforderungen
% https://vfhswt.eduloop.de/loop/Was_machen_gute_Architekt/innen%3F
% https://vfhswt.eduloop.de/loop/Architektur_Ebenen_ARC
% https://vfhswt.eduloop.de/loop/Packages_und_Namensgebung_ARC

\section{Zusammenfassung}

\begin{frame}{Zusammenfassung}
	Gute Architektur hilft Software..
	\begin{itemize}
		\item zu vereinheitlichen / zu standardisieren (u.\ a. Code, Tools..)
		\item testbarer zu machen und die qualität sicherzustellen
		\item erweiterbar und wartbar zu halten
		\item übersichtlich zu halten
	\end{itemize}
\end{frame}

\begin{frame}{Zusammenfassung}
	Danke fürs zuhören! \textbf{Gibt es Fragen?}
\end{frame}

\end{document}

