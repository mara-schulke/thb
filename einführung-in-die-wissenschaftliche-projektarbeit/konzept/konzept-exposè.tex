\documentclass{article}

\usepackage[style=apa]{biblatex}
\usepackage[a4paper]{geometry}
\usepackage[ngerman]{babel}
\usepackage[utf8]{inputenc}
\usepackage[T1]{fontenc}
\usepackage{hyperref}
\usepackage{xcolor}
\usepackage{csquotes}
\usepackage{doc}

\hypersetup{colorlinks=true, allcolors=black}

\newcommand{\titlecontent}{Drahtlose Low Power Kommunikationsprotokolle und ihre Sicherheit}

\title{\titlecontent}
\author{Friedemann Richard Pruß, Mara Schulke}

\bibliography{references}

\begin{document}

\begin{titlepage}
	\begin{flushleft}
		Technische Hochschule Brandenburg \\
		Online Studiengang Medieninformatik \\
		Fachbereich Informatik und Medien \\
		Einführung in die wissenschaftliche Projekarbeit\\
		Prof.\ Dr.\ rer.\ nat.\ Martin Christof Kindsmüller
	\end{flushleft}

	\vfill

	\begin{center}
		\large{Konzept:}\\
		\Huge{
			Drahtlose Low Power Kommunikationsprotokolle\\
			und ihre Sicherheit
		}\\[0.5em]
		\large{Wintersemester 2021}\\[0.25em]
		\large{Abgabetermin \today}
	\end{center}

	\vfill

	\begin{flushright}
		Friedemann Richard Pruß (Matrikel-Nr. 20215742)\\
		Mara Schulke (Matrikel-Nr. 20215853)
	\end{flushright}
\end{titlepage}

\newpage

\tableofcontents

\section{Thematische Fokussierung}

In einer vernetzten Welt mit unzähligen IoT-Geräten wird sichere Kommunikation
mit sogenannten Low-Power-Protokollen immer wichtiger. Smart-Devices wie zum
Beispiel Sicherheitskameras, Smart-Lampen oder allgemeiner – jegliche
Embedded-Hardware, lässt sich mittlerweile so gut wie überall wiederfinden.

Diese Geräte zeichnen sich dadurch aus, dass sie oft stark begrenzte Ressourcen
haben. In der Regel ist ein niedriger Stromverbrauch gewünscht, da viele Geräte
mit Akku- oder Batterie betrieben werden. Bei diesen ist folglich die
Rechenleistung eingeschränkt, da mit einer leistungsstarken CPU die
Batterielaufzeit enorm verkürzt würde.

Da Kommunikation dennoch meist eine elementare Aufgabe dieser Geräte ist, gibt
es zahlreiche Entwicklungen, um die Problematik energieeffizient zu lösen.
Low-Power-Protokolle benötigen weniger Strom als vergleichbare Protokolle aus
dem Desktop-Umfeld. Weiterhin ist die Vertraulichkeit \& Sicherheit der
kommunizierten Daten unabdinglich, da gerade im Smart-Home-Bereich ein Angriff
fatale Folgen haben könnte.

Die entstehende Arbeit soll in dem oben genannten Kontext das Thema der
sicheren Kommunikation genauer beleuchten. Dazu werden zwei der am weitesten
verbreiteten Protokolle, namentlich Thread und Bluetooth-Low-Energy (kurz
\textit{BLE}) hinsichtlich ihrer Sicherheit verglichen. Insbesondere ist die
Frage interessant, ob das jüngere Thread-Protokoll die bekannten
sicherheitsrelevanten Problematiken von Bluetooth-Low-Energy behebt.

\section{Motivation \& Abgrenzung}

Auch nach ausführlicher Recherche ließ sich, trotz der bereits großen und immer
weiter zunehmenden Bekanntheit der beiden Protokolle, keine Publikation finden,
in der ein detaillierter Sicherheitsvergleich stattgefunden hat. In der Regel
wurde eines der beiden Protokolle alleinstehend betrachtet
\parencite{ThreadInteroperabilityIEEE, BluetoothMeshIntroBLTW:online}, oder es
fand ein allgemeiner konzeptioneller Vergleich statt – zum Beispiel
hinsichtlich Architektur, Protkoll-Aufbau, Funktionsumfang etc.
\parencite{ThreadMeshVsOtherWirelessIEEE, ComparativeAnalysisIEEE,
ThreadVsBluetoothEnterpriseIotInsights:online}. Desweiteren sind die
Protokollspezifikationen für beide Protokolle öffentlich einsehbar
\parencite{ThreadSpec:online, BluetoothSpec:online}.

\textit{Antonioli, Tippenhauer und Rasmussen} haben im Jahr 2020 bereits eine
sehr detaillierte Arbeit über die Sicherheitsprobleme und Angriffsmöglichkeiten
auf Bluetooth-Low-Energy veröffentlicht.
\parencite{BluetoothLowEnergyAttackOxford}

In verschiedensten Arbeiten werden diverse Anwendungsszenarien erläutert und
die Vorteile der jeweiligen Architektur für die betrachtete Problemstellung
herausgestellt. \parencite{ThreadApplicationIEEE,
ThreadApplicationSmartBuildingsIEEE}

Die geplante Arbeit grenzt sich durch die Fokussierung auf den
Sicherheitsaspekt von den oben genannten Arbeiten ab. Im besten Fall soll diese
eine konkrete Empfehlung für oder gegen die Benutzung eines der Protokolle
geben – eventuell auch einen Ausblick über eine mögliche zukünftige
Entwicklung. Dafür werden mögliche Sicherheitslücken und -mechanismen von
Thread und Bluetooth-Low-Energy untersucht. Insbesondere werden sich diese
Untersuchungen auf mögliche Angriffe und die Mechanismen der Protokolle, um
sich vor eben diesen zu schützen, beziehen.
\parencite{BluetoothPracticalAttacksICACCS, ThreadSecurityCSIAC:online} Zu
typischen Angriffen zählen beispielsweise Man-In-The-Middle-Attacken.
\parencite{GeneralManInTheMiddle}

Andere Arbeiten stellen Eigenschaften und Problematiken, teils durch größer
angelegte Feldversuche, heraus. Anhand dieser experimentellen Untersuchungen
kann man Chancen und Risiken der verglichenen Protokolle gegeneinander abwägen.
Daraus soll die angestrebte Empfehlung abgeleitet werden.

\section{Recherchebericht}

Um einen Einstieg in das Thema zu finden, wurde graue und nicht-zitierfähige
Literatur gesichtet, wie zum Beispiel Blogposts, Videos und Wikipedia-Artikel.
Dadurch konnte ein grober Überblick über die zu bearbeitende Thematik –
\textit{Sicherheit im Smart-Home} – gewonnen werden.

Es wurde deutlich, dass momentan ein Umbruch der Kommunikationsprotokolle im
IoT-Bereich stattfindet. Da dies große Relevanz für die Zukunft der Branche
hat, lag es nahe, das Thema Sicherheit mit den am weitest verbreiteten
Kommunikationsprotokollen zu verbinden. Dies führte letztendlich zu dem Thema
der Arbeit: \textit{\titlecontent}.

% Mindmap + Notizen oder so.

% Thread vs Bluetooth -> wieso? Thread sehr neu, Bluetooth sehr bekannt und
% arbeit zu bluetooth.

Nachdem die Themenfindung abgeschlossen war, begann die fokussierte Recherche
bei den großen wissenschaftlichen Datenbanken. Dabei zeigten sich die
Datenbanken der IEEE, ACM und die Gesellschaft für Informatik besonders
ergiebig für das gewählte Thema.  Zuerst wurden die Publikationen nach
Schlagwörtern, Titeln und Aktualität vorgefiltert. Nachdem eine grobe
Vorauswahl bestand, hat sich das Überfliegen des Abstracts und der
Zusammenfassung als effiziente Möglichkeit zur Überprüfung der Relevanz einer
Arbeit erwiesen. 

Dadurch fanden sich besonders interessante und relevante Quellen wie zum
Beispiel \textcite{BluetoothLowEnergyAttackOxford} oder
\textcite{ThreadMeshVsOtherWirelessIEEE}. Diese und weitere sehr
vielversprechende Arbeiten boten einen guten Anhaltspunkt um über deren
Literaturliste weitere themennahe Publikationen zu finden. Nach der ersten
Kollektion von Quellen, war es ein sinnvoller Schritt diese nach Kategorie (zum
Beispiel \textit{Bluetooth}, \textit{Thread} und \textit{Vergleiche}) und
Relevanz zu sortieren. Priorisiert wurde nach Aktualität, thematischer Nähe und
wissenschaftlicher Wertigkeit.

Jede relevante Publikation wurde in \BibTeX\ erfasst. Nachdem eine erste Liste
vorhanden war, lag es nahe, diese erneut auf Vollständigkeit und Verfügbarkeit –
insbesondere bei online Quellen – zu überprüfen. Dieses Vorgehen wurde so lange
wiederholt, bis sich für den Umfang der zu erstellenden Arbeit genügend
hochwertige Quellen ergeben haben.

%Bei der suche nach bestimmten werden ähnliche vorgeschlagen. Nach relevanz und
%zitathäufgigkeit filtern. 

\newpage

\section{Zeitplan}

\begin{enumerate}
	\item[KW43]    – Recherche \& mediale vorbereitung Präsentation
	\item[KW44]    – Recherche \& Präsentation
	\item[KW45-48] – Inhaltliche Ausarbeitung
	\item[KW49]    – Schreiben des Extended Abstract
	\item[KW50-02] – Entwicklung des wissenschaftlichen Posters
\end{enumerate}

\section{Vorläufige Literaturliste}
\printbibliography[heading=none]
\nocite{*}

%Infos                         | Quelle
%Bluetooth ist unsicher weil.. | Quelle X

%Argumentation
%- Sicherheitslücke X gefunden in Jahr Y von Z
%- Thread bietet neuerung für Angriff 

\end{document}
