\documentclass{article}

\usepackage[a4paper]{geometry}
\usepackage[ngerman]{babel}
\usepackage[utf8]{inputenc}
\usepackage[T1]{fontenc}
\usepackage[table]{xcolor}
\usepackage{enumerate}


\renewcommand\thesubsection{\alph{subsection})}

\begin{document}

\begin{titlepage}
	\begin{flushleft}
		TH Brandenburg \\
		Online Studiengang Medieninformatik \\
		Fachbereich Informatik und Medien \\
		Kommunikationsnetze 1 \\
		Prof. Dr.-Ing. habil. Michael Syrjakow
	\end{flushleft}

	\vfill

	\begin{center}
		\Large{Einsendeaufgabe 1}\\[0.5em]
		\large{Sommersemester 2021}\\[0.25em]
		\large{Abgabetermin 18.04.2021}
	\end{center}

	\vfill

	\begin{flushright}
		Maximilian Schulke \\
		Matrikel-Nr. 20215853
	\end{flushright}
\end{titlepage}

\newpage

\section{Datensicherungsschicht}

\subsection{Bestimmen Sie die Quer- und Längsparität (gerade Parität) bei dem zum übertragenden Datenblock}

\begin{center}
	\begin{tabular}{|c|c|c|c|c|c|c|c|c|}
		\hline
		0                    & 1                    & 1                    & 0                    & 1                    & 1                    & 0                    & 0                    & \cellcolor{gray!25}0 \\
		\hline
		1                    & 1                    & 0                    & 0                    & 0                    & 1                    & 1                    & 0                    & \cellcolor{gray!25}0 \\
		\hline
		0                    & 0                    & 0                    & 1                    & 1                    & 1                    & 0                    & 0                    & \cellcolor{gray!25}1 \\
		\hline
		\cellcolor{gray!25}1 & \cellcolor{gray!25}0 & \cellcolor{gray!25}1 & \cellcolor{gray!25}1 & \cellcolor{gray!25}0 & \cellcolor{gray!25}1 & \cellcolor{gray!25}1 & \cellcolor{gray!25}0 & \cellcolor{gray!25}1 \\
		\hline
	\end{tabular}
\end{center}

\subsection{An die zu übertragende Bitfolge \texttt{1100011011} (Nutzdaten) soll zur Fehlersicherung ein 4 Bit langer Kontrollblock (Frame Check Sequence) angehängt werden. Das Generatorpolynom sei \texttt{10011}. Berechnen Sie den Kontrollblock mittels Polynomdivision.}

\begin{verbatim}
    1 1 0 0 0 1 1 0 1 1 0 0 0 0
    1 0 0 1 1
    ---------
    0 1 0 1 1 1
      1 0 0 1 1
      ---------
      0 0 1 0 0 1 0
          1 0 0 1 1
          ---------
          0 0 0 0 1 1 1 0 0
                  1 0 0 1 1
                  ---------
                  0 1 1 1 1 0
                    1 0 0 1 1
                    ---------
                    0 1 1 0 1 0
                      1 0 0 1 1
                      ---------
                      0 1 0 0 1
\end{verbatim}

Dadurch ergibt sich der Kontrollblock \texttt{1001}

\subsection{Wie groß ist die MTU bei Ethernet?}

Bei Ethernet II 1500 Byte und bei IEEE 802.3 1492 Byte.

\subsection{Welches Vielfachzugriffsverfahren wird von Ethernet benutzt?}

\emph{Carrier Sense Multiple Access/Collision Detection} (kurz \emph{CSMCA/CD})

\section{Internet-Protokoll}

\subsection{Wie wird das Endlos-Kreisen von IP-Paketen im Netz verhindert?}

Durch die Header TTL bei IPv4 bzw. Hop-Limit bei IPv6 werden die maximale Anzahl der Weiterleitungen über Router festgelegt.
Er wird bei jeder Weiterleitungen durch einen Router um 1 verringert – erreicht er den Wert 0 wird das betroffene IP-Paket verworfen.

\subsection{Welche verschiedenen Klassen von IP-Adressen gibt es?}

Bis 1993 wurden IP-Adressen in 5 Klassen eingeteilt:

\begin{center}
	\begin{tabular}{|l|l|l|l|l|}
		\hline
		\textbf{Klasse} & \textbf{Beginnt mit} & \textbf{Subnetzmaske}  & \textbf{Netzeanzahl} & \textbf{Hostanzahl} \\
		\hline
		A               & 0-128            & \texttt{255.0.0.0}     & $2^7$                & $2^{24} - 2$        \\
		\hline
		B               & 129-191          & \texttt{255.255.0.0}   & $2^14$               & $2^{16} - 2$        \\
		\hline
		C               & 192-223          & \texttt{255.255.255.0} & $2^21$               & $2^{8} - 2$         \\
		\hline
		D               & 224-239          & Nicht vorhanden        &                      &                     \\
		\hline
		E               & 240-255          & Nicht vorhanden        &                      &                     \\
		\hline
	\end{tabular}
\end{center}

\subsection{Wie viele Hosts können in einem Class C Netz maximal installiert werden?}

$2^8 - 2 = 254$

\subsection{Zu welcher Klasse gehört die IP-Adresse 129.3.1.13? Ist das eine Netz oder Host-Adresse?}

Da diese Adresse mit 129 beginnt, gehört sie zur Klasse B und hat die Subnetzmasken 255.255.0.0.
Dementsprechend ist der für diese Frage relevante Teil \texttt{1.13}. In diesen beiden Bytes sind
weder alle Bits auf 0 noch alle Bits auf 1, deshalb handelt es sich um eine Host-Adresse.

\subsection{Kann die IP-Adresse 192.168.128.4 im Internet vorkommen}

Nein, Adressen die mit \texttt{192.168} Anfangen sind immer privat.

\subsection{Ist 192.256.20.132 eine gültige IP-Adresse?}

Nein \texttt{256} kann man nicht in einem Byte codieren.

\subsection{Was bedeutet Subnetting?}

Durch Subnetting lassen sich größere Netze in kleinere aufteilen, in dem der Host-Anteil eines zugewiesenen IP-Adressbereichs
noch weiter in Subnetz-ID und Host-ID unterteilt wird. Somit können Adressbereiche effizienter ausgenutzt werden.

\subsection{Sie benötigen im Netz 198.200.40.0 mehrere Subnetze, wobei in jedem Subnetz bis maximal 13 Hosts installiert werden sollen. Definieren Sie alle möglichen Subnetzmasken}

\begin{center}
	\begin{tabular}{|l|l|l|}
		\hline
		\textbf{Subnetzmaske}    & \textbf{Subnetze} & \textbf{Hosts} \\
		\hline
		\texttt{255.255.255.240} & $2^4$             & $2^4 - 2$   \\
		\hline
		\texttt{255.255.255.224} & $2^3$             & $2^5 - 2$   \\
		\hline
		\texttt{255.255.255.192} & $2^2$             & $2^6 - 2$    \\
		\hline
		\texttt{255.255.255.128} & $2^1$             & $2^7 - 2$    \\
		\hline
	\end{tabular}
\end{center}

\end{document}
