\documentclass[journal]{IEEEtran}

\usepackage[a4paper]{geometry}\usepackage[ngerman]{babel}
\usepackage[utf8]{inputenc}
\usepackage[T1]{fontenc}
\usepackage{listings}
\usepackage{hyperref}
\usepackage{graphicx}
\usepackage{xcolor}

\pagenumbering{Roman}

\begin{document}

\markboth{Ethik in der IT Sicherheit – Mara Schulke — 20215853}{}

\begin{onecolumn}

\section{Das Studienmodul im Überblick}

Ethik und IT-Sicherheit als gegensätzliche Disziplinen
\\
Parallelen zwischen Ethik und IT-Sicherheit
\\
Schutzgut in der IT-Sicherheit
\\
Informationelle Selbstbestimmung als Schutzgut
\\
Kontextabhängigkeit des Rechts auf informationelle Selbstbestimmung
\\
Definition und Charakteristik von Ethik
\\
Reflexion der Moral als Bestandteil der Ethik
\\
Komplexität ethischer Fragen aufgrund unterschiedlicher Wertevorstellungen
\\
Verantwortung bei komplexen und umfangreichen digitalen Themen
\\
Veränderungspotential durch Digitalisierung
\\
Arten von Hackern und ethisches Dilemma
\\
Einführung in das Modul und Lernziele

\newpage

Überblick und Lernziele BBA (Begriffliche Bestimmung und Abgrenzung)
\\

\newpage

Erklärung und Abgrenzung von Grundbegriffen der Ethik
\\
Unterschied zwischen Moral und Ethik
\\
Erklärung des Berufsethos und seine Bedeutung für eine Gemeinschaft
\\
Hinterfragen bestehender Wertesysteme unter Einbezug eines moralischen Konflikts (moralische Kompetenz)
\\
Ethik beschäftigt sich mit der Suche nach einer besseren Moral
\\
Diskussion von Relativismus in der Ethik
\\
Verbindung von Recht und Moral in Bezug auf die Informationstechnologie
\\
Übungsaufgaben, Zusammenfassung, Wissensüberprüfung und Literaturangaben als Teil der Lerneinheit.


\end{onecolumn}
\end{document}