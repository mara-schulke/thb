\documentclass[journal]{IEEEtran}

\usepackage[a4paper,top=5cm,bottom=5cm]{geometry}
\usepackage[ngerman]{babel}
\usepackage[utf8]{inputenc}
\usepackage[T1]{fontenc}
\usepackage{listings}
\usepackage{hyperref}
\usepackage{graphicx}
\usepackage{xcolor}

\pagenumbering{Roman}

\begin{document}

\markboth{Ethik in der IT Sicherheit – Mara Schulke — 20215853}{}

\begin{onecolumn}

\section{Das Studienmodul im Überblick}

In der ersten Einheit dieses Kurses werden grundlegende Gegebenheiten aufgezeigt wie beispielsweise
die Gegenüberstellung der Disziplinen IT Sicherheit und Ethik und deren inhärenter Widerspruch.
Ein wichtiger elementarer Unterschied zwischen diesen beiden Disziplin ist die technologische Natur
der IT Sicherheit und die geisteswissenschaftliche der Ethik. In beiden Disziplinen spielt allerdings
die Verantwortung eine zentrale Rolle – die Verantwortung in der IT Sicherheit auf Nachweisbarkeit, 
informationelle Selbstbestimmung und die Wahrung der Vertraulichkeit und in der Ethik die Verantwortung
im Bezug auf getroffene unter Umständen risikobehaftete Entscheidungen. 
Die Digitalisierung im Allgemeinen und ihr Einfluss auf die Gesellschaft hinsichtlich ihrer Möglichkeiten
zukünftigen Entwicklung etc. kommt mit einer großen Ungewissheit einher und wirft viele 
verantwortungsbehaftete Fragestellungen auf die eine Schnittmenge der IT Sicherheit und Ethik darstellen. 
Ein weiterer zentraler Aspekt der Lerneinheit befasst sich mit der Cybersecurity und den damit 
einhergehenden ethischen Fragestellungen bezogen auf die Privatsphäre und Überwachung.

\section*{Fragestellung I}

Im Kapitel wurde das Recht auf informationelle Selbstbestimmung immer in einem bestimmten Kontext 
betrachtet – eine interessante Fragestellung ist hier, welche Ausnahmen es von diesem Recht gibt – und 
anhand welcher Regeln Situationen als Ausnahme bewertet werden können.

\section{Überblick und Lernziele BBA (Begriffliche Bestimmung und Abgrenzung)}

Ein Kernbestandteil dieser Lerneinheit stellt der ``begriffliche Werkzeugkoffer'' dar - der zur akkuraten 
Beschreibung von Situationen und Fragestellungen dienen soll und eine Bewertung dieser erleichtert.
Somit können zuverlässiger möglichst belastbare ethische Entscheidungen zustande kommen.

Das Wertesystem einer Gesellschaft lässt sich als sinngebendes Leitbild verstehen, anhand dessen 
Entscheidungen getroffen werden, können und das eine gemeinsame Basis innerhalb einer Gruppe darstellt. 
Wertesysteme beziehungsweise Moralvorstellungen sind flexible Konzepte, die sich von Teilgemeinde zu 
Teilgemeinde innerhalb einer Gesellschaft unterscheiden oder teilweise auch widersprechen können. 
Bestimmte Berufsgruppen, die sich über mehrere Gesellschaften hinweg erstrecken, können ebenfalls eigene 
abgewandelte Wertesysteme - einen sogenannten Berufsethos - entwickeln, der ein Leitbild für Berufstätige 
in dieser Gruppe darstellen kann. Die sogenannte moralische Kompetenz tritt auf, wenn spezielle
individuelle Werte im Konflikt mit dem geltenden Wertesystem stehen.

Der Relativismus führt bestimmte moralische Überzeugungen auf soziale, kulturelle und historische
Gegebenheiten zurück und setzt sie somit in Verhältnis und gibt ihnen Kontext.
In der Informationstechnologie müssen bestimmte Rechtsnormen verbindlich umgesetzt werden um eine 
straffreie Implementierung von moralisch fragwürdigen Problemen voranzutreiben. Diese Vorgabe ist
losgelöst von den spezifischen Wertesystemen der Person, die diese Implementierung erstellt um eine
gesamtgesellschaftlich-einheitliche Vorgehensweise zu etablieren.

\section*{Fragestellung II}

Im Bezug auf Wertvorstellungen wird sofort der Einsatz von künstlicher Intelligenz in kritischen 
Bereichen präsent. Anhand welcher Regeln und Leitsätze sollen sich selbstdenkende beziehungsweise 
kognitive Systeme verhalten und welche Entscheidungen dürfen, sie überhaupt treffen? Ist es überhaupt 
möglich gewisse ethisch extrem fordernde Fragestellungen wie z. B. die Triage in der Medizin oder den
Abschuss eines Ziels im Militärkontext zu automatisieren?


\end{onecolumn}

\end{document}