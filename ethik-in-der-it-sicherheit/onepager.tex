\documentclass[journal]{IEEEtran}

\usepackage[a4paper]{geometry}\usepackage[ngerman]{babel}
\usepackage[utf8]{inputenc}
\usepackage[T1]{fontenc}
\usepackage{listings}
\usepackage{hyperref}
\usepackage{graphicx}
\usepackage{xcolor}

\pagenumbering{Roman}

\begin{document}

\markboth{Ethik in der IT Sicherheit – Mara Schulke — 20215853}{}

\begin{onecolumn}

\section{Das Studienmodul im Überblick}

In dem ersten Studienmodul dieses Kurses werden grundlegende Fakten aufgezeigt, wie beispielsweise
die gegenüberstellung der Disziplinen IT Sicherheit und Ethik und deren inherenter Widerspruch.
Ein wichtiger elementarer Unterschied zwischen diesen beiden Disziplin ist die technologische Natur
der IT Sicherheit und die geisteswissenschaftliche der Ethik. In beiden Disziplinen spielt allerdings
die Verantwortung eine zentrale Rolle -  die Verantwortung in der IT Sicherheit auf Nachweisbarkeit,
informationelle Selbstbestimmung und die Wahrung der Vertraulichkeit und in der Ethik die Verantwortung
im bezug auf getroffene unter Umständen risikobehaftete Entscheidungen. 

Die Digitalisierung im allgemeinen und ihr Einfluss auf die Gesellschaft hinsichtlich ihrer Möglichkeiten,
zukünftigen Entwicklung etc. kommt mit einer großen Ungewissheit einher und wirft viele 
verantwortungsbehaftete Fragestellungen auf die eine Schnittmenge der IT Sicherheit und Ethik darstellen.   


--------------------------


 Vor dem Hintergrund der Digitalisierung, deren technologische Möglichkeiten die Gesellschaft (national wie global) vor unbekannte und neuartige Herausforderungen stellt, potenziert sich Verantwortungsfrage.




Das Schutzgut in der IT-Sicherheit ist die informationelle Selbstbestimmung, die das Recht des Einzelnen darstellt, über seine personenbezogenen Daten zu bestimmen.

 Die Digitalisierung hat Auswirkungen auf alle Lebensbereiche und stellt neue ethische Fragen, die die IT-Sicherheit beeinflussen. Die Cybersecurity stellt Fragen zur Privatsphäre und Überwachung und fordert sowohl technische als auch ethische Kompetenzen der Akteure heraus. Verantwortung spielt eine zentrale Rolle bei den Herausforderungen der Digitalisierung.


%Ethik und IT-Sicherheit als gegensätzliche Disziplinen
\\
%Parallelen zwischen Ethik und IT-Sicherheit
\\
%Schutzgut in der IT-Sicherheit
\\
%Informationelle Selbstbestimmung als Schutzgut
\\
%Kontextabhängigkeit des Rechts auf informationelle Selbstbestimmung
\\
%Definition und Charakteristik von Ethik
\\
%Reflexion der Moral als Bestandteil der Ethik
\\
%Komplexität ethischer Fragen aufgrund unterschiedlicher Wertevorstellungen
\\
%Verantwortung bei komplexen und umfangreichen digitalen Themen
\\
%Veränderungspotential durch Digitalisierung
\\
%Arten von Hackern und ethisches Dilemma
\\
%Einführung in das Modul und Lernziele

\section{Überblick und Lernziele BBA (Begriffliche Bestimmung und Abgrenzung)}
\\

Erklärung und Abgrenzung von Grundbegriffen der Ethik
\\
Unterschied zwischen Moral und Ethik
\\
Erklärung des Berufsethos und seine Bedeutung für eine Gemeinschaft
\\
Hinterfragen bestehender Wertesysteme unter Einbezug eines moralischen Konflikts (moralische Kompetenz)
\\
Ethik beschäftigt sich mit der Suche nach einer besseren Moral
\\
Diskussion von Relativismus in der Ethik
\\
Verbindung von Recht und Moral in Bezug auf die Informationstechnologie
\\
Übungsaufgaben, Zusammenfassung, Wissensüberprüfung und Literaturangaben als Teil der Lerneinheit.


\end{onecolumn}
\end{document}+