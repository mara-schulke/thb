\documentclass[journal]{IEEEtran}

\usepackage[a4paper,top=5cm,bottom=5cm]{geometry}
\usepackage[ngerman]{babel}
\usepackage[utf8]{inputenc}
\usepackage[T1]{fontenc}
\usepackage{listings}
\usepackage{hyperref}
\usepackage{graphicx}
\usepackage{xcolor}

\pagenumbering{Roman}

\begin{document}

\markboth{Ethik in der IT Sicherheit – Mara Schulke — 20215853}{}

\begin{onecolumn}

\section{Das Studienmodul im Überblick}

In der ersten Einheit dieses Kurses werden grundlegende Gegebenheiten aufgezeigt wie beispielsweise
die Gegenüberstellung der Disziplinen IT Sicherheit und Ethik und deren inhärenter Widerspruch.
Ein wichtiger elementarer Unterschied zwischen diesen beiden Disziplin ist die technologische Natur
der IT Sicherheit und die geisteswissenschaftliche der Ethik. In beiden Disziplinen spielt allerdings
die Verantwortung eine zentrale Rolle – die Verantwortung in der IT Sicherheit auf Nachweisbarkeit, 
informationelle Selbstbestimmung und die Wahrung der Vertraulichkeit und in der Ethik die Verantwortung
im Bezug auf getroffene unter Umständen risikobehaftete Entscheidungen. 
Die Digitalisierung im Allgemeinen und ihr Einfluss auf die Gesellschaft hinsichtlich ihrer Möglichkeiten
zukünftigen Entwicklung etc. kommt mit einer großen Ungewissheit einher und wirft viele 
verantwortungsbehaftete Fragestellungen auf die eine Schnittmenge der IT Sicherheit und Ethik darstellen. 
Ein weiterer zentraler Aspekt der Lerneinheit befasst sich mit der Cybersecurity und den damit 
einhergehenden ethischen Fragestellungen bezogen auf die Privatsphäre und Überwachung.

\section*{Fragestellung I}

Im Kapitel wird das Recht auf informationelle Selbstbestimmung immer in einem bestimmten Kontext 
betrachtet – eine interessante Fragestellung ist hier, welche Ausnahmen es von diesem Recht gibt – und 
anhand welcher Regeln Situationen als Ausnahme bewertet werden können.

\section{Überblick und Lernziele BBA (Begriffliche Bestimmung und Abgrenzung)}

Ein Kernbestandteil dieser Lerneinheit stellt der ``begriffliche Werkzeugkoffer'' dar - der zur akkuraten 
Beschreibung von Situationen und Fragestellungen dienen soll und eine Bewertung dieser erleichtert.
Somit können zuverlässiger möglichst belastbare ethische Entscheidungen zustande kommen.

Das Wertesystem einer Gesellschaft lässt sich als sinngebendes Leitbild verstehen, anhand dessen 
Entscheidungen getroffen werden können und das eine gemeinsame Basis innerhalb einer Gruppe darstellt. 
Wertesysteme beziehungsweise Moralvorstellungen sind flexible Konzepte, die sich von Teilgemeinde zu 
Teilgemeinde innerhalb einer Gesellschaft unterscheiden oder teilweise auch widersprechen können. 
Bestimmte Berufsgruppen, die sich über mehrere Gesellschaften hinweg erstrecken, können ebenfalls eigene 
abgewandelte Wertesysteme - einen sogenannten Berufsethos - entwickeln, der ein Leitbild für Berufstätige 
in dieser Gruppe darstellen kann. Die sogenannte moralische Kompetenz tritt auf, wenn spezielle
individuelle Werte im Konflikt mit dem geltenden Wertesystem stehen.

Der Relativismus führt bestimmte moralische Überzeugungen auf soziale, kulturelle und historische
Gegebenheiten zurück und setzt sie somit in Verhältnis und gibt ihnen Kontext.
In der Informationstechnologie müssen bestimmte Rechtsnormen verbindlich umgesetzt werden um eine 
straffreie Implementierung von moralisch fragwürdigen Problemen voranzutreiben. Diese Vorgabe ist
losgelöst von den spezifischen Wertesystemen der Person, die diese Implementierung erstellt um eine
gesamtgesellschaftlich-einheitliche Vorgehensweise zu etablieren.

\section*{Fragestellung II}

Im Bezug auf Wertvorstellungen wird sofort der Einsatz von künstlicher Intelligenz in kritischen 
Bereichen präsent. Anhand welcher Regeln und Leitsätze sollen sich selbstdenkende beziehungsweise 
kognitive Systeme verhalten und welche Entscheidungen dürfen, sie überhaupt treffen? Ist es überhaupt 
möglich gewisse ethisch extrem fordernde Fragestellungen wie z. B. die Triage in der Medizin oder den
Abschuss eines Ziels im Militärkontext zu automatisieren?

\newpage

\section{Verschiedene Bezugspunkte von Ethik}

Das Skript beschreibt hier die vier wichtigen Bezugspunkte zur Ethik: die Pflicht, der Diskurs,
der Nutzen und die Tugend. Des Weiteren wird zwischen deskriptiver Ethik (die bereits vorhandene
moralische Werte-Systeme und Vorstellungen beschreibt) und der normativen Ethik die sich mit der
Urteilsfällung und einem Regelsystem befasst. Somit gibt es klare parallelen zwischen der normativen
Ethik und der Anwendung in den Rechtssystemen von demokratisch regierten Staaten.

Im Bezug zur Handlungsnotwendigkeit gibt es die zwei Dimensionen der materiellen und formalen Ethik.
die materielle Ethik arbeitet einen konkreten Handlungsplan heraus währen die formale Ethik sich mit
allgemein anwendbaren aber ebenso gewichteten Leitlinien befasst. Die Goldene Regel ``Was Du nicht willst,
das man Dir tu, das füg auch keinem Andern zu!'' legt mehr Verantwortung in die Hände der 
interpretierenden Person als konkrete Regeln die durch allgemeinen gesellschaftlichen Konsens
definiert wurden. Somit kommt es zusätzlich zu den Gesetzen innerhalb einer Gesellschaft auch auf
die praktische Urteilskraft jeder einzelnen an, da in eine liberalen Demokratie zu komplexe
Herangehensweisen an bestimmte Probleme erfordert um sie in einem allumfassenden Regelwerk

abschließend herunter zu schreiben. 

\section*{Fragestellung III}

Eine interessante Frage die mir im Bezug auf die Gewichtung zwischen der materiellen und formalen Ethik
innerhalb einer Gesellschaft stellt ist der Grad: Bis zu welchem Punkt sollten Gesetze und Richtlinien
verfasst werden und ab welchem Punkt muss die Eigenverantwortung jeder Einzelperson greifen?

\section{Ausdifferenzierung in komplexe Bezugskontexte}

In dem vierten Kapitel des Lehrmaterials wird im Kern auf die Abhängigkeit der ethischen Kompetenz von
den jeweiligen Einsatz-Szenarien eingegangen. Dies wird erreicht in dem zu erst zwei verschiedene
Definitionen eingeführt werden - die angewandte Ethik und die Bereichsethik:

--- Die angewandte Ethik stellt eine komplexe Herausforderung da die die theoretischen Aspekte einer
ethischen Fragestellung auf ein praktisches Problem anwendet um somit zu einem ethisch korrekten 
Handlungsplan zu gelangen. Die Konzepte die in einer konkreten Problemsituation angewendet werden
formen eine ethische Position zu einem Thema - dies kann auch im Rahmen von Politik stattfinden um die
Meinung und das Wertesystem der breiten Masse einer Gesellschaft zu beeinflussen. Ein im Skript benanntes
Beispiel das diese Funktion erfüllt ist der Deutsche Ethikrat.

--- Die Bereichsethik wird als Teil der angewandten Ethik definiert in dessen sich mit moralischen Fragen
im Kontext von spezifischen Bereichen auseinandergesetzt wird. Das Kapitel nennt als Beispiele die Bio-,
Medizin-, Wirtschafts-, Umwelt- und IT-Ethik als Beispiele. Die Notwendigkeit für die Bereichsethik ergibt
sich aus der limitierten Anwendbarkeit von allgemeinen ethischen Leitsätzen in konkreten bereichsspezifischen 
Problemstellungen (wie zum Beispiel der Triage in der Medizin). 

Ein gutes Beispiel für die differenzierte Betrachtung stellt die Corona-Krise dar, die starke Auswirkungen über
viele Lebensbereiche hinweg hatte und die keine einfache ethische Betrachtung erlaubte. Die Problematik erstreckte
sich über viele verschiedene Bereiche wie Medizin, Wirtschaft, Sozialwissenschaften etc. für die Wissen und
Ethikkenntnisse aus diesen Bereichen notwendig ist um eine Angemessene Entscheidung zu treffen.

\section*{Fragestellung IV}

Wie sollten ethische Probleme gehandhabt werden bei denen ethische Grundsätze aus einzelnen Bereichen sich
widersprechen und einen Konflikt darstellen? Anders formuliert: gibt es eine implizite Hierarchie der
Bereichsethiken?

\newpage

\section{Relevanz des Begriffs der Verantwortung}

Die Verantwortung ist ein dialogisches Grundprinzip – also eines das mehrere ``Akteure'' benötigt um
Anwendung zu finden. Es ist eine explizite oder implizite Vereinbarung einem moralischen / ethischen 
Anspruch innerhalb einer Beziehung gerecht zu werden – dies kann sich in verschiedenen Formen darstellen
und erstreckt sich über zwischenmenschliche Beziehungen hinaus. Das Eintreten eines Ereignisses, dass der
zugrundeliegenden Vereinbarung der Verantwortung widerspricht, bringt eine Schuld und die Vereinbarung der
Übernahme von eventuellen Konsequenzen mit sich. Ein einfaches Beispiel stellt hier der bekannte Spruch
"Eltern haften für ihre Kinder" dar - die Verantwortung in der Beziehungskonstellation liegt hier bei den
Eltern und bei einer Vernachlässigung der Sorgfaltspflicht gehen eventuelle entstandene Schäden auf diese
über. In zwischenmenschlichen Beziehungen ist allerdings die Suche nach dem ``Schuldigen'' oftmals einfacher
als in rein digitalen Kontexten in denen eine hohe Anonymisierung herrscht. Eine große Aufgabe der IT-Sicherheit
bzw. IT-Forensik ist die Nachvollziehbarkeit von Ereignissen und die Beweis-Erbringung. Im Kontext von
ethischen Diskussionen steht oft die Verantwortung im Vergleich zur Schuld im Vordergrund.

\section*{Fragestellung V}

Wie kann im digitalen Kontext eine klare Verantwortungsverteilung etabliert werden, um im Falle von Verstößen
eine angemessene Schuldzuweisung vornehmen zu können und eventuelle Konsequenzen zu übernehmen? Wann ist ein
Betreiber der nicht ausreichende Daten über Handlungen durch sein Produkt speichert haftbar für Handlungen dritter
die dadurch verschleiert werden konnten?  

\section{Moderne Herausforderungen an die Verantwortung}

\section*{Fragestellung VI}

\end{onecolumn}

\end{document}