\documentclass[journal]{IEEEtran}

\usepackage[a4paper,top=5cm,bottom=5cm]{geometry}
\usepackage[ngerman]{babel}
\usepackage[utf8]{inputenc}
\usepackage[T1]{fontenc}
\usepackage{listings}
\usepackage{hyperref}
\usepackage{graphicx}
\usepackage{xcolor}

\pagenumbering{Roman}

\begin{document}

\markboth{Ethik in der IT Sicherheit – Mara Schulke — 20215853}{}

\begin{onecolumn}

\begin{center}
	\Large{Geistiges Eigentum im Zeitalter der Informationstechnik}\\
	\vspace{18pt}\hline
\end{center}
    
\section*{Einleitung}

Das geistige Eigentum und die Sicherstellung von Eigentümerrechten ist von
elementarer Bedeutung für unsere Gesellschaft und ihre Ordnung. Die Werke einzelner 
Individuen oder ganzer Organisationen zu schützen ist eine fundamentale ethische 
Verpflichtung der die Gesellschaft während der voranschreitenden Digitalisierung 
gerecht werden muss. Der technologische Fortschritt darf nicht zur Verwahrlosung des 
Eigentumsrechts führen, lässt sich aber gleichzeitig deutlich schwerer Kontrollieren 
als physisches Eigentum. Es ist ein komplexer Balance-Akt zwischen dem Schutz von 
einzelnen gegenüber der Mehrheit. 

Mit der Betrachtung von Schutzregelungen zum Eigentum werfen sich interessante 
Fragestellungen auf, wie z.B. dürfen oder müssen in gewissen Situationen Ausnahmen 
von diesem Schutz gemacht werden, wenn das dem Gemeinwohl der Gesellschaft dienlich 
sind?

/// IMPFSTOFF DEBATTE

\section*{Einfluss der Informationstechnik auf digitales Eigentum}

Die Digitalisierung hat einen erheblichen Einfluss auf unseren Umgang 
mit geistigen und digitalen Eigentum. 
Der technologische Fortschritt hat in unserer Gesellschaft zu einer weitreichenden
Digitalisierung von Waren geführt – viele Gegenstände die früher 
nur in physischer Form dementsprechend nur begrenzt erhältlich waren, sind nun 
jederzeit auf Abruf verfügbar und lassen sich unbegrenzt oft vervielfältigen. 
Hierdurch sind ganze Teilbereiche der Kriminalität im Bezug auf den digitalen Bereich 
entstanden, wie zum Beispiel die Piraterie oder die illegale Verbreitung von
lizensierten / gekauften Inhalten.

Diese Veränderungen werfen neue Fragen und Herausforderungen auf wie zum Beispiel 
die Notwendigkeit für einen Kopierschutz, Lizenzierungen von Software etc.
Mit diesen neuen Herausforderung bring der Fortschritt allerdings auch 
neue Möglichkeiten zur Sicherung des Eigentums in der digitalen Welt mit sich.
Konzepte wie der Einsatz von Kryptographie und Plagiatserkennungs-Algorithmen 
eröffnen Möglichkeiten, die Integrität und Nachvollziehbarkeit des geistigen und 
digitalen Eigentums sicherzustellen.
Es ist wichtig diese Probleme und Herausforderungen ernst zunehmen um die Wahrung von 
Ideen und geistigem Eigentum heute und künftig sicherzustellen. 

\section*{Ethik und Eigentum}

Bei der ethischen Betrachtung dieses Themas stellen sich Fragen im Bezug auf die 
Verantwortung, wie z.B. wer ist für die Sicherstellung der Eigentumsrechte im digitalen 
Raum verantwortlich?

Im Falle von Musik ist ein bekannter Anbieter der Musik-Streaming-Dienst Spotify: Wer ist 
in diesem Fall für Sicherstellung von Urheberrechten verantwortlich und wo ziehen sich 
Verantwortungsbereiche durch diese Thematik? Ist bei der Verhinderung von Raubkopien 
das Unternehmen, dass die Plattform zur Verbreitung bereitstellt, der Gesetzgeber, oder 
das Individuum, das die Plattform zur Verbreitung verwendet gefragt? In der Realität zeigt 
sich, dass die Wahrung des geistigen Eigentums im Verantwortungsbereich von allen drei 
Akteuren liegt. Die Plattform muss die Verletzung feststellen – ein gutes Beispiel ist 
hier YouTube, die die Monetarisierung von urheberrechtlich geschützten Inhalten verbieten. 
Der Gesetzgeber muss einen rechtlichen Rahmen stellen um strafrechtlich gegen die 
Verletzung von Urheberrechten vorzugehen – und jedes Individuum trägt einen Teil 
Eigenverantwortung keine Urheberrechtsverletzung vorzunehmen, und dessen eigene Werke 
so gut es geht davor zu schützen.

Die zugrundeliegende ethische Fragestellung führt auf das gesellschaftliche Wertesystem 
zurück und der Umgang einer jeden Gesellschaft ist abhängig davon, wie sehr Fairness, 
Respekt und Gerechtigkeit in diesem einen Platz finden.

Eine Anerkennung der Rechte der Hersteller und Besitzer stehen der Problematik der 
gerechten Verteilung und dem Grundprinzip der Chancengleichheit gegenüber. Bildet sich
konzentriertes Besitztum innerhalb eine Gesellschaft hat dies ab einem gewissen Grat
einen Einfluss auf die Machtverteilung – sowohl in der physischen als auch in der 
digitalen Welt. Ein gutes Beispiel hierfür sind die Registrare von Domains im Internet, 
oder Firmen die über eine immense Rechenleistung verfügen. Die Entscheidungsposition in
die diese Akteure durch ihr Eigentum geraten bringt ein hohes Maß an Verantwortung mit 
sich, um eine möglichst ausgeglichene und demokratische Gesellschaft beizubehalten.

Die ethische Betrachtung des Themas Eigentum im Zeitalter der Informationstechnik 
erfordert daher eine tiefgreifende Auseinandersetzung mit ethischen Prinzipien wie 
Machtverteilung, Chancengleichheit und Gerechtigkeit in der digitalen Welt.

% Bezüge:
% - Bezugspunkte
% - Verantwortung 
% - Menschen, Organisationen und die Gesellschaft

\section*{Konflikte und Dilemmata im Zusammenhang mit Eigentum und Informationstechnik}
A. Datenschutz und Privatsphäre
B. Geistiges Eigentum und Urheberrechte
C. Zugang zu Information und Wissen

\section*{Auswirkungen auf die Wirtschaft und Gesellschaft}
VI. Auswirkungen auf Wirtschaft und Gesellschaft
A. Neue Geschäftsmodelle und Märkte
B. Soziale und ethische Implikationen
C. Auswirkungen auf den Arbeitsmarkt

VII. Schlussfolgerung
A. Zusammenfassung der wichtigsten Erkenntnisse
B. Ausblick auf zukünftige Entwicklungen und Herausforderungen

\section*{Literaturverzeichnis}
VIII. Literaturverzeichnis


\end{onecolumn}

\end{document}