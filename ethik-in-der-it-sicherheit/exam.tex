\documentclass[journal]{IEEEtran}

\usepackage[a4paper,top=5cm,bottom=5cm]{geometry}
\usepackage[ngerman]{babel}
\usepackage[utf8]{inputenc}
\usepackage[T1]{fontenc}
\usepackage{listings}
\usepackage{hyperref}
\usepackage{graphicx}
\usepackage{xcolor}

\pagenumbering{Roman}

\begin{document}

\markboth{Ethik in der IT Sicherheit – Mara Schulke — 20215853}{}

\begin{onecolumn}

\begin{center}
	\Large{Geistiges Eigentum im Zeitalter der Informationstechnik}\\
	\vspace{18pt}\hline
\end{center}
    
\section*{Einleitung}

Das geistige Eigentum und die Sicherstellung von Eigentümerrechten ist von
elementarer Bedeutung für unsere Gesellschaft und ihre Ordnung. Die Werke einzelner 
Individuen oder ganzer Organisationen zu schützen ist eine fundamentale ethische 
Verpflichtung der die Gesellschaft während der voranschreitenden Digitalisierung 
gerecht werden muss. Der technologische Fortschritt darf nicht zur Verwahrlosung des 
Eigentumsrechts führen, lässt sich aber gleichzeitig deutlich schwerer Kontrollieren 
als physisches Eigentum. Es ist ein komplexer Balance-Akt zwischen dem Schutz von 
einzelnen gegenüber der Mehrheit. 

Mit der Betrachtung von Schutzregelungen zum Eigentum werfen sich interessante 
Fragestellungen auf, wie z.B. dürfen oder müssen in gewissen Situationen Ausnahmen 
von diesem Schutz gemacht werden, wenn das dem Gemeinwohl der Gesellschaft dienlich 
sind?

/// IMPFSTOFF DEBATTE

\section*{Einfluss der Informationstechnik auf digitales Eigentum}

Die Digitalisierung hat einen erheblichen Einfluss auf unseren Umgang 
mit geistigen und digitalen Eigentum. 
Der technologische Fortschritt hat in unserer Gesellschaft zu einer weitreichenden
Digitalisierung von Waren geführt – viele Gegenstände die früher 
nur in physischer Form dementsprechend nur begrenzt erhältlich waren, sind nun 
jederzeit auf Abruf verfügbar und lassen sich unbegrenzt oft vervielfältigen. 
Hierdurch sind ganze Teilbereiche der Kriminalität im Bezug auf den digitalen Bereich 
entstanden, wie zum Beispiel die Piraterie oder die illegale Verbreitung von
lizensierten / gekauften Inhalten.

Diese Veränderungen werfen neue Fragen und Herausforderungen auf wie zum Beispiel 
die Notwendigkeit für einen Kopierschutz, Lizenzierungen von Software etc.
Mit diesen neuen Herausforderung bring der Fortschritt allerdings auch 
neue Möglichkeiten zur Sicherung des Eigentums in der digitalen Welt mit sich.
Konzepte wie der Einsatz von Kryptographie und Plagiatserkennungs-Algorithmen 
eröffnen Möglichkeiten, die Integrität und Nachvollziehbarkeit des geistigen und 
digitalen Eigentums sicherzustellen.
Es ist wichtig diese Probleme und Herausforderungen ernst zunehmen um die Wahrung von 
Ideen und geistigem Eigentum heute und künftig sicherzustellen. 

\section*{Ethik und Eigentum}
// unbearbeitet & quellen suchen!!! \\

Im Zeitalter der Informationstechnik spielen ethische Prinzipien eine zentrale Rolle in Kontext des Eigentums.

Erstens sind grundlegende ethische Prinzipien wie Fairness, Gerechtigkeit und Respekt von großer Bedeutung für den Schutz des Eigentums. Die Anerkennung der Rechte von Schöpfern und Eigentümern, sowie eine gerechte Verteilung der Ressourcen sind dabei maßgeblich. Zweitens erfordert die Ethik der Informationsgesellschaft eine kritische Betrachtung der Machtverhältnisse, die durch den Zugang zu und die Kontrolle über Informationen entstehen. Es ist wichtig, die Auswirkungen von Monopolisierung, Überwachung und Manipulation auf das Eigentum und die damit verbundenen Rechte zu analysieren. Drittens ist die Frage der digitalen Rechte und sozialen Gerechtigkeit von hoher Relevanz. Es besteht die Notwendigkeit sicherzustellen, dass alle Mitglieder der Gesellschaft Zugang zu digitalen Ressourcen haben und ihre Rechte in der digitalen Welt geschützt sind. Eine ethische Perspektive auf das Eigentum im Zeitalter der Informationstechnik erfordert daher eine kritische Auseinandersetzung mit den grundlegenden Prinzipien, der Machtverteilung und der sozialen Gerechtigkeit im digitalen Raum.

\section*{Konflikte und Dilemmata im Zusammenhang mit Eigentum und Informationstechnik}
A. Datenschutz und Privatsphäre
B. Geistiges Eigentum und Urheberrechte
C. Zugang zu Information und Wissen

\section*{Auswirkungen auf die Wirtschaft und Gesellschaft}
VI. Auswirkungen auf Wirtschaft und Gesellschaft
A. Neue Geschäftsmodelle und Märkte
B. Soziale und ethische Implikationen
C. Auswirkungen auf den Arbeitsmarkt

VII. Schlussfolgerung
A. Zusammenfassung der wichtigsten Erkenntnisse
B. Ausblick auf zukünftige Entwicklungen und Herausforderungen

\section*{Literaturverzeichnis}
VIII. Literaturverzeichnis


\end{onecolumn}

\end{document}