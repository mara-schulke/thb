\documentclass{article}

\usepackage[a4paper]{geometry}
\usepackage[ngerman]{babel}
\usepackage[utf8]{inputenc}
\usepackage[T1]{fontenc}
\usepackage{graphicx}
\usepackage{fancyhdr}
\usepackage{xcolor}
\usepackage{float}
\usepackage{hyperref}

\graphicspath{{./images/}}

\pagestyle{fancyplain}
\fancyhf{}
\lhead{\fancyplain{}{Mara Schulke} }
\rhead{\fancyplain{}{\today}}
\cfoot{\fancyplain{}{\thepage}}

\newcommand{\annotation}[1]{
\begin{quote}
	\begin{textit}
		{#1}
	\end{textit}
\end{quote}
}

\renewcommand*{\thesection}{Kapitel \arabic{section}}
\renewcommand*{\thesubsection}{\alph{subsection})}

\begin{document}

\begin{titlepage}
	\begin{center}
		\begin{Large}
			Technische Hochschule Brandenburg \\[1em]
		\end{Large}
		
		IT Sicherheit \\
		Informatik und Medien \\
		Biometrie – Dr.\ Tobias Scheidat
	\end{center}
	
	\vfill

	\begin{center}
		\Large{Übungsaufgaben}\\[0.5em]
		\large{Wintersemester 2024}\\[0.25em]
		\large{Abgabetermin \today}
	\end{center}

	\vfill

	\begin{center}
		Mara Schulke \\ Matr-Nr. 20215853
	\end{center}
\end{titlepage}


\tableofcontents

\listoffigures

\newpage

\section{}

\subsection{Erl\"autern Sie den Begriff Benutzerauthentifizierung}

Benutzerauthentifizierung beschreibt den formalen Prozess der Verifikation der Nutzeridentität innerhalb 
eines Systems. Dies kann anhand einer der drei Authentifizierungsschemata stattfinden: Wissen, Besitz oder 
Biometrie.

\subsection{Definieren Sie den Begriff biometrische (Benutzer-)Erkennung}
\annotation{Grenzen Sie zu der vorherigen Aufgabe ab.}
Die biometrische Benutzer-Erkennung ist streng genommen ein Teilbereich der Benutzerauthentifizierung. Wie 
oben genannt lässt sich die Benutzerauthentifizierung mittels Wissen, Besitz oder Biometrie durchführen.

Im Gegensatz zu den Schemata Wissen und Besitz fokussiert sich die biometrie-gestützte Erkennung/
Authentifizierung nicht auf Nachweismöglichkeiten die extrinsisch mit einem Individuum verknüpft sind 
(Passwort, Schlüssel, ..) sondern auf intrinsische Nachweismöglichkeiten wie die Physiologie oder das
Verhalten eines Individuums.

Außerhalb der Authentifizierung befasst sich die biometrische Erkennung mit der statistischen Zuordnung 
von Eingabedaten zu einem hinterlegten biometrischen Fingerprint. Dies umfasst z.B. die Auswertung von 
Sprachaufnahmen um regionale Sprachmuster zu erkennen.

\subsection{Erläutern Sie die 11 Merkmale der Bertillonawge}
\annotation{{\"U}berlegen Sie: wie konnte mit den damaligen Mitteln eine Identifikation durchgeführt werden?}
Die 11 Merkmale mit denen damals die Bertillonage durchgeführt wurde lauten: Körpergröße, Spannweite der 
Arme, Sitzhöhe, Kopflänge, Kopfbreite, Länge und Breite des rechten Ohrs, Länge des linken Fußes sowie 
Längen des linken Mittelfingers, des linken kleinen Fingers und des linken Unterarms (Auszug aus dem 
Skript 1.4).

Die Vorgehensweise der Bertillonage ist der der in der DNA-basierten Identifikation recht ähnlich (zumindest 
oberflächlich). In beiden Fällen werden bestimmte Merkmale eines Individuums extrahiert (ie. Marker).
Je mehr Marker zwischen zwei Datensätzen übereinstimmen, desto wahrscheinlicher ist es, dass es sich um 
das gleiche Individuum handelt. Der Abgleich eines einzelnen Individuums ist nicht rechenaufwendig, daher 
war dieses Verfahren bereits im 19. Jahrhundert möglich.

\subsection{Was sind biometrische Charakteristika und welche zwei grundsätzlichen Kategorien gibt es hier?}
\annotation{Klassifizieren Sie biometrische Merkmale in die zwei grundsätzlichen Kategorien und benennen Sie jeweils mindestens vier Beispiele für jede?}	
Es gibt die grundsätzliche Unterteilung zwischen Online- und Offline- Merkmalen. Diese unterscheiden sich
primär hinsichtlich ihres Aufzeichnungszeitpunktes: Online Merkmale können nur während einer Handlung 
aufgenommen werden (z.B. während eine Notiz geschrieben wird) wohingegen Offline-Daten auch im Nachhinein
verfügbar sind (z.B. der beschriebene Notizzettel).

Beispiele für Online-Merkmale sind z.B. Sprachaufnahmen, ein Video von einer handschriftlichen Notiz, das 
Tippverhalten an dem Computer, eine Aufnahme vom Geh-Verhalten. Beispiele für Offline-Merkmale sind z.B. 
Fußstapfen, ein Foto von einer handschriftlichen Notiz, Fingerabdrücke und Körpermaße.

\subsection{Erklären Sie: was verstehen wir unter dem Begriff Lebenderkennung?}

Die Lebenderkennung fokussiert sich primär auf verhaltensbasierte Merkmale, da diese im Gegensatz zu
physiologischen Merkmalen inhärent nur von lebenden Subjekten entnehmen lassen. Diese Merkmale bieten 
somit eine (weitgehende) Sicherheit gegen Spoofing-Attacken.


\end{document}
