\documentclass{beamer}

\title{Open-Source-Lizenzen}
\subtitle{Einordnung und Abgrenzung}


% https://choosealicense.com/licenses/
% https://www.gnu.org/licenses/license-list.de.html#GPLCompatibleLicenses

% https://www.gnu.org/licenses/copyleft
% https://www.ifross.org/?q=urteile

\begin{document}

\maketitle

Aufhänger -> Aktueller Fall

# Definition Open Source

Der begriff "open source software" kurz oss umfasst jegliche quell offene
software. Da software anders als materielle dinge sehr leicht zu
vervielfältigen ist stellt sich schnell die frage einer regelung bzgl.
vervielfältigung / änderung / verkauf von offener software.

# Gesetzes Grundlage für Lizenzen bzw. Nutzungsrechte

https://www.gesetze-im-internet.de/urhg/__69d.html
https://www.gesetze-im-internet.de/urhg/__16.html
https://de.wikipedia.org/wiki/Lizenz

> Für das bloße Ausführen eines Programms im nicht-öffentlichen Rahmen ist keine Lizenz erforderlich, da dies keinem Verbot unterliegt. Eine urheberrechtliche Lizenz, also eine urheberrechtliche Nutzungs-/Verwertungserlaubnis, ist bei urheberrechtlich geschützten Computerprogrammen nur erforderlich, wenn eine urheberrechtlich relevante Nutzungs-/Verwertungshandlung erfolgt, die nicht bereits durch die in § 69d Abs. 1 UrhG verankerte gesetzliche Lizenz erfasst ist. Vor allem aus dem Lager der großen Softwarehersteller wird dies regelmäßig negiert bzw. in Abrede gestellt und hierzu auch gerne versucht, bereits den Lauf eines Computerprogramms als urheberrechtliche Verwertungshandlung erscheinen zu lassen. Ignoriert wird hierbei aber, dass nicht jeder technische Kopiervorgang, wie er definitiv beim Lauf eines Computerprogramms innerhalb eines Computers vieltausendfach erfolgt, auch eine urheberrechtliche Vervielfältigung i. S. d. § 16 UrhG darstellt. Dies schon grundsätzlich deswegen nicht, weil ein rein computerinterner Kopiervorgang nicht zu einem weiteren (zusätzlichen) Werkexemplar führt, das eine zusätzliche Werknutzung ermöglichen würde – wie es etwa beim Herstellen einer Kopie der Programm-CD/DVD oder bei dem Installieren der Software auf einem anderen/weiteren Computer der Fall ist –, sondern nichts daran ändert, dass der Computer von außen betrachtet nur ein einziges Vervielfältigungsexemplar der darauf installierten Software darstellt.[15] Daraus folgt aber auch, dass etwa der Lauf einer von einem zentralen Server oder einem WAN (ASP) bezogenen/gestarteten Software insofern anders bewertet werden muss, als die jeweils vervielfältigten Programmteile Werkcharakter besitzen.
> Ein weiterer Fall ist der, dass ein Werk nicht urheberrechtlich geschützt ist. In diesem Fall ist für keinerlei Nutzungsart eine Lizenz vonnöten. Ein Werk ist dann urheberrechtlich nicht geschützt („gemeinfrei“, „in der Public Domain“), wenn es nicht schutzfähig oder seine Schutzdauer abgelaufen ist. In einigen Rechtssystemen können Urheber auch per Willenserklärung den urheberrechtlichen Schutz aufheben. Nach deutschem Recht ist dies zwar nicht möglich; eine derartige Willenserklärung wird aber in der Rechtsprechung als entsprechend weitreichende Lizenzierung interpretiert.

Copy-Left Effekt:

Copyleft ist eine allgemeine Methode, ein Programm (oder ein anderes Werk) frei
(im Sinne von Freiheit, nicht „Nullpreis“) zu machen und zu verlangen, dass
alle modifizierten und erweiterten Programmversionen ebenfalls frei sind.

Diese Bestimmung gilt nicht für die Weiterverarbeitung von Software die der
Nutzer privat oder intern nutzt. Der Grad des Einflusses von Copyleft variiert
je nach Lizenz.

https://www.gnu.org/licenses/copyleft.de.html

Open Source Lizenzen:

- Strenge Copy Left:
  - GNU GPLv2
  - GNU GPLv3
- Non Copy Left:
  - Apache License v2.0
  - MIT-Lizenz
  - BSD-Lizenz
- Eingeschränktes Copyleft
  - GNU LGPL
  - Mozilla Public License
  - Eclipse Public License
- Nicht Lizensierte Software

Urteile und Rechtsfolgen bei Lizenzverletzung























\begin{frame}
	\Large{GNU AGPLv3}
	Unterschiede v1 bis v3
\end{frame}

\begin{frame}
	\Large{GNU GPLv3}

	Unterschiede v1 bis v3

	GPL-2.0 = Linux
	GPL-2.0 = Git
\end{frame}

\begin{frame}
	\Large{GNU LGPLv3}

	Unterschiede v1 bis v3
\end{frame}

\begin{frame}
	\Large{Mozilla Public License 2.0}

	Unterschiede v1 bis v2
\end{frame}

\begin{frame}
	\Large{Apache License 2.0}

	Unterschiede v1 bis v2
\end{frame}

\begin{frame}
	\Large{MIT License}
\end{frame}

\begin{frame}
	\Large{Boost Software License 1.0}
\end{frame}

\begin{frame}
	\Large{The Unlicense}
\end{frame}

\end{document}
